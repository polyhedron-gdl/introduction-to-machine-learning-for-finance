\documentclass[11pt]{article}

    \usepackage[breakable]{tcolorbox}
    \usepackage{parskip} % Stop auto-indenting (to mimic markdown behaviour)
    
    \usepackage{iftex}
    \ifPDFTeX
    	\usepackage[T1]{fontenc}
    	\usepackage{mathpazo}
    \else
    	\usepackage{fontspec}
    \fi

    % Basic figure setup, for now with no caption control since it's done
    % automatically by Pandoc (which extracts ![](path) syntax from Markdown).
    \usepackage{graphicx}
    % Maintain compatibility with old templates. Remove in nbconvert 6.0
    \let\Oldincludegraphics\includegraphics
    % Ensure that by default, figures have no caption (until we provide a
    % proper Figure object with a Caption API and a way to capture that
    % in the conversion process - todo).
    \usepackage{caption}
    \DeclareCaptionFormat{nocaption}{}
    \captionsetup{format=nocaption,aboveskip=0pt,belowskip=0pt}

    \usepackage{float}
    \floatplacement{figure}{H} % forces figures to be placed at the correct location
    \usepackage{xcolor} % Allow colors to be defined
    \usepackage{enumerate} % Needed for markdown enumerations to work
    \usepackage{geometry} % Used to adjust the document margins
    \usepackage{amsmath} % Equations
    \usepackage{amssymb} % Equations
    \usepackage{textcomp} % defines textquotesingle
    % Hack from http://tex.stackexchange.com/a/47451/13684:
    \AtBeginDocument{%
        \def\PYZsq{\textquotesingle}% Upright quotes in Pygmentized code
    }
    \usepackage{upquote} % Upright quotes for verbatim code
    \usepackage{eurosym} % defines \euro
    \usepackage[mathletters]{ucs} % Extended unicode (utf-8) support
    \usepackage{fancyvrb} % verbatim replacement that allows latex
    \usepackage{grffile} % extends the file name processing of package graphics 
                         % to support a larger range
    \makeatletter % fix for old versions of grffile with XeLaTeX
    \@ifpackagelater{grffile}{2019/11/01}
    {
      % Do nothing on new versions
    }
    {
      \def\Gread@@xetex#1{%
        \IfFileExists{"\Gin@base".bb}%
        {\Gread@eps{\Gin@base.bb}}%
        {\Gread@@xetex@aux#1}%
      }
    }
    \makeatother
    \usepackage[Export]{adjustbox} % Used to constrain images to a maximum size
    \adjustboxset{max size={0.9\linewidth}{0.9\paperheight}}

    % The hyperref package gives us a pdf with properly built
    % internal navigation ('pdf bookmarks' for the table of contents,
    % internal cross-reference links, web links for URLs, etc.)
    \usepackage{hyperref}
    % The default LaTeX title has an obnoxious amount of whitespace. By default,
    % titling removes some of it. It also provides customization options.
    \usepackage{titling}
    \usepackage{longtable} % longtable support required by pandoc >1.10
    \usepackage{booktabs}  % table support for pandoc > 1.12.2
    \usepackage[inline]{enumitem} % IRkernel/repr support (it uses the enumerate* environment)
    \usepackage[normalem]{ulem} % ulem is needed to support strikethroughs (\sout)
                                % normalem makes italics be italics, not underlines
    \usepackage{mathrsfs}
    

    
    % Colors for the hyperref package
    \definecolor{urlcolor}{rgb}{0,.145,.698}
    \definecolor{linkcolor}{rgb}{.71,0.21,0.01}
    \definecolor{citecolor}{rgb}{.12,.54,.11}

    % ANSI colors
    \definecolor{ansi-black}{HTML}{3E424D}
    \definecolor{ansi-black-intense}{HTML}{282C36}
    \definecolor{ansi-red}{HTML}{E75C58}
    \definecolor{ansi-red-intense}{HTML}{B22B31}
    \definecolor{ansi-green}{HTML}{00A250}
    \definecolor{ansi-green-intense}{HTML}{007427}
    \definecolor{ansi-yellow}{HTML}{DDB62B}
    \definecolor{ansi-yellow-intense}{HTML}{B27D12}
    \definecolor{ansi-blue}{HTML}{208FFB}
    \definecolor{ansi-blue-intense}{HTML}{0065CA}
    \definecolor{ansi-magenta}{HTML}{D160C4}
    \definecolor{ansi-magenta-intense}{HTML}{A03196}
    \definecolor{ansi-cyan}{HTML}{60C6C8}
    \definecolor{ansi-cyan-intense}{HTML}{258F8F}
    \definecolor{ansi-white}{HTML}{C5C1B4}
    \definecolor{ansi-white-intense}{HTML}{A1A6B2}
    \definecolor{ansi-default-inverse-fg}{HTML}{FFFFFF}
    \definecolor{ansi-default-inverse-bg}{HTML}{000000}

    % common color for the border for error outputs.
    \definecolor{outerrorbackground}{HTML}{FFDFDF}

    % commands and environments needed by pandoc snippets
    % extracted from the output of `pandoc -s`
    \providecommand{\tightlist}{%
      \setlength{\itemsep}{0pt}\setlength{\parskip}{0pt}}
    \DefineVerbatimEnvironment{Highlighting}{Verbatim}{commandchars=\\\{\}}
    % Add ',fontsize=\small' for more characters per line
    \newenvironment{Shaded}{}{}
    \newcommand{\KeywordTok}[1]{\textcolor[rgb]{0.00,0.44,0.13}{\textbf{{#1}}}}
    \newcommand{\DataTypeTok}[1]{\textcolor[rgb]{0.56,0.13,0.00}{{#1}}}
    \newcommand{\DecValTok}[1]{\textcolor[rgb]{0.25,0.63,0.44}{{#1}}}
    \newcommand{\BaseNTok}[1]{\textcolor[rgb]{0.25,0.63,0.44}{{#1}}}
    \newcommand{\FloatTok}[1]{\textcolor[rgb]{0.25,0.63,0.44}{{#1}}}
    \newcommand{\CharTok}[1]{\textcolor[rgb]{0.25,0.44,0.63}{{#1}}}
    \newcommand{\StringTok}[1]{\textcolor[rgb]{0.25,0.44,0.63}{{#1}}}
    \newcommand{\CommentTok}[1]{\textcolor[rgb]{0.38,0.63,0.69}{\textit{{#1}}}}
    \newcommand{\OtherTok}[1]{\textcolor[rgb]{0.00,0.44,0.13}{{#1}}}
    \newcommand{\AlertTok}[1]{\textcolor[rgb]{1.00,0.00,0.00}{\textbf{{#1}}}}
    \newcommand{\FunctionTok}[1]{\textcolor[rgb]{0.02,0.16,0.49}{{#1}}}
    \newcommand{\RegionMarkerTok}[1]{{#1}}
    \newcommand{\ErrorTok}[1]{\textcolor[rgb]{1.00,0.00,0.00}{\textbf{{#1}}}}
    \newcommand{\NormalTok}[1]{{#1}}
    
    % Additional commands for more recent versions of Pandoc
    \newcommand{\ConstantTok}[1]{\textcolor[rgb]{0.53,0.00,0.00}{{#1}}}
    \newcommand{\SpecialCharTok}[1]{\textcolor[rgb]{0.25,0.44,0.63}{{#1}}}
    \newcommand{\VerbatimStringTok}[1]{\textcolor[rgb]{0.25,0.44,0.63}{{#1}}}
    \newcommand{\SpecialStringTok}[1]{\textcolor[rgb]{0.73,0.40,0.53}{{#1}}}
    \newcommand{\ImportTok}[1]{{#1}}
    \newcommand{\DocumentationTok}[1]{\textcolor[rgb]{0.73,0.13,0.13}{\textit{{#1}}}}
    \newcommand{\AnnotationTok}[1]{\textcolor[rgb]{0.38,0.63,0.69}{\textbf{\textit{{#1}}}}}
    \newcommand{\CommentVarTok}[1]{\textcolor[rgb]{0.38,0.63,0.69}{\textbf{\textit{{#1}}}}}
    \newcommand{\VariableTok}[1]{\textcolor[rgb]{0.10,0.09,0.49}{{#1}}}
    \newcommand{\ControlFlowTok}[1]{\textcolor[rgb]{0.00,0.44,0.13}{\textbf{{#1}}}}
    \newcommand{\OperatorTok}[1]{\textcolor[rgb]{0.40,0.40,0.40}{{#1}}}
    \newcommand{\BuiltInTok}[1]{{#1}}
    \newcommand{\ExtensionTok}[1]{{#1}}
    \newcommand{\PreprocessorTok}[1]{\textcolor[rgb]{0.74,0.48,0.00}{{#1}}}
    \newcommand{\AttributeTok}[1]{\textcolor[rgb]{0.49,0.56,0.16}{{#1}}}
    \newcommand{\InformationTok}[1]{\textcolor[rgb]{0.38,0.63,0.69}{\textbf{\textit{{#1}}}}}
    \newcommand{\WarningTok}[1]{\textcolor[rgb]{0.38,0.63,0.69}{\textbf{\textit{{#1}}}}}
    
    
    % Define a nice break command that doesn't care if a line doesn't already
    % exist.
    \def\br{\hspace*{\fill} \\* }
    % Math Jax compatibility definitions
    \def\gt{>}
    \def\lt{<}
    \let\Oldtex\TeX
    \let\Oldlatex\LaTeX
    \renewcommand{\TeX}{\textrm{\Oldtex}}
    \renewcommand{\LaTeX}{\textrm{\Oldlatex}}
    % Document parameters
    % Document title
    \title{chapter-1-2}
    
    
    
    
    
% Pygments definitions
\makeatletter
\def\PY@reset{\let\PY@it=\relax \let\PY@bf=\relax%
    \let\PY@ul=\relax \let\PY@tc=\relax%
    \let\PY@bc=\relax \let\PY@ff=\relax}
\def\PY@tok#1{\csname PY@tok@#1\endcsname}
\def\PY@toks#1+{\ifx\relax#1\empty\else%
    \PY@tok{#1}\expandafter\PY@toks\fi}
\def\PY@do#1{\PY@bc{\PY@tc{\PY@ul{%
    \PY@it{\PY@bf{\PY@ff{#1}}}}}}}
\def\PY#1#2{\PY@reset\PY@toks#1+\relax+\PY@do{#2}}

\@namedef{PY@tok@w}{\def\PY@tc##1{\textcolor[rgb]{0.73,0.73,0.73}{##1}}}
\@namedef{PY@tok@c}{\let\PY@it=\textit\def\PY@tc##1{\textcolor[rgb]{0.25,0.50,0.50}{##1}}}
\@namedef{PY@tok@cp}{\def\PY@tc##1{\textcolor[rgb]{0.74,0.48,0.00}{##1}}}
\@namedef{PY@tok@k}{\let\PY@bf=\textbf\def\PY@tc##1{\textcolor[rgb]{0.00,0.50,0.00}{##1}}}
\@namedef{PY@tok@kp}{\def\PY@tc##1{\textcolor[rgb]{0.00,0.50,0.00}{##1}}}
\@namedef{PY@tok@kt}{\def\PY@tc##1{\textcolor[rgb]{0.69,0.00,0.25}{##1}}}
\@namedef{PY@tok@o}{\def\PY@tc##1{\textcolor[rgb]{0.40,0.40,0.40}{##1}}}
\@namedef{PY@tok@ow}{\let\PY@bf=\textbf\def\PY@tc##1{\textcolor[rgb]{0.67,0.13,1.00}{##1}}}
\@namedef{PY@tok@nb}{\def\PY@tc##1{\textcolor[rgb]{0.00,0.50,0.00}{##1}}}
\@namedef{PY@tok@nf}{\def\PY@tc##1{\textcolor[rgb]{0.00,0.00,1.00}{##1}}}
\@namedef{PY@tok@nc}{\let\PY@bf=\textbf\def\PY@tc##1{\textcolor[rgb]{0.00,0.00,1.00}{##1}}}
\@namedef{PY@tok@nn}{\let\PY@bf=\textbf\def\PY@tc##1{\textcolor[rgb]{0.00,0.00,1.00}{##1}}}
\@namedef{PY@tok@ne}{\let\PY@bf=\textbf\def\PY@tc##1{\textcolor[rgb]{0.82,0.25,0.23}{##1}}}
\@namedef{PY@tok@nv}{\def\PY@tc##1{\textcolor[rgb]{0.10,0.09,0.49}{##1}}}
\@namedef{PY@tok@no}{\def\PY@tc##1{\textcolor[rgb]{0.53,0.00,0.00}{##1}}}
\@namedef{PY@tok@nl}{\def\PY@tc##1{\textcolor[rgb]{0.63,0.63,0.00}{##1}}}
\@namedef{PY@tok@ni}{\let\PY@bf=\textbf\def\PY@tc##1{\textcolor[rgb]{0.60,0.60,0.60}{##1}}}
\@namedef{PY@tok@na}{\def\PY@tc##1{\textcolor[rgb]{0.49,0.56,0.16}{##1}}}
\@namedef{PY@tok@nt}{\let\PY@bf=\textbf\def\PY@tc##1{\textcolor[rgb]{0.00,0.50,0.00}{##1}}}
\@namedef{PY@tok@nd}{\def\PY@tc##1{\textcolor[rgb]{0.67,0.13,1.00}{##1}}}
\@namedef{PY@tok@s}{\def\PY@tc##1{\textcolor[rgb]{0.73,0.13,0.13}{##1}}}
\@namedef{PY@tok@sd}{\let\PY@it=\textit\def\PY@tc##1{\textcolor[rgb]{0.73,0.13,0.13}{##1}}}
\@namedef{PY@tok@si}{\let\PY@bf=\textbf\def\PY@tc##1{\textcolor[rgb]{0.73,0.40,0.53}{##1}}}
\@namedef{PY@tok@se}{\let\PY@bf=\textbf\def\PY@tc##1{\textcolor[rgb]{0.73,0.40,0.13}{##1}}}
\@namedef{PY@tok@sr}{\def\PY@tc##1{\textcolor[rgb]{0.73,0.40,0.53}{##1}}}
\@namedef{PY@tok@ss}{\def\PY@tc##1{\textcolor[rgb]{0.10,0.09,0.49}{##1}}}
\@namedef{PY@tok@sx}{\def\PY@tc##1{\textcolor[rgb]{0.00,0.50,0.00}{##1}}}
\@namedef{PY@tok@m}{\def\PY@tc##1{\textcolor[rgb]{0.40,0.40,0.40}{##1}}}
\@namedef{PY@tok@gh}{\let\PY@bf=\textbf\def\PY@tc##1{\textcolor[rgb]{0.00,0.00,0.50}{##1}}}
\@namedef{PY@tok@gu}{\let\PY@bf=\textbf\def\PY@tc##1{\textcolor[rgb]{0.50,0.00,0.50}{##1}}}
\@namedef{PY@tok@gd}{\def\PY@tc##1{\textcolor[rgb]{0.63,0.00,0.00}{##1}}}
\@namedef{PY@tok@gi}{\def\PY@tc##1{\textcolor[rgb]{0.00,0.63,0.00}{##1}}}
\@namedef{PY@tok@gr}{\def\PY@tc##1{\textcolor[rgb]{1.00,0.00,0.00}{##1}}}
\@namedef{PY@tok@ge}{\let\PY@it=\textit}
\@namedef{PY@tok@gs}{\let\PY@bf=\textbf}
\@namedef{PY@tok@gp}{\let\PY@bf=\textbf\def\PY@tc##1{\textcolor[rgb]{0.00,0.00,0.50}{##1}}}
\@namedef{PY@tok@go}{\def\PY@tc##1{\textcolor[rgb]{0.53,0.53,0.53}{##1}}}
\@namedef{PY@tok@gt}{\def\PY@tc##1{\textcolor[rgb]{0.00,0.27,0.87}{##1}}}
\@namedef{PY@tok@err}{\def\PY@bc##1{{\setlength{\fboxsep}{\string -\fboxrule}\fcolorbox[rgb]{1.00,0.00,0.00}{1,1,1}{\strut ##1}}}}
\@namedef{PY@tok@kc}{\let\PY@bf=\textbf\def\PY@tc##1{\textcolor[rgb]{0.00,0.50,0.00}{##1}}}
\@namedef{PY@tok@kd}{\let\PY@bf=\textbf\def\PY@tc##1{\textcolor[rgb]{0.00,0.50,0.00}{##1}}}
\@namedef{PY@tok@kn}{\let\PY@bf=\textbf\def\PY@tc##1{\textcolor[rgb]{0.00,0.50,0.00}{##1}}}
\@namedef{PY@tok@kr}{\let\PY@bf=\textbf\def\PY@tc##1{\textcolor[rgb]{0.00,0.50,0.00}{##1}}}
\@namedef{PY@tok@bp}{\def\PY@tc##1{\textcolor[rgb]{0.00,0.50,0.00}{##1}}}
\@namedef{PY@tok@fm}{\def\PY@tc##1{\textcolor[rgb]{0.00,0.00,1.00}{##1}}}
\@namedef{PY@tok@vc}{\def\PY@tc##1{\textcolor[rgb]{0.10,0.09,0.49}{##1}}}
\@namedef{PY@tok@vg}{\def\PY@tc##1{\textcolor[rgb]{0.10,0.09,0.49}{##1}}}
\@namedef{PY@tok@vi}{\def\PY@tc##1{\textcolor[rgb]{0.10,0.09,0.49}{##1}}}
\@namedef{PY@tok@vm}{\def\PY@tc##1{\textcolor[rgb]{0.10,0.09,0.49}{##1}}}
\@namedef{PY@tok@sa}{\def\PY@tc##1{\textcolor[rgb]{0.73,0.13,0.13}{##1}}}
\@namedef{PY@tok@sb}{\def\PY@tc##1{\textcolor[rgb]{0.73,0.13,0.13}{##1}}}
\@namedef{PY@tok@sc}{\def\PY@tc##1{\textcolor[rgb]{0.73,0.13,0.13}{##1}}}
\@namedef{PY@tok@dl}{\def\PY@tc##1{\textcolor[rgb]{0.73,0.13,0.13}{##1}}}
\@namedef{PY@tok@s2}{\def\PY@tc##1{\textcolor[rgb]{0.73,0.13,0.13}{##1}}}
\@namedef{PY@tok@sh}{\def\PY@tc##1{\textcolor[rgb]{0.73,0.13,0.13}{##1}}}
\@namedef{PY@tok@s1}{\def\PY@tc##1{\textcolor[rgb]{0.73,0.13,0.13}{##1}}}
\@namedef{PY@tok@mb}{\def\PY@tc##1{\textcolor[rgb]{0.40,0.40,0.40}{##1}}}
\@namedef{PY@tok@mf}{\def\PY@tc##1{\textcolor[rgb]{0.40,0.40,0.40}{##1}}}
\@namedef{PY@tok@mh}{\def\PY@tc##1{\textcolor[rgb]{0.40,0.40,0.40}{##1}}}
\@namedef{PY@tok@mi}{\def\PY@tc##1{\textcolor[rgb]{0.40,0.40,0.40}{##1}}}
\@namedef{PY@tok@il}{\def\PY@tc##1{\textcolor[rgb]{0.40,0.40,0.40}{##1}}}
\@namedef{PY@tok@mo}{\def\PY@tc##1{\textcolor[rgb]{0.40,0.40,0.40}{##1}}}
\@namedef{PY@tok@ch}{\let\PY@it=\textit\def\PY@tc##1{\textcolor[rgb]{0.25,0.50,0.50}{##1}}}
\@namedef{PY@tok@cm}{\let\PY@it=\textit\def\PY@tc##1{\textcolor[rgb]{0.25,0.50,0.50}{##1}}}
\@namedef{PY@tok@cpf}{\let\PY@it=\textit\def\PY@tc##1{\textcolor[rgb]{0.25,0.50,0.50}{##1}}}
\@namedef{PY@tok@c1}{\let\PY@it=\textit\def\PY@tc##1{\textcolor[rgb]{0.25,0.50,0.50}{##1}}}
\@namedef{PY@tok@cs}{\let\PY@it=\textit\def\PY@tc##1{\textcolor[rgb]{0.25,0.50,0.50}{##1}}}

\def\PYZbs{\char`\\}
\def\PYZus{\char`\_}
\def\PYZob{\char`\{}
\def\PYZcb{\char`\}}
\def\PYZca{\char`\^}
\def\PYZam{\char`\&}
\def\PYZlt{\char`\<}
\def\PYZgt{\char`\>}
\def\PYZsh{\char`\#}
\def\PYZpc{\char`\%}
\def\PYZdl{\char`\$}
\def\PYZhy{\char`\-}
\def\PYZsq{\char`\'}
\def\PYZdq{\char`\"}
\def\PYZti{\char`\~}
% for compatibility with earlier versions
\def\PYZat{@}
\def\PYZlb{[}
\def\PYZrb{]}
\makeatother


    % For linebreaks inside Verbatim environment from package fancyvrb. 
    \makeatletter
        \newbox\Wrappedcontinuationbox 
        \newbox\Wrappedvisiblespacebox 
        \newcommand*\Wrappedvisiblespace {\textcolor{red}{\textvisiblespace}} 
        \newcommand*\Wrappedcontinuationsymbol {\textcolor{red}{\llap{\tiny$\m@th\hookrightarrow$}}} 
        \newcommand*\Wrappedcontinuationindent {3ex } 
        \newcommand*\Wrappedafterbreak {\kern\Wrappedcontinuationindent\copy\Wrappedcontinuationbox} 
        % Take advantage of the already applied Pygments mark-up to insert 
        % potential linebreaks for TeX processing. 
        %        {, <, #, %, $, ' and ": go to next line. 
        %        _, }, ^, &, >, - and ~: stay at end of broken line. 
        % Use of \textquotesingle for straight quote. 
        \newcommand*\Wrappedbreaksatspecials {% 
            \def\PYGZus{\discretionary{\char`\_}{\Wrappedafterbreak}{\char`\_}}% 
            \def\PYGZob{\discretionary{}{\Wrappedafterbreak\char`\{}{\char`\{}}% 
            \def\PYGZcb{\discretionary{\char`\}}{\Wrappedafterbreak}{\char`\}}}% 
            \def\PYGZca{\discretionary{\char`\^}{\Wrappedafterbreak}{\char`\^}}% 
            \def\PYGZam{\discretionary{\char`\&}{\Wrappedafterbreak}{\char`\&}}% 
            \def\PYGZlt{\discretionary{}{\Wrappedafterbreak\char`\<}{\char`\<}}% 
            \def\PYGZgt{\discretionary{\char`\>}{\Wrappedafterbreak}{\char`\>}}% 
            \def\PYGZsh{\discretionary{}{\Wrappedafterbreak\char`\#}{\char`\#}}% 
            \def\PYGZpc{\discretionary{}{\Wrappedafterbreak\char`\%}{\char`\%}}% 
            \def\PYGZdl{\discretionary{}{\Wrappedafterbreak\char`\$}{\char`\$}}% 
            \def\PYGZhy{\discretionary{\char`\-}{\Wrappedafterbreak}{\char`\-}}% 
            \def\PYGZsq{\discretionary{}{\Wrappedafterbreak\textquotesingle}{\textquotesingle}}% 
            \def\PYGZdq{\discretionary{}{\Wrappedafterbreak\char`\"}{\char`\"}}% 
            \def\PYGZti{\discretionary{\char`\~}{\Wrappedafterbreak}{\char`\~}}% 
        } 
        % Some characters . , ; ? ! / are not pygmentized. 
        % This macro makes them "active" and they will insert potential linebreaks 
        \newcommand*\Wrappedbreaksatpunct {% 
            \lccode`\~`\.\lowercase{\def~}{\discretionary{\hbox{\char`\.}}{\Wrappedafterbreak}{\hbox{\char`\.}}}% 
            \lccode`\~`\,\lowercase{\def~}{\discretionary{\hbox{\char`\,}}{\Wrappedafterbreak}{\hbox{\char`\,}}}% 
            \lccode`\~`\;\lowercase{\def~}{\discretionary{\hbox{\char`\;}}{\Wrappedafterbreak}{\hbox{\char`\;}}}% 
            \lccode`\~`\:\lowercase{\def~}{\discretionary{\hbox{\char`\:}}{\Wrappedafterbreak}{\hbox{\char`\:}}}% 
            \lccode`\~`\?\lowercase{\def~}{\discretionary{\hbox{\char`\?}}{\Wrappedafterbreak}{\hbox{\char`\?}}}% 
            \lccode`\~`\!\lowercase{\def~}{\discretionary{\hbox{\char`\!}}{\Wrappedafterbreak}{\hbox{\char`\!}}}% 
            \lccode`\~`\/\lowercase{\def~}{\discretionary{\hbox{\char`\/}}{\Wrappedafterbreak}{\hbox{\char`\/}}}% 
            \catcode`\.\active
            \catcode`\,\active 
            \catcode`\;\active
            \catcode`\:\active
            \catcode`\?\active
            \catcode`\!\active
            \catcode`\/\active 
            \lccode`\~`\~ 	
        }
    \makeatother

    \let\OriginalVerbatim=\Verbatim
    \makeatletter
    \renewcommand{\Verbatim}[1][1]{%
        %\parskip\z@skip
        \sbox\Wrappedcontinuationbox {\Wrappedcontinuationsymbol}%
        \sbox\Wrappedvisiblespacebox {\FV@SetupFont\Wrappedvisiblespace}%
        \def\FancyVerbFormatLine ##1{\hsize\linewidth
            \vtop{\raggedright\hyphenpenalty\z@\exhyphenpenalty\z@
                \doublehyphendemerits\z@\finalhyphendemerits\z@
                \strut ##1\strut}%
        }%
        % If the linebreak is at a space, the latter will be displayed as visible
        % space at end of first line, and a continuation symbol starts next line.
        % Stretch/shrink are however usually zero for typewriter font.
        \def\FV@Space {%
            \nobreak\hskip\z@ plus\fontdimen3\font minus\fontdimen4\font
            \discretionary{\copy\Wrappedvisiblespacebox}{\Wrappedafterbreak}
            {\kern\fontdimen2\font}%
        }%
        
        % Allow breaks at special characters using \PYG... macros.
        \Wrappedbreaksatspecials
        % Breaks at punctuation characters . , ; ? ! and / need catcode=\active 	
        \OriginalVerbatim[#1,codes*=\Wrappedbreaksatpunct]%
    }
    \makeatother

    % Exact colors from NB
    \definecolor{incolor}{HTML}{303F9F}
    \definecolor{outcolor}{HTML}{D84315}
    \definecolor{cellborder}{HTML}{CFCFCF}
    \definecolor{cellbackground}{HTML}{F7F7F7}
    
    % prompt
    \makeatletter
    \newcommand{\boxspacing}{\kern\kvtcb@left@rule\kern\kvtcb@boxsep}
    \makeatother
    \newcommand{\prompt}[4]{
        {\ttfamily\llap{{\color{#2}[#3]:\hspace{3pt}#4}}\vspace{-\baselineskip}}
    }
    

    
    % Prevent overflowing lines due to hard-to-break entities
    \sloppy 
    % Setup hyperref package
    \hypersetup{
      breaklinks=true,  % so long urls are correctly broken across lines
      colorlinks=true,
      urlcolor=urlcolor,
      linkcolor=linkcolor,
      citecolor=citecolor,
      }
    % Slightly bigger margins than the latex defaults
    
    \geometry{verbose,tmargin=1in,bmargin=1in,lmargin=1in,rmargin=1in}
    
    

\begin{document}
    
    \maketitle
    
    

    
    Run in Google Colab

    \hypertarget{python-library-for-data-science-a-quick-glance}{%
\section{Python Library for Data Science: A Quick
Glance}\label{python-library-for-data-science-a-quick-glance}}

    \hypertarget{introduction}{%
\subsection{Introduction}\label{introduction}}

In this lesson we will present the main scientific computation libraries
in python used in data analysis. These are the most important libraries
of a general nature for analyzing data in Python. In particular we will
focus on:

\begin{itemize}
\item
  \textbf{Numpy} Numpy is a Python library with math functionalities. It
  allows us to work with multi-dimensional arrays, matrices, generate
  random numbers, linear algebra routines, and more.
\item
  \textbf{Matplotlib/Seaborn} Matplotlib is a library that allows us to
  make basic plots, while Seaborn specializes in statistics
  visualization. The main difference is in the lines of code you need to
  write to create a plot. Seaborn is easier to learn, has default
  themes, and makes better-looking plots than Matplotlib by default.
\item
  \textbf{Pandas} Pandas is a powerful tool that offers a variety of
  ways to manipulate and clean data. Pandas work with dataframes that
  structures data in a table similar to an Excel spreadsheet, but faster
  and with all the power of Python.
\end{itemize}

We will reserve a specific study in the course of the other lessons on
two extremely important libraries for machine learning applications:
scikit-learn and keras.

    \hypertarget{basic-numpy}{%
\subsection{Basic NumPy}\label{basic-numpy}}

Numpy (it stands for Numerical Python) is the core library for
scientific computing in Python. It provides a high-performance
multidimensional array object, and tools for working with these arrays.
If you are already familiar with MATLAB, you might find this
\href{http://wiki.scipy.org/NumPy_for_Matlab_Users}{tutorial} useful to
get started with Numpy. NumPy helps to create arrays (multidimensional
arrays), with the help of bindings of C++. Therefore, it is quite fast.
There are in-built functions of NumPy as well. It is the fundamental
package for scientific computing with Python.

Numpy arrays are collections of things, all of which must be the same
type, that work similarly to lists (as we've described them so far). The
most important are:

\begin{enumerate}
\def\labelenumi{\arabic{enumi}.}
\tightlist
\item
  You can easily perform elementwise operations (and matrix algebra) on
  arrays
\item
  Arrays can be n-dimensional
\item
  There is no equivalent to append, although arrays can be concatenated
\end{enumerate}

As we shall see, arrays can be created from existing collections such as
lists, or instantiated ``from scratch'' in a few useful ways.

    \begin{tcolorbox}[breakable, size=fbox, boxrule=1pt, pad at break*=1mm,colback=cellbackground, colframe=cellborder]
\prompt{In}{incolor}{5}{\boxspacing}
\begin{Verbatim}[commandchars=\\\{\}]
\PY{c+c1}{\PYZsh{} We need to import the numpy library to have access to it }
\PY{c+c1}{\PYZsh{} We can also create an alias for a library, this is something you will commonly see with numpy}
\PY{k+kn}{import} \PY{n+nn}{numpy} \PY{k}{as} \PY{n+nn}{np}
\end{Verbatim}
\end{tcolorbox}

    \hypertarget{why-do-we-need-numpy}{%
\subsubsection{Why do we need NumPy?}\label{why-do-we-need-numpy}}

Does a question arise that why do we need a NumPy array when we have
python lists? The answer is we can perform operations on all the
elements of a NumPy array at once, which are not possible with python
lists. For example, we can't multiply two lists directly we will have to
do it element-wise. This is where the role of NumPy comes into play.

    \begin{tcolorbox}[breakable, size=fbox, boxrule=1pt, pad at break*=1mm,colback=cellbackground, colframe=cellborder]
\prompt{In}{incolor}{6}{\boxspacing}
\begin{Verbatim}[commandchars=\\\{\}]
\PY{n}{list1} \PY{o}{=} \PY{p}{[}\PY{l+m+mi}{2}\PY{p}{,} \PY{l+m+mi}{4}\PY{p}{,} \PY{l+m+mi}{6}\PY{p}{,} \PY{l+m+mi}{7}\PY{p}{,} \PY{l+m+mi}{8}\PY{p}{]}
\PY{n}{list2} \PY{o}{=} \PY{p}{[}\PY{l+m+mi}{3}\PY{p}{,} \PY{l+m+mi}{4}\PY{p}{,} \PY{l+m+mi}{6}\PY{p}{,} \PY{l+m+mi}{1}\PY{p}{,} \PY{l+m+mi}{5}\PY{p}{]}

\PY{n+nb}{print}\PY{p}{(}\PY{n}{list1}\PY{o}{*}\PY{n}{list2}\PY{p}{)}
\end{Verbatim}
\end{tcolorbox}

    \begin{Verbatim}[commandchars=\\\{\}, frame=single, framerule=2mm, rulecolor=\color{outerrorbackground}]
\textcolor{ansi-red-intense}{\textbf{---------------------------------------------------------------------------}}
\textcolor{ansi-red-intense}{\textbf{TypeError}}                                 Traceback (most recent call last)
\textcolor{ansi-green-intense}{\textbf{<ipython-input-6-0193bd45e9db>}} in \textcolor{ansi-cyan}{<module>}
\textcolor{ansi-green}{      2} list2 \textcolor{ansi-yellow-intense}{\textbf{=}} \textcolor{ansi-yellow-intense}{\textbf{[}}\textcolor{ansi-cyan-intense}{\textbf{3}}\textcolor{ansi-yellow-intense}{\textbf{,}} \textcolor{ansi-cyan-intense}{\textbf{4}}\textcolor{ansi-yellow-intense}{\textbf{,}} \textcolor{ansi-cyan-intense}{\textbf{6}}\textcolor{ansi-yellow-intense}{\textbf{,}} \textcolor{ansi-cyan-intense}{\textbf{1}}\textcolor{ansi-yellow-intense}{\textbf{,}} \textcolor{ansi-cyan-intense}{\textbf{5}}\textcolor{ansi-yellow-intense}{\textbf{]}}
\textcolor{ansi-green}{      3} 
\textcolor{ansi-green-intense}{\textbf{----> 4}}\textcolor{ansi-yellow-intense}{\textbf{ }}print\textcolor{ansi-yellow-intense}{\textbf{(}}list1\textcolor{ansi-yellow-intense}{\textbf{*}}list2\textcolor{ansi-yellow-intense}{\textbf{)}}

\textcolor{ansi-red-intense}{\textbf{TypeError}}: can't multiply sequence by non-int of type 'list'
    \end{Verbatim}

    \begin{tcolorbox}[breakable, size=fbox, boxrule=1pt, pad at break*=1mm,colback=cellbackground, colframe=cellborder]
\prompt{In}{incolor}{7}{\boxspacing}
\begin{Verbatim}[commandchars=\\\{\}]
\PY{k+kn}{import} \PY{n+nn}{numpy} \PY{k}{as} \PY{n+nn}{np}

\PY{n}{list1} \PY{o}{=} \PY{p}{[}\PY{l+m+mi}{2}\PY{p}{,} \PY{l+m+mi}{4}\PY{p}{,} \PY{l+m+mi}{6}\PY{p}{,} \PY{l+m+mi}{7}\PY{p}{,} \PY{l+m+mi}{8}\PY{p}{]}
\PY{n}{list2} \PY{o}{=} \PY{p}{[}\PY{l+m+mi}{3}\PY{p}{,} \PY{l+m+mi}{4}\PY{p}{,} \PY{l+m+mi}{6}\PY{p}{,} \PY{l+m+mi}{1}\PY{p}{,} \PY{l+m+mi}{5}\PY{p}{]}

\PY{n}{arr1} \PY{o}{=} \PY{n}{np}\PY{o}{.}\PY{n}{array}\PY{p}{(}\PY{n}{list1}\PY{p}{)}
\PY{n}{arr2} \PY{o}{=} \PY{n}{np}\PY{o}{.}\PY{n}{array}\PY{p}{(}\PY{n}{list2}\PY{p}{)}

\PY{n+nb}{print}\PY{p}{(}\PY{n}{arr1}\PY{o}{*}\PY{n}{arr2}\PY{p}{)}
\end{Verbatim}
\end{tcolorbox}

    \begin{Verbatim}[commandchars=\\\{\}]
[ 6 16 36  7 40]
    \end{Verbatim}

    \hypertarget{arrays}{%
\subsubsection{Arrays}\label{arrays}}

    A numpy array is a grid of values, all of the same type, and is indexed
by a tuple of nonnegative integers. \textbf{\emph{The number of
dimensions is the rank of the array}}; the \textbf{\emph{shape}} of an
array is a tuple of integers giving \textbf{\emph{the size of the array
along each dimension}}.

    We can initialize numpy arrays from nested Python lists, and access
elements using square brackets:

    \begin{tcolorbox}[breakable, size=fbox, boxrule=1pt, pad at break*=1mm,colback=cellbackground, colframe=cellborder]
\prompt{In}{incolor}{8}{\boxspacing}
\begin{Verbatim}[commandchars=\\\{\}]
\PY{n}{a} \PY{o}{=} \PY{n}{np}\PY{o}{.}\PY{n}{array}\PY{p}{(}\PY{p}{[}\PY{l+m+mi}{1}\PY{p}{,} \PY{l+m+mi}{2}\PY{p}{,} \PY{l+m+mi}{3}\PY{p}{]}\PY{p}{)}  \PY{c+c1}{\PYZsh{} Create a rank 1 array}
\PY{n+nb}{print}\PY{p}{(}\PY{n+nb}{type}\PY{p}{(}\PY{n}{a}\PY{p}{)}\PY{p}{,} \PY{n}{a}\PY{o}{.}\PY{n}{shape}\PY{p}{,} \PY{n}{a}\PY{p}{[}\PY{l+m+mi}{0}\PY{p}{]}\PY{p}{,} \PY{n}{a}\PY{p}{[}\PY{l+m+mi}{1}\PY{p}{]}\PY{p}{,} \PY{n}{a}\PY{p}{[}\PY{l+m+mi}{2}\PY{p}{]}\PY{p}{)}
\PY{n}{a}\PY{p}{[}\PY{l+m+mi}{0}\PY{p}{]} \PY{o}{=} \PY{l+m+mi}{5}                 \PY{c+c1}{\PYZsh{} Change an element of the array}
\PY{n+nb}{print}\PY{p}{(}\PY{n}{a}\PY{p}{)}                  
\end{Verbatim}
\end{tcolorbox}

    \begin{Verbatim}[commandchars=\\\{\}]
<class 'numpy.ndarray'> (3,) 1 2 3
[5 2 3]
    \end{Verbatim}

    \begin{tcolorbox}[breakable, size=fbox, boxrule=1pt, pad at break*=1mm,colback=cellbackground, colframe=cellborder]
\prompt{In}{incolor}{9}{\boxspacing}
\begin{Verbatim}[commandchars=\\\{\}]
\PY{n}{b} \PY{o}{=} \PY{n}{np}\PY{o}{.}\PY{n}{array}\PY{p}{(}\PY{p}{[}\PY{p}{[}\PY{l+m+mi}{1}\PY{p}{,}\PY{l+m+mi}{2}\PY{p}{,}\PY{l+m+mi}{3}\PY{p}{]}\PY{p}{,}\PY{p}{[}\PY{l+m+mi}{4}\PY{p}{,}\PY{l+m+mi}{5}\PY{p}{,}\PY{l+m+mi}{6}\PY{p}{]}\PY{p}{]}\PY{p}{)}   \PY{c+c1}{\PYZsh{} Create a rank 2 array}
\PY{n+nb}{print}\PY{p}{(}\PY{n}{b}\PY{p}{)}
\PY{n+nb}{print}\PY{p}{(}\PY{l+s+s1}{\PYZsq{}}\PY{l+s+s1}{The dimension of b is : }\PY{l+s+s1}{\PYZsq{}} \PY{o}{+} \PY{n+nb}{str}\PY{p}{(}\PY{n}{b}\PY{o}{.}\PY{n}{ndim}\PY{p}{)}\PY{p}{)}
\end{Verbatim}
\end{tcolorbox}

    \begin{Verbatim}[commandchars=\\\{\}]
[[1 2 3]
 [4 5 6]]
The dimension of b is : 2
    \end{Verbatim}

    \begin{tcolorbox}[breakable, size=fbox, boxrule=1pt, pad at break*=1mm,colback=cellbackground, colframe=cellborder]
\prompt{In}{incolor}{10}{\boxspacing}
\begin{Verbatim}[commandchars=\\\{\}]
\PY{n+nb}{print}\PY{p}{(}\PY{n}{b}\PY{o}{.}\PY{n}{shape}\PY{p}{)}
\PY{n+nb}{print}\PY{p}{(}\PY{n}{b}\PY{p}{[}\PY{l+m+mi}{0}\PY{p}{,} \PY{l+m+mi}{0}\PY{p}{]}\PY{p}{,} \PY{n}{b}\PY{p}{[}\PY{l+m+mi}{0}\PY{p}{,} \PY{l+m+mi}{1}\PY{p}{]}\PY{p}{,} \PY{n}{b}\PY{p}{[}\PY{l+m+mi}{1}\PY{p}{,} \PY{l+m+mi}{0}\PY{p}{]}\PY{p}{)}
\end{Verbatim}
\end{tcolorbox}

    \begin{Verbatim}[commandchars=\\\{\}]
(2, 3)
1 2 4
    \end{Verbatim}

    Numpy also provides many functions to create arrays:

    \begin{tcolorbox}[breakable, size=fbox, boxrule=1pt, pad at break*=1mm,colback=cellbackground, colframe=cellborder]
\prompt{In}{incolor}{11}{\boxspacing}
\begin{Verbatim}[commandchars=\\\{\}]
\PY{n}{a} \PY{o}{=} \PY{n}{np}\PY{o}{.}\PY{n}{zeros}\PY{p}{(}\PY{p}{(}\PY{l+m+mi}{2}\PY{p}{,}\PY{l+m+mi}{2}\PY{p}{)}\PY{p}{)}  \PY{c+c1}{\PYZsh{} Create an array of all zeros}
\PY{n+nb}{print}\PY{p}{(}\PY{n}{a}\PY{p}{)}

\PY{n}{b} \PY{o}{=} \PY{n}{np}\PY{o}{.}\PY{n}{ones}\PY{p}{(}\PY{p}{(}\PY{l+m+mi}{1}\PY{p}{,}\PY{l+m+mi}{2}\PY{p}{)}\PY{p}{)}   \PY{c+c1}{\PYZsh{} Create an array of all ones}
\PY{n+nb}{print}\PY{p}{(}\PY{n}{b}\PY{p}{)}

\PY{n}{c} \PY{o}{=} \PY{n}{np}\PY{o}{.}\PY{n}{full}\PY{p}{(}\PY{p}{(}\PY{l+m+mi}{2}\PY{p}{,}\PY{l+m+mi}{2}\PY{p}{)}\PY{p}{,} \PY{l+m+mi}{7}\PY{p}{)} \PY{c+c1}{\PYZsh{} Create a constant array}
\PY{n+nb}{print}\PY{p}{(}\PY{n}{c}\PY{p}{)}

\PY{n}{d} \PY{o}{=} \PY{n}{np}\PY{o}{.}\PY{n}{eye}\PY{p}{(}\PY{l+m+mi}{2}\PY{p}{)}        \PY{c+c1}{\PYZsh{} Create a 2x2 identity matrix}
\PY{n+nb}{print}\PY{p}{(}\PY{n}{d}\PY{p}{)}

\PY{n}{e} \PY{o}{=} \PY{n}{np}\PY{o}{.}\PY{n}{random}\PY{o}{.}\PY{n}{random}\PY{p}{(}\PY{p}{(}\PY{l+m+mi}{2}\PY{p}{,}\PY{l+m+mi}{2}\PY{p}{)}\PY{p}{)} \PY{c+c1}{\PYZsh{} Create an array filled with random values}
\PY{n+nb}{print}\PY{p}{(}\PY{n}{e}\PY{p}{)}
\end{Verbatim}
\end{tcolorbox}

    \begin{Verbatim}[commandchars=\\\{\}]
[[0. 0.]
 [0. 0.]]
[[1. 1.]]
[[7 7]
 [7 7]]
[[1. 0.]
 [0. 1.]]
[[0.81289678 0.98905384]
 [0.02005508 0.52106272]]
    \end{Verbatim}

    \begin{tcolorbox}[breakable, size=fbox, boxrule=1pt, pad at break*=1mm,colback=cellbackground, colframe=cellborder]
\prompt{In}{incolor}{12}{\boxspacing}
\begin{Verbatim}[commandchars=\\\{\}]
\PY{c+c1}{\PYZsh{} Make an array from a list}
\PY{n}{alist} \PY{o}{=} \PY{p}{[}\PY{l+m+mi}{2}\PY{p}{,} \PY{l+m+mi}{3}\PY{p}{,} \PY{l+m+mi}{4}\PY{p}{]}
\PY{n}{blist} \PY{o}{=} \PY{p}{[}\PY{l+m+mi}{5}\PY{p}{,} \PY{l+m+mi}{6}\PY{p}{,} \PY{l+m+mi}{7}\PY{p}{]}
\PY{n}{a} \PY{o}{=} \PY{n}{np}\PY{o}{.}\PY{n}{array}\PY{p}{(}\PY{n}{alist}\PY{p}{)}
\PY{n}{b} \PY{o}{=} \PY{n}{np}\PY{o}{.}\PY{n}{array}\PY{p}{(}\PY{n}{blist}\PY{p}{)}
\PY{n+nb}{print}\PY{p}{(}\PY{n}{a}\PY{p}{,} \PY{n+nb}{type}\PY{p}{(}\PY{n}{a}\PY{p}{)}\PY{p}{)}
\PY{n+nb}{print}\PY{p}{(}\PY{n}{b}\PY{p}{,} \PY{n+nb}{type}\PY{p}{(}\PY{n}{b}\PY{p}{)}\PY{p}{)}
\end{Verbatim}
\end{tcolorbox}

    \begin{Verbatim}[commandchars=\\\{\}]
[2 3 4] <class 'numpy.ndarray'>
[5 6 7] <class 'numpy.ndarray'>
    \end{Verbatim}

    \hypertarget{array-indexing}{%
\subsubsection{Array Indexing}\label{array-indexing}}

    Numpy offers several ways to index into arrays.

    Slicing: Similar to Python lists, numpy arrays can be sliced. Since
arrays may be multidimensional, you must specify a slice for each
dimension of the array:

    \begin{tcolorbox}[breakable, size=fbox, boxrule=1pt, pad at break*=1mm,colback=cellbackground, colframe=cellborder]
\prompt{In}{incolor}{13}{\boxspacing}
\begin{Verbatim}[commandchars=\\\{\}]
\PY{k+kn}{import} \PY{n+nn}{numpy} \PY{k}{as} \PY{n+nn}{np}

\PY{c+c1}{\PYZsh{} Create the following rank 2 array with shape (3, 4)}
\PY{c+c1}{\PYZsh{} [[ 1  2  3  4]}
\PY{c+c1}{\PYZsh{}  [ 5  6  7  8]}
\PY{c+c1}{\PYZsh{}  [ 9 10 11 12]]}
\PY{n}{a} \PY{o}{=} \PY{n}{np}\PY{o}{.}\PY{n}{array}\PY{p}{(}\PY{p}{[}\PY{p}{[}\PY{l+m+mi}{1}\PY{p}{,}\PY{l+m+mi}{2}\PY{p}{,}\PY{l+m+mi}{3}\PY{p}{,}\PY{l+m+mi}{4}\PY{p}{]}\PY{p}{,} \PY{p}{[}\PY{l+m+mi}{5}\PY{p}{,}\PY{l+m+mi}{6}\PY{p}{,}\PY{l+m+mi}{7}\PY{p}{,}\PY{l+m+mi}{8}\PY{p}{]}\PY{p}{,} \PY{p}{[}\PY{l+m+mi}{9}\PY{p}{,}\PY{l+m+mi}{10}\PY{p}{,}\PY{l+m+mi}{11}\PY{p}{,}\PY{l+m+mi}{12}\PY{p}{]}\PY{p}{]}\PY{p}{)}
\PY{n+nb}{print}\PY{p}{(}\PY{n}{a}\PY{p}{)}
\end{Verbatim}
\end{tcolorbox}

    \begin{Verbatim}[commandchars=\\\{\}]
[[ 1  2  3  4]
 [ 5  6  7  8]
 [ 9 10 11 12]]
    \end{Verbatim}

    \begin{tcolorbox}[breakable, size=fbox, boxrule=1pt, pad at break*=1mm,colback=cellbackground, colframe=cellborder]
\prompt{In}{incolor}{14}{\boxspacing}
\begin{Verbatim}[commandchars=\\\{\}]
\PY{n+nb}{print}\PY{p}{(}\PY{n}{a}\PY{o}{.}\PY{n}{shape}\PY{p}{)}
\end{Verbatim}
\end{tcolorbox}

    \begin{Verbatim}[commandchars=\\\{\}]
(3, 4)
    \end{Verbatim}

    \begin{tcolorbox}[breakable, size=fbox, boxrule=1pt, pad at break*=1mm,colback=cellbackground, colframe=cellborder]
\prompt{In}{incolor}{15}{\boxspacing}
\begin{Verbatim}[commandchars=\\\{\}]
\PY{c+c1}{\PYZsh{} Use slicing to pull out the subarray consisting of the first 2 rows}
\PY{c+c1}{\PYZsh{} and columns 1 and 2; b is the following array of shape (2, 2):}
\PY{c+c1}{\PYZsh{} [[2 3]}
\PY{c+c1}{\PYZsh{}  [6 7]]}
\PY{n}{b} \PY{o}{=} \PY{n}{a}\PY{p}{[}\PY{p}{:}\PY{l+m+mi}{2}\PY{p}{,} \PY{l+m+mi}{1}\PY{p}{:}\PY{l+m+mi}{3}\PY{p}{]}
\PY{n+nb}{print}\PY{p}{(}\PY{n}{b}\PY{p}{)}
\end{Verbatim}
\end{tcolorbox}

    \begin{Verbatim}[commandchars=\\\{\}]
[[2 3]
 [6 7]]
    \end{Verbatim}

    \textbf{IMPORTANT} : \textbf{\emph{A slice of an array is a view into
the same data, so modifying it will modify the original array}}.

    \begin{tcolorbox}[breakable, size=fbox, boxrule=1pt, pad at break*=1mm,colback=cellbackground, colframe=cellborder]
\prompt{In}{incolor}{16}{\boxspacing}
\begin{Verbatim}[commandchars=\\\{\}]
\PY{n}{a} \PY{o}{=} \PY{n}{np}\PY{o}{.}\PY{n}{array}\PY{p}{(}\PY{p}{[}\PY{p}{[}\PY{l+m+mi}{1}\PY{p}{,}\PY{l+m+mi}{2}\PY{p}{,}\PY{l+m+mi}{3}\PY{p}{,}\PY{l+m+mi}{4}\PY{p}{]}\PY{p}{,} \PY{p}{[}\PY{l+m+mi}{5}\PY{p}{,}\PY{l+m+mi}{6}\PY{p}{,}\PY{l+m+mi}{7}\PY{p}{,}\PY{l+m+mi}{8}\PY{p}{]}\PY{p}{,} \PY{p}{[}\PY{l+m+mi}{9}\PY{p}{,}\PY{l+m+mi}{10}\PY{p}{,}\PY{l+m+mi}{11}\PY{p}{,}\PY{l+m+mi}{12}\PY{p}{]}\PY{p}{]}\PY{p}{)}
\PY{n}{b} \PY{o}{=} \PY{n}{a}\PY{p}{[}\PY{p}{:}\PY{l+m+mi}{2}\PY{p}{,} \PY{l+m+mi}{1}\PY{p}{:}\PY{l+m+mi}{3}\PY{p}{]}
\PY{c+c1}{\PYZsh{}}
\PY{n+nb}{print}\PY{p}{(}\PY{l+s+s2}{\PYZdq{}}\PY{l+s+se}{\PYZbs{}n}\PY{l+s+s2}{\PYZsq{}}\PY{l+s+s2}{a}\PY{l+s+s2}{\PYZsq{}}\PY{l+s+s2}{ matrix before slicing}\PY{l+s+se}{\PYZbs{}n}\PY{l+s+s2}{\PYZdq{}}\PY{p}{)}
\PY{n+nb}{print}\PY{p}{(}\PY{n}{a}\PY{p}{)}
\PY{c+c1}{\PYZsh{} }
\PY{c+c1}{\PYZsh{} BEWARE: b[0, 0] is the same piece of data as a[0, 1] !!!}
\PY{c+c1}{\PYZsh{}}
\PY{n}{b}\PY{p}{[}\PY{l+m+mi}{0}\PY{p}{,} \PY{l+m+mi}{0}\PY{p}{]} \PY{o}{=} \PY{l+m+mi}{77}
\PY{c+c1}{\PYZsh{}}
\PY{n+nb}{print}\PY{p}{(}\PY{l+s+s1}{\PYZsq{}}\PY{l+s+se}{\PYZbs{}n}\PY{l+s+s1}{\PYZsq{}}\PY{o}{+} \PY{l+m+mi}{100}\PY{o}{*}\PY{l+s+s1}{\PYZsq{}}\PY{l+s+s1}{\PYZhy{}}\PY{l+s+s1}{\PYZsq{}} \PY{o}{+} \PY{l+s+s2}{\PYZdq{}}\PY{l+s+se}{\PYZbs{}n}\PY{l+s+se}{\PYZbs{}n}\PY{l+s+s2}{\PYZsq{}}\PY{l+s+s2}{a}\PY{l+s+s2}{\PYZsq{}}\PY{l+s+s2}{ matrix after slicing}\PY{l+s+se}{\PYZbs{}n}\PY{l+s+s2}{\PYZdq{}}\PY{p}{)}
\PY{n+nb}{print}\PY{p}{(}\PY{n}{a}\PY{p}{)} 
\end{Verbatim}
\end{tcolorbox}

    \begin{Verbatim}[commandchars=\\\{\}]

'a' matrix before slicing

[[ 1  2  3  4]
 [ 5  6  7  8]
 [ 9 10 11 12]]

--------------------------------------------------------------------------------
--------------------

'a' matrix after slicing

[[ 1 77  3  4]
 [ 5  6  7  8]
 [ 9 10 11 12]]
    \end{Verbatim}

    \hypertarget{integer-indexing-vs-slicing}{%
\paragraph{Integer Indexing Vs
Slicing}\label{integer-indexing-vs-slicing}}

    \begin{tcolorbox}[breakable, size=fbox, boxrule=1pt, pad at break*=1mm,colback=cellbackground, colframe=cellborder]
\prompt{In}{incolor}{17}{\boxspacing}
\begin{Verbatim}[commandchars=\\\{\}]
\PY{c+c1}{\PYZsh{} Create the following rank 2 array with shape (3, 4)}
\PY{n}{a} \PY{o}{=} \PY{n}{np}\PY{o}{.}\PY{n}{array}\PY{p}{(}\PY{p}{[}\PY{p}{[}\PY{l+m+mi}{1}\PY{p}{,}\PY{l+m+mi}{2}\PY{p}{,}\PY{l+m+mi}{3}\PY{p}{,}\PY{l+m+mi}{4}\PY{p}{]}\PY{p}{,} \PY{p}{[}\PY{l+m+mi}{5}\PY{p}{,}\PY{l+m+mi}{6}\PY{p}{,}\PY{l+m+mi}{7}\PY{p}{,}\PY{l+m+mi}{8}\PY{p}{]}\PY{p}{,} \PY{p}{[}\PY{l+m+mi}{9}\PY{p}{,}\PY{l+m+mi}{10}\PY{p}{,}\PY{l+m+mi}{11}\PY{p}{,}\PY{l+m+mi}{12}\PY{p}{]}\PY{p}{]}\PY{p}{)}
\PY{n+nb}{print}\PY{p}{(}\PY{n}{a}\PY{p}{)}
\end{Verbatim}
\end{tcolorbox}

    \begin{Verbatim}[commandchars=\\\{\}]
[[ 1  2  3  4]
 [ 5  6  7  8]
 [ 9 10 11 12]]
    \end{Verbatim}

    Two ways of accessing the data in the middle row of the array. Using
integer indexing yields an array of lower rank, while using slicing
yields an array of the same rank as the original array:

    \begin{tcolorbox}[breakable, size=fbox, boxrule=1pt, pad at break*=1mm,colback=cellbackground, colframe=cellborder]
\prompt{In}{incolor}{18}{\boxspacing}
\begin{Verbatim}[commandchars=\\\{\}]
\PY{n}{row\PYZus{}r1} \PY{o}{=} \PY{n}{a}\PY{p}{[}\PY{l+m+mi}{1}\PY{p}{,} \PY{p}{:}\PY{p}{]}    \PY{c+c1}{\PYZsh{} Rank 1 view of the second row of a  }
\PY{n}{row\PYZus{}r2} \PY{o}{=} \PY{n}{a}\PY{p}{[}\PY{l+m+mi}{1}\PY{p}{:}\PY{l+m+mi}{2}\PY{p}{,} \PY{p}{:}\PY{p}{]}  \PY{c+c1}{\PYZsh{} Rank 2 view of the second row of a}
\PY{n+nb}{print}\PY{p}{(}\PY{n}{row\PYZus{}r1}\PY{p}{,} \PY{n}{row\PYZus{}r1}\PY{o}{.}\PY{n}{shape}\PY{p}{)}
\PY{n+nb}{print}\PY{p}{(}\PY{n}{row\PYZus{}r2}\PY{p}{,} \PY{n}{row\PYZus{}r2}\PY{o}{.}\PY{n}{shape}\PY{p}{)}
\end{Verbatim}
\end{tcolorbox}

    \begin{Verbatim}[commandchars=\\\{\}]
[5 6 7 8] (4,)
[[5 6 7 8]] (1, 4)
    \end{Verbatim}

    \begin{tcolorbox}[breakable, size=fbox, boxrule=1pt, pad at break*=1mm,colback=cellbackground, colframe=cellborder]
\prompt{In}{incolor}{19}{\boxspacing}
\begin{Verbatim}[commandchars=\\\{\}]
\PY{c+c1}{\PYZsh{} We can make the same distinction when accessing columns of an array:}
\PY{n}{col\PYZus{}r1} \PY{o}{=} \PY{n}{a}\PY{p}{[}\PY{p}{:}\PY{p}{,} \PY{l+m+mi}{1}\PY{p}{]}
\PY{n}{col\PYZus{}r2} \PY{o}{=} \PY{n}{a}\PY{p}{[}\PY{p}{:}\PY{p}{,} \PY{l+m+mi}{1}\PY{p}{:}\PY{l+m+mi}{2}\PY{p}{]}
\PY{n+nb}{print}\PY{p}{(}\PY{n}{col\PYZus{}r1}\PY{p}{,} \PY{n}{col\PYZus{}r1}\PY{o}{.}\PY{n}{shape}\PY{p}{)}
\PY{n+nb}{print}\PY{p}{(}\PY{p}{)}
\PY{n+nb}{print}\PY{p}{(}\PY{n}{col\PYZus{}r2}\PY{p}{,} \PY{n}{col\PYZus{}r2}\PY{o}{.}\PY{n}{shape}\PY{p}{)}
\end{Verbatim}
\end{tcolorbox}

    \begin{Verbatim}[commandchars=\\\{\}]
[ 2  6 10] (3,)

[[ 2]
 [ 6]
 [10]] (3, 1)
    \end{Verbatim}

    When you index into numpy arrays using slicing, the resulting array view
will always be a subarray of the original array. In contrast, integer
array indexing allows you to construct arbitrary arrays using the data
from another array. Here is an example:

    \begin{tcolorbox}[breakable, size=fbox, boxrule=1pt, pad at break*=1mm,colback=cellbackground, colframe=cellborder]
\prompt{In}{incolor}{20}{\boxspacing}
\begin{Verbatim}[commandchars=\\\{\}]
\PY{n+nb}{print}\PY{p}{(}\PY{n}{a}\PY{p}{)}
\end{Verbatim}
\end{tcolorbox}

    \begin{Verbatim}[commandchars=\\\{\}]
[[ 1  2  3  4]
 [ 5  6  7  8]
 [ 9 10 11 12]]
    \end{Verbatim}

    \begin{tcolorbox}[breakable, size=fbox, boxrule=1pt, pad at break*=1mm,colback=cellbackground, colframe=cellborder]
\prompt{In}{incolor}{21}{\boxspacing}
\begin{Verbatim}[commandchars=\\\{\}]
\PY{c+c1}{\PYZsh{} An example of integer array indexing.}
\PY{c+c1}{\PYZsh{} The returned array will have shape (3,)}
\PY{n}{c} \PY{o}{=} \PY{n}{a}\PY{p}{[}\PY{p}{[}\PY{l+m+mi}{0}\PY{p}{,} \PY{l+m+mi}{1}\PY{p}{,} \PY{l+m+mi}{2}\PY{p}{]}\PY{p}{,} \PY{p}{[}\PY{l+m+mi}{0}\PY{p}{,} \PY{l+m+mi}{2}\PY{p}{,} \PY{l+m+mi}{3}\PY{p}{]}\PY{p}{]}
\PY{n+nb}{print}\PY{p}{(}\PY{n}{c}\PY{p}{)}
\PY{n+nb}{print}\PY{p}{(}\PY{n}{c}\PY{o}{.}\PY{n}{shape}\PY{p}{)}
\end{Verbatim}
\end{tcolorbox}

    \begin{Verbatim}[commandchars=\\\{\}]
[ 1  7 12]
(3,)
    \end{Verbatim}

    \begin{tcolorbox}[breakable, size=fbox, boxrule=1pt, pad at break*=1mm,colback=cellbackground, colframe=cellborder]
\prompt{In}{incolor}{22}{\boxspacing}
\begin{Verbatim}[commandchars=\\\{\}]
\PY{c+c1}{\PYZsh{} for example you can get immediately all the diagonal elements of a matrix}
\PY{n}{a} \PY{o}{=} \PY{n}{np}\PY{o}{.}\PY{n}{array}\PY{p}{(}\PY{p}{[}\PY{p}{[}\PY{l+m+mi}{1}\PY{p}{,}\PY{l+m+mi}{2}\PY{p}{,}\PY{l+m+mi}{3}\PY{p}{,}\PY{l+m+mi}{4}\PY{p}{]}\PY{p}{,} \PY{p}{[}\PY{l+m+mi}{5}\PY{p}{,}\PY{l+m+mi}{6}\PY{p}{,}\PY{l+m+mi}{7}\PY{p}{,}\PY{l+m+mi}{8}\PY{p}{]}\PY{p}{,} \PY{p}{[}\PY{l+m+mi}{9}\PY{p}{,}\PY{l+m+mi}{10}\PY{p}{,}\PY{l+m+mi}{11}\PY{p}{,}\PY{l+m+mi}{12}\PY{p}{]}\PY{p}{,} \PY{p}{[}\PY{l+m+mi}{13}\PY{p}{,}\PY{l+m+mi}{14}\PY{p}{,}\PY{l+m+mi}{15}\PY{p}{,}\PY{l+m+mi}{16}\PY{p}{]}\PY{p}{]}\PY{p}{)}
\PY{n}{c}\PY{o}{=} \PY{n}{a}\PY{p}{[}\PY{n}{np}\PY{o}{.}\PY{n}{arange}\PY{p}{(}\PY{n}{a}\PY{o}{.}\PY{n}{shape}\PY{p}{[}\PY{l+m+mi}{0}\PY{p}{]}\PY{p}{)}\PY{p}{,} \PY{n}{np}\PY{o}{.}\PY{n}{arange}\PY{p}{(}\PY{n}{a}\PY{o}{.}\PY{n}{shape}\PY{p}{[}\PY{l+m+mi}{1}\PY{p}{]}\PY{p}{)}\PY{p}{]}
\PY{n+nb}{print}\PY{p}{(}\PY{n}{a}\PY{p}{)}
\PY{n+nb}{print}\PY{p}{(}\PY{l+s+s1}{\PYZsq{}}\PY{l+s+se}{\PYZbs{}n}\PY{l+s+s1}{\PYZsq{}} \PY{o}{+} \PY{l+m+mi}{100}\PY{o}{*}\PY{l+s+s1}{\PYZsq{}}\PY{l+s+s1}{\PYZhy{}}\PY{l+s+s1}{\PYZsq{}} \PY{o}{+} \PY{l+s+s1}{\PYZsq{}}\PY{l+s+se}{\PYZbs{}n}\PY{l+s+s1}{\PYZsq{}}\PY{p}{)}
\PY{n+nb}{print}\PY{p}{(}\PY{n}{c}\PY{p}{)}
\end{Verbatim}
\end{tcolorbox}

    \begin{Verbatim}[commandchars=\\\{\}]
[[ 1  2  3  4]
 [ 5  6  7  8]
 [ 9 10 11 12]
 [13 14 15 16]]

--------------------------------------------------------------------------------
--------------------

[ 1  6 11 16]
    \end{Verbatim}

    \textbf{IMPORTANT :} In case of slice, a view of the array is returned
but \textbf{index array a copy of the original array is returned.}

    \begin{tcolorbox}[breakable, size=fbox, boxrule=1pt, pad at break*=1mm,colback=cellbackground, colframe=cellborder]
\prompt{In}{incolor}{23}{\boxspacing}
\begin{Verbatim}[commandchars=\\\{\}]
\PY{n}{c}\PY{p}{[}\PY{p}{:}\PY{p}{]} \PY{o}{=} \PY{l+m+mi}{42}
\PY{n+nb}{print}\PY{p}{(}\PY{n}{c}\PY{p}{)}
\PY{n+nb}{print}\PY{p}{(}\PY{l+s+s1}{\PYZsq{}}\PY{l+s+se}{\PYZbs{}n}\PY{l+s+s1}{\PYZsq{}} \PY{o}{+} \PY{l+m+mi}{100}\PY{o}{*}\PY{l+s+s1}{\PYZsq{}}\PY{l+s+s1}{\PYZhy{}}\PY{l+s+s1}{\PYZsq{}} \PY{o}{+} \PY{l+s+s1}{\PYZsq{}}\PY{l+s+se}{\PYZbs{}n}\PY{l+s+s1}{\PYZsq{}}\PY{p}{)}
\PY{n+nb}{print}\PY{p}{(}\PY{n}{a}\PY{p}{)}
\end{Verbatim}
\end{tcolorbox}

    \begin{Verbatim}[commandchars=\\\{\}]
[42 42 42 42]

--------------------------------------------------------------------------------
--------------------

[[ 1  2  3  4]
 [ 5  6  7  8]
 [ 9 10 11 12]
 [13 14 15 16]]
    \end{Verbatim}

    \begin{tcolorbox}[breakable, size=fbox, boxrule=1pt, pad at break*=1mm,colback=cellbackground, colframe=cellborder]
\prompt{In}{incolor}{24}{\boxspacing}
\begin{Verbatim}[commandchars=\\\{\}]
\PY{c+c1}{\PYZsh{} When using integer array indexing, you can reuse the same}
\PY{c+c1}{\PYZsh{} element from the source array:}
\PY{n+nb}{print}\PY{p}{(}\PY{n}{a}\PY{p}{[}\PY{p}{[}\PY{l+m+mi}{0}\PY{p}{,} \PY{l+m+mi}{0}\PY{p}{]}\PY{p}{,} \PY{p}{[}\PY{l+m+mi}{1}\PY{p}{,} \PY{l+m+mi}{1}\PY{p}{]}\PY{p}{]}\PY{p}{)}
\end{Verbatim}
\end{tcolorbox}

    \begin{Verbatim}[commandchars=\\\{\}]
[2 2]
    \end{Verbatim}

    One useful trick with integer array indexing is selecting or mutating
one element from each row of a matrix:

    \begin{tcolorbox}[breakable, size=fbox, boxrule=1pt, pad at break*=1mm,colback=cellbackground, colframe=cellborder]
\prompt{In}{incolor}{25}{\boxspacing}
\begin{Verbatim}[commandchars=\\\{\}]
\PY{n+nb}{print}\PY{p}{(}\PY{n}{a}\PY{p}{)}
\end{Verbatim}
\end{tcolorbox}

    \begin{Verbatim}[commandchars=\\\{\}]
[[ 1  2  3  4]
 [ 5  6  7  8]
 [ 9 10 11 12]
 [13 14 15 16]]
    \end{Verbatim}

    \begin{tcolorbox}[breakable, size=fbox, boxrule=1pt, pad at break*=1mm,colback=cellbackground, colframe=cellborder]
\prompt{In}{incolor}{26}{\boxspacing}
\begin{Verbatim}[commandchars=\\\{\}]
\PY{c+c1}{\PYZsh{} Create an array of indices}
\PY{n}{b} \PY{o}{=} \PY{n}{np}\PY{o}{.}\PY{n}{array}\PY{p}{(}\PY{p}{[}\PY{l+m+mi}{0}\PY{p}{,} \PY{l+m+mi}{2}\PY{p}{,} \PY{l+m+mi}{0}\PY{p}{,} \PY{l+m+mi}{1}\PY{p}{]}\PY{p}{)}

\PY{c+c1}{\PYZsh{} Select one element from each row of a using the indices in b}
\PY{n+nb}{print}\PY{p}{(}\PY{n}{a}\PY{p}{[}\PY{n}{np}\PY{o}{.}\PY{n}{arange}\PY{p}{(}\PY{l+m+mi}{4}\PY{p}{)}\PY{p}{,} \PY{n}{b}\PY{p}{]}\PY{p}{)}  \PY{c+c1}{\PYZsh{} Prints \PYZdq{}[ 1  7  9 14]\PYZdq{}}
\end{Verbatim}
\end{tcolorbox}

    \begin{Verbatim}[commandchars=\\\{\}]
[ 1  7  9 14]
    \end{Verbatim}

    \begin{tcolorbox}[breakable, size=fbox, boxrule=1pt, pad at break*=1mm,colback=cellbackground, colframe=cellborder]
\prompt{In}{incolor}{27}{\boxspacing}
\begin{Verbatim}[commandchars=\\\{\}]
\PY{c+c1}{\PYZsh{} Mutate one element from each row of a using the indices in b}
\PY{n}{a}\PY{p}{[}\PY{n}{np}\PY{o}{.}\PY{n}{arange}\PY{p}{(}\PY{l+m+mi}{4}\PY{p}{)}\PY{p}{,} \PY{n}{b}\PY{p}{]} \PY{o}{=}\PY{l+m+mi}{42}
\PY{n+nb}{print}\PY{p}{(}\PY{n}{a}\PY{p}{)}
\end{Verbatim}
\end{tcolorbox}

    \begin{Verbatim}[commandchars=\\\{\}]
[[42  2  3  4]
 [ 5  6 42  8]
 [42 10 11 12]
 [13 42 15 16]]
    \end{Verbatim}

    Slicing and indexing in a multidimensional array can be a little bit
tricky compared to slicing and indexing in a one-dimensional array.

    \begin{tcolorbox}[breakable, size=fbox, boxrule=1pt, pad at break*=1mm,colback=cellbackground, colframe=cellborder]
\prompt{In}{incolor}{28}{\boxspacing}
\begin{Verbatim}[commandchars=\\\{\}]
\PY{n}{array} \PY{o}{=} \PY{n}{np}\PY{o}{.}\PY{n}{array}\PY{p}{(}\PY{p}{[}
    \PY{p}{[}\PY{l+m+mi}{2}\PY{p}{,} \PY{l+m+mi}{4}\PY{p}{,} \PY{l+m+mi}{5}\PY{p}{,} \PY{l+m+mi}{6}\PY{p}{]}\PY{p}{,}
    \PY{p}{[}\PY{l+m+mi}{3}\PY{p}{,} \PY{l+m+mi}{1}\PY{p}{,} \PY{l+m+mi}{6}\PY{p}{,} \PY{l+m+mi}{9}\PY{p}{]}\PY{p}{,}
    \PY{p}{[}\PY{l+m+mi}{4}\PY{p}{,} \PY{l+m+mi}{5}\PY{p}{,} \PY{l+m+mi}{1}\PY{p}{,} \PY{l+m+mi}{9}\PY{p}{]}\PY{p}{,}
    \PY{p}{[}\PY{l+m+mi}{2}\PY{p}{,} \PY{l+m+mi}{9}\PY{p}{,} \PY{l+m+mi}{1}\PY{p}{,} \PY{l+m+mi}{7}\PY{p}{]}
\PY{p}{]}\PY{p}{)}
\PY{n+nb}{print}\PY{p}{(}\PY{n}{array}\PY{p}{)}

\PY{c+c1}{\PYZsh{} Slicing and indexing in 4x4 array}
\PY{c+c1}{\PYZsh{} Print first two rows and first two columns}
\PY{n+nb}{print}\PY{p}{(}\PY{l+s+s2}{\PYZdq{}}\PY{l+s+se}{\PYZbs{}n}\PY{l+s+s2}{Print first two rows and first two columns :}\PY{l+s+se}{\PYZbs{}n}\PY{l+s+se}{\PYZbs{}n}\PY{l+s+s2}{\PYZdq{}}\PY{p}{,} \PY{n}{array}\PY{p}{[}\PY{l+m+mi}{0}\PY{p}{:}\PY{l+m+mi}{2}\PY{p}{,} \PY{l+m+mi}{0}\PY{p}{:}\PY{l+m+mi}{2}\PY{p}{]}\PY{p}{)}

\PY{c+c1}{\PYZsh{} Print all rows and last two columns}
\PY{n+nb}{print}\PY{p}{(}\PY{l+s+s2}{\PYZdq{}}\PY{l+s+se}{\PYZbs{}n}\PY{l+s+s2}{Print all rows and last two columns        :}\PY{l+s+se}{\PYZbs{}n}\PY{l+s+se}{\PYZbs{}n}\PY{l+s+s2}{\PYZdq{}}\PY{p}{,} \PY{n}{array}\PY{p}{[}\PY{p}{:}\PY{p}{,} \PY{l+m+mi}{2}\PY{p}{:}\PY{l+m+mi}{4}\PY{p}{]}\PY{p}{)}

\PY{c+c1}{\PYZsh{} Print all column but middle two rows}
\PY{n+nb}{print}\PY{p}{(}\PY{l+s+s2}{\PYZdq{}}\PY{l+s+se}{\PYZbs{}n}\PY{l+s+s2}{Print all column but middle two rows       :}\PY{l+s+se}{\PYZbs{}n}\PY{l+s+se}{\PYZbs{}n}\PY{l+s+s2}{\PYZdq{}}\PY{p}{,} \PY{n}{array}\PY{p}{[}\PY{l+m+mi}{1}\PY{p}{:}\PY{l+m+mi}{3}\PY{p}{,} \PY{p}{:}\PY{p}{]}\PY{p}{)}
\end{Verbatim}
\end{tcolorbox}

    \begin{Verbatim}[commandchars=\\\{\}]
[[2 4 5 6]
 [3 1 6 9]
 [4 5 1 9]
 [2 9 1 7]]

Print first two rows and first two columns :

 [[2 4]
 [3 1]]

Print all rows and last two columns        :

 [[5 6]
 [6 9]
 [1 9]
 [1 7]]

Print all column but middle two rows       :

 [[3 1 6 9]
 [4 5 1 9]]
    \end{Verbatim}

    \hypertarget{boolean-array-indexing}{%
\paragraph{Boolean Array Indexing}\label{boolean-array-indexing}}

Boolean array indexing lets you pick out arbitrary elements of an array.
Frequently this type of indexing is used to select the elements of an
array that satisfy some condition. Here is an example:

    \begin{tcolorbox}[breakable, size=fbox, boxrule=1pt, pad at break*=1mm,colback=cellbackground, colframe=cellborder]
\prompt{In}{incolor}{29}{\boxspacing}
\begin{Verbatim}[commandchars=\\\{\}]
\PY{k+kn}{import} \PY{n+nn}{numpy} \PY{k}{as} \PY{n+nn}{np}

\PY{n}{a} \PY{o}{=} \PY{n}{np}\PY{o}{.}\PY{n}{array}\PY{p}{(}\PY{p}{[}\PY{p}{[}\PY{l+m+mi}{1}\PY{p}{,}\PY{l+m+mi}{2}\PY{p}{]}\PY{p}{,} \PY{p}{[}\PY{l+m+mi}{3}\PY{p}{,} \PY{l+m+mi}{4}\PY{p}{]}\PY{p}{,} \PY{p}{[}\PY{l+m+mi}{5}\PY{p}{,} \PY{l+m+mi}{6}\PY{p}{]}\PY{p}{]}\PY{p}{)}

\PY{n}{bool\PYZus{}idx} \PY{o}{=} \PY{p}{(}\PY{n}{a} \PY{o}{\PYZgt{}} \PY{l+m+mi}{2}\PY{p}{)}  \PY{c+c1}{\PYZsh{} Find the elements of a that are bigger than 2;}
                    \PY{c+c1}{\PYZsh{} this returns a numpy array of Booleans of the same}
                    \PY{c+c1}{\PYZsh{} shape as a, where each slot of bool\PYZus{}idx tells}
                    \PY{c+c1}{\PYZsh{} whether that element of a is \PYZgt{} 2.}

\PY{n+nb}{print}\PY{p}{(}\PY{n}{bool\PYZus{}idx}\PY{p}{)}
\end{Verbatim}
\end{tcolorbox}

    \begin{Verbatim}[commandchars=\\\{\}]
[[False False]
 [ True  True]
 [ True  True]]
    \end{Verbatim}

    \begin{tcolorbox}[breakable, size=fbox, boxrule=1pt, pad at break*=1mm,colback=cellbackground, colframe=cellborder]
\prompt{In}{incolor}{30}{\boxspacing}
\begin{Verbatim}[commandchars=\\\{\}]
\PY{c+c1}{\PYZsh{} We use boolean array indexing to construct a rank 1 array}
\PY{c+c1}{\PYZsh{} consisting of the elements of a corresponding to the True values}
\PY{c+c1}{\PYZsh{} of bool\PYZus{}idx}
\PY{n+nb}{print}\PY{p}{(}\PY{n}{a}\PY{p}{[}\PY{n}{bool\PYZus{}idx}\PY{p}{]}\PY{p}{)}

\PY{c+c1}{\PYZsh{} We can do all of the above in a single concise statement:}
\PY{n+nb}{print}\PY{p}{(}\PY{n}{a}\PY{p}{[}\PY{n}{a} \PY{o}{\PYZgt{}} \PY{l+m+mi}{2}\PY{p}{]}\PY{p}{)}
\end{Verbatim}
\end{tcolorbox}

    \begin{Verbatim}[commandchars=\\\{\}]
[3 4 5 6]
[3 4 5 6]
    \end{Verbatim}

    For brevity we have left out a lot of details about numpy array
indexing; if you want to know more you should read the documentation.

    \hypertarget{datatypes}{%
\subsubsection{Datatypes}\label{datatypes}}

    Every numpy array is a grid of elements of the same type. Numpy provides
a large set of numeric datatypes that you can use to construct arrays.
Numpy tries to guess a datatype when you create an array, but functions
that construct arrays usually also include an optional argument to
explicitly specify the datatype. Here is an example:

    \begin{tcolorbox}[breakable, size=fbox, boxrule=1pt, pad at break*=1mm,colback=cellbackground, colframe=cellborder]
\prompt{In}{incolor}{31}{\boxspacing}
\begin{Verbatim}[commandchars=\\\{\}]
\PY{n}{x} \PY{o}{=} \PY{n}{np}\PY{o}{.}\PY{n}{array}\PY{p}{(}\PY{p}{[}\PY{l+m+mi}{1}\PY{p}{,} \PY{l+m+mi}{2}\PY{p}{]}\PY{p}{)}  \PY{c+c1}{\PYZsh{} Let numpy choose the datatype}
\PY{n}{y} \PY{o}{=} \PY{n}{np}\PY{o}{.}\PY{n}{array}\PY{p}{(}\PY{p}{[}\PY{l+m+mf}{1.0}\PY{p}{,} \PY{l+m+mf}{2.0}\PY{p}{]}\PY{p}{)}  \PY{c+c1}{\PYZsh{} Let numpy choose the datatype}
\PY{n}{z} \PY{o}{=} \PY{n}{np}\PY{o}{.}\PY{n}{array}\PY{p}{(}\PY{p}{[}\PY{l+m+mi}{1}\PY{p}{,} \PY{l+m+mi}{2}\PY{p}{]}\PY{p}{,} \PY{n}{dtype}\PY{o}{=}\PY{n}{np}\PY{o}{.}\PY{n}{int64}\PY{p}{)}  \PY{c+c1}{\PYZsh{} Force a particular datatype}

\PY{n+nb}{print}\PY{p}{(}\PY{n}{x}\PY{o}{.}\PY{n}{dtype}\PY{p}{,} \PY{n}{y}\PY{o}{.}\PY{n}{dtype}\PY{p}{,} \PY{n}{z}\PY{o}{.}\PY{n}{dtype}\PY{p}{)}
\end{Verbatim}
\end{tcolorbox}

    \begin{Verbatim}[commandchars=\\\{\}]
int32 float64 int64
    \end{Verbatim}

    You can read all about numpy datatypes in the
\href{http://docs.scipy.org/doc/numpy/reference/arrays.dtypes.html}{documentation}.

    \hypertarget{array-math}{%
\subsubsection{Array Math}\label{array-math}}

    Basic mathematical functions operate elementwise on arrays, and are
available both as operator overloads and as functions in the numpy
module:

    \begin{tcolorbox}[breakable, size=fbox, boxrule=1pt, pad at break*=1mm,colback=cellbackground, colframe=cellborder]
\prompt{In}{incolor}{32}{\boxspacing}
\begin{Verbatim}[commandchars=\\\{\}]
\PY{n}{x} \PY{o}{=} \PY{n}{np}\PY{o}{.}\PY{n}{array}\PY{p}{(}\PY{p}{[}\PY{p}{[}\PY{l+m+mi}{1}\PY{p}{,}\PY{l+m+mi}{2}\PY{p}{]}\PY{p}{,}\PY{p}{[}\PY{l+m+mi}{3}\PY{p}{,}\PY{l+m+mi}{4}\PY{p}{]}\PY{p}{]}\PY{p}{,} \PY{n}{dtype}\PY{o}{=}\PY{n}{np}\PY{o}{.}\PY{n}{float64}\PY{p}{)}
\PY{n}{y} \PY{o}{=} \PY{n}{np}\PY{o}{.}\PY{n}{array}\PY{p}{(}\PY{p}{[}\PY{p}{[}\PY{l+m+mi}{5}\PY{p}{,}\PY{l+m+mi}{6}\PY{p}{]}\PY{p}{,}\PY{p}{[}\PY{l+m+mi}{7}\PY{p}{,}\PY{l+m+mi}{8}\PY{p}{]}\PY{p}{]}\PY{p}{,} \PY{n}{dtype}\PY{o}{=}\PY{n}{np}\PY{o}{.}\PY{n}{float64}\PY{p}{)}

\PY{c+c1}{\PYZsh{} Elementwise sum; both produce the array}
\PY{n+nb}{print}\PY{p}{(}\PY{n}{x} \PY{o}{+} \PY{n}{y}\PY{p}{)}
\PY{n+nb}{print}\PY{p}{(}\PY{n}{np}\PY{o}{.}\PY{n}{add}\PY{p}{(}\PY{n}{x}\PY{p}{,} \PY{n}{y}\PY{p}{)}\PY{p}{)}
\end{Verbatim}
\end{tcolorbox}

    \begin{Verbatim}[commandchars=\\\{\}]
[[ 6.  8.]
 [10. 12.]]
[[ 6.  8.]
 [10. 12.]]
    \end{Verbatim}

    \begin{tcolorbox}[breakable, size=fbox, boxrule=1pt, pad at break*=1mm,colback=cellbackground, colframe=cellborder]
\prompt{In}{incolor}{33}{\boxspacing}
\begin{Verbatim}[commandchars=\\\{\}]
\PY{c+c1}{\PYZsh{} Elementwise difference; both produce the array}
\PY{n+nb}{print}\PY{p}{(}\PY{n}{x} \PY{o}{\PYZhy{}} \PY{n}{y}\PY{p}{)}
\PY{n+nb}{print}\PY{p}{(}\PY{n}{np}\PY{o}{.}\PY{n}{subtract}\PY{p}{(}\PY{n}{x}\PY{p}{,} \PY{n}{y}\PY{p}{)}\PY{p}{)}
\end{Verbatim}
\end{tcolorbox}

    \begin{Verbatim}[commandchars=\\\{\}]
[[-4. -4.]
 [-4. -4.]]
[[-4. -4.]
 [-4. -4.]]
    \end{Verbatim}

    \begin{tcolorbox}[breakable, size=fbox, boxrule=1pt, pad at break*=1mm,colback=cellbackground, colframe=cellborder]
\prompt{In}{incolor}{34}{\boxspacing}
\begin{Verbatim}[commandchars=\\\{\}]
\PY{c+c1}{\PYZsh{} Elementwise product; both produce the array}
\PY{n+nb}{print}\PY{p}{(}\PY{n}{x} \PY{o}{*} \PY{n}{y}\PY{p}{)}
\PY{n+nb}{print}\PY{p}{(}\PY{n}{np}\PY{o}{.}\PY{n}{multiply}\PY{p}{(}\PY{n}{x}\PY{p}{,} \PY{n}{y}\PY{p}{)}\PY{p}{)}
\end{Verbatim}
\end{tcolorbox}

    \begin{Verbatim}[commandchars=\\\{\}]
[[ 5. 12.]
 [21. 32.]]
[[ 5. 12.]
 [21. 32.]]
    \end{Verbatim}

    \begin{tcolorbox}[breakable, size=fbox, boxrule=1pt, pad at break*=1mm,colback=cellbackground, colframe=cellborder]
\prompt{In}{incolor}{35}{\boxspacing}
\begin{Verbatim}[commandchars=\\\{\}]
\PY{c+c1}{\PYZsh{} Elementwise division; both produce the array}
\PY{c+c1}{\PYZsh{} [[ 0.2         0.33333333]}
\PY{c+c1}{\PYZsh{}  [ 0.42857143  0.5       ]]}
\PY{n+nb}{print}\PY{p}{(}\PY{n}{x} \PY{o}{/} \PY{n}{y}\PY{p}{)}
\PY{n+nb}{print}\PY{p}{(}\PY{n}{np}\PY{o}{.}\PY{n}{divide}\PY{p}{(}\PY{n}{x}\PY{p}{,} \PY{n}{y}\PY{p}{)}\PY{p}{)}
\end{Verbatim}
\end{tcolorbox}

    \begin{Verbatim}[commandchars=\\\{\}]
[[0.2        0.33333333]
 [0.42857143 0.5       ]]
[[0.2        0.33333333]
 [0.42857143 0.5       ]]
    \end{Verbatim}

    \begin{tcolorbox}[breakable, size=fbox, boxrule=1pt, pad at break*=1mm,colback=cellbackground, colframe=cellborder]
\prompt{In}{incolor}{36}{\boxspacing}
\begin{Verbatim}[commandchars=\\\{\}]
\PY{c+c1}{\PYZsh{} Elementwise square root; produces the array}
\PY{c+c1}{\PYZsh{} [[ 1.          1.41421356]}
\PY{c+c1}{\PYZsh{}  [ 1.73205081  2.        ]]}
\PY{n+nb}{print}\PY{p}{(}\PY{n}{np}\PY{o}{.}\PY{n}{sqrt}\PY{p}{(}\PY{n}{x}\PY{p}{)}\PY{p}{)}
\end{Verbatim}
\end{tcolorbox}

    \begin{Verbatim}[commandchars=\\\{\}]
[[1.         1.41421356]
 [1.73205081 2.        ]]
    \end{Verbatim}

    Note that unlike MATLAB, \texttt{*} is elementwise multiplication, not
matrix multiplication. We instead use the dot function to compute inner
products of vectors, to multiply a vector by a matrix, and to multiply
matrices. dot is available both as a function in the numpy module and as
an instance method of array objects:

    \begin{tcolorbox}[breakable, size=fbox, boxrule=1pt, pad at break*=1mm,colback=cellbackground, colframe=cellborder]
\prompt{In}{incolor}{37}{\boxspacing}
\begin{Verbatim}[commandchars=\\\{\}]
\PY{n}{x} \PY{o}{=} \PY{n}{np}\PY{o}{.}\PY{n}{array}\PY{p}{(}\PY{p}{[}\PY{p}{[}\PY{l+m+mi}{1}\PY{p}{,}\PY{l+m+mi}{2}\PY{p}{]}\PY{p}{,}\PY{p}{[}\PY{l+m+mi}{3}\PY{p}{,}\PY{l+m+mi}{4}\PY{p}{]}\PY{p}{]}\PY{p}{)}
\PY{n}{y} \PY{o}{=} \PY{n}{np}\PY{o}{.}\PY{n}{array}\PY{p}{(}\PY{p}{[}\PY{p}{[}\PY{l+m+mi}{5}\PY{p}{,}\PY{l+m+mi}{6}\PY{p}{]}\PY{p}{,}\PY{p}{[}\PY{l+m+mi}{7}\PY{p}{,}\PY{l+m+mi}{8}\PY{p}{]}\PY{p}{]}\PY{p}{)}

\PY{n}{v} \PY{o}{=} \PY{n}{np}\PY{o}{.}\PY{n}{array}\PY{p}{(}\PY{p}{[}\PY{l+m+mi}{9}\PY{p}{,}\PY{l+m+mi}{10}\PY{p}{]}\PY{p}{)}
\PY{n}{w} \PY{o}{=} \PY{n}{np}\PY{o}{.}\PY{n}{array}\PY{p}{(}\PY{p}{[}\PY{l+m+mi}{11}\PY{p}{,} \PY{l+m+mi}{12}\PY{p}{]}\PY{p}{)}

\PY{c+c1}{\PYZsh{} Inner product of vectors; both produce 219}
\PY{n+nb}{print}\PY{p}{(}\PY{n}{v}\PY{o}{.}\PY{n}{dot}\PY{p}{(}\PY{n}{w}\PY{p}{)}\PY{p}{)}
\PY{n+nb}{print}\PY{p}{(}\PY{n}{np}\PY{o}{.}\PY{n}{dot}\PY{p}{(}\PY{n}{v}\PY{p}{,} \PY{n}{w}\PY{p}{)}\PY{p}{)}
\end{Verbatim}
\end{tcolorbox}

    \begin{Verbatim}[commandchars=\\\{\}]
219
219
    \end{Verbatim}

    You can also use the \texttt{@} operator which is equivalent to numpy's
\texttt{dot} operator.

    \begin{tcolorbox}[breakable, size=fbox, boxrule=1pt, pad at break*=1mm,colback=cellbackground, colframe=cellborder]
\prompt{In}{incolor}{38}{\boxspacing}
\begin{Verbatim}[commandchars=\\\{\}]
\PY{n+nb}{print}\PY{p}{(}\PY{n}{v} \PY{o}{@} \PY{n}{w}\PY{p}{)}
\end{Verbatim}
\end{tcolorbox}

    \begin{Verbatim}[commandchars=\\\{\}]
219
    \end{Verbatim}

    \begin{tcolorbox}[breakable, size=fbox, boxrule=1pt, pad at break*=1mm,colback=cellbackground, colframe=cellborder]
\prompt{In}{incolor}{39}{\boxspacing}
\begin{Verbatim}[commandchars=\\\{\}]
\PY{c+c1}{\PYZsh{} Matrix / vector product; both produce the rank 1 array [29 67]}
\PY{n+nb}{print}\PY{p}{(}\PY{n}{x}\PY{o}{.}\PY{n}{dot}\PY{p}{(}\PY{n}{v}\PY{p}{)}\PY{p}{)}
\PY{n+nb}{print}\PY{p}{(}\PY{n}{np}\PY{o}{.}\PY{n}{dot}\PY{p}{(}\PY{n}{x}\PY{p}{,} \PY{n}{v}\PY{p}{)}\PY{p}{)}
\PY{n+nb}{print}\PY{p}{(}\PY{n}{x} \PY{o}{@} \PY{n}{v}\PY{p}{)}
\end{Verbatim}
\end{tcolorbox}

    \begin{Verbatim}[commandchars=\\\{\}]
[29 67]
[29 67]
[29 67]
    \end{Verbatim}

    \begin{tcolorbox}[breakable, size=fbox, boxrule=1pt, pad at break*=1mm,colback=cellbackground, colframe=cellborder]
\prompt{In}{incolor}{40}{\boxspacing}
\begin{Verbatim}[commandchars=\\\{\}]
\PY{c+c1}{\PYZsh{} Matrix / matrix product; both produce the rank 2 array}
\PY{c+c1}{\PYZsh{} [[19 22]}
\PY{c+c1}{\PYZsh{}  [43 50]]}
\PY{n+nb}{print}\PY{p}{(}\PY{n}{x}\PY{o}{.}\PY{n}{dot}\PY{p}{(}\PY{n}{y}\PY{p}{)}\PY{p}{)}
\PY{n+nb}{print}\PY{p}{(}\PY{n}{np}\PY{o}{.}\PY{n}{dot}\PY{p}{(}\PY{n}{x}\PY{p}{,} \PY{n}{y}\PY{p}{)}\PY{p}{)}
\PY{n+nb}{print}\PY{p}{(}\PY{n}{x} \PY{o}{@} \PY{n}{y}\PY{p}{)}
\end{Verbatim}
\end{tcolorbox}

    \begin{Verbatim}[commandchars=\\\{\}]
[[19 22]
 [43 50]]
[[19 22]
 [43 50]]
[[19 22]
 [43 50]]
    \end{Verbatim}

    Numpy provides many useful functions for performing computations on
arrays; one of the most useful is \texttt{sum}:

    \begin{tcolorbox}[breakable, size=fbox, boxrule=1pt, pad at break*=1mm,colback=cellbackground, colframe=cellborder]
\prompt{In}{incolor}{41}{\boxspacing}
\begin{Verbatim}[commandchars=\\\{\}]
\PY{n}{x} \PY{o}{=} \PY{n}{np}\PY{o}{.}\PY{n}{array}\PY{p}{(}\PY{p}{[}\PY{p}{[}\PY{l+m+mi}{1}\PY{p}{,}\PY{l+m+mi}{2}\PY{p}{]}\PY{p}{,}\PY{p}{[}\PY{l+m+mi}{3}\PY{p}{,}\PY{l+m+mi}{4}\PY{p}{]}\PY{p}{]}\PY{p}{)}

\PY{n+nb}{print}\PY{p}{(}\PY{n}{np}\PY{o}{.}\PY{n}{sum}\PY{p}{(}\PY{n}{x}\PY{p}{)}\PY{p}{)}  \PY{c+c1}{\PYZsh{} Compute sum of all elements; prints \PYZdq{}10\PYZdq{}}
\PY{n+nb}{print}\PY{p}{(}\PY{n}{np}\PY{o}{.}\PY{n}{sum}\PY{p}{(}\PY{n}{x}\PY{p}{,} \PY{n}{axis}\PY{o}{=}\PY{l+m+mi}{0}\PY{p}{)}\PY{p}{)}  \PY{c+c1}{\PYZsh{} Compute sum of each column; prints \PYZdq{}[4 6]\PYZdq{}}
\PY{n+nb}{print}\PY{p}{(}\PY{n}{np}\PY{o}{.}\PY{n}{sum}\PY{p}{(}\PY{n}{x}\PY{p}{,} \PY{n}{axis}\PY{o}{=}\PY{l+m+mi}{1}\PY{p}{)}\PY{p}{)}  \PY{c+c1}{\PYZsh{} Compute sum of each row; prints \PYZdq{}[3 7]\PYZdq{}}
\end{Verbatim}
\end{tcolorbox}

    \begin{Verbatim}[commandchars=\\\{\}]
10
[4 6]
[3 7]
    \end{Verbatim}

    \begin{tcolorbox}[breakable, size=fbox, boxrule=1pt, pad at break*=1mm,colback=cellbackground, colframe=cellborder]
\prompt{In}{incolor}{42}{\boxspacing}
\begin{Verbatim}[commandchars=\\\{\}]
\PY{k+kn}{import} \PY{n+nn}{math}

\PY{n}{x} \PY{o}{=} \PY{n}{np}\PY{o}{.}\PY{n}{arange}\PY{p}{(}\PY{l+m+mi}{0}\PY{p}{,} \PY{l+m+mi}{2}\PY{o}{*}\PY{n}{math}\PY{o}{.}\PY{n}{pi}\PY{p}{,} \PY{l+m+mf}{0.01}\PY{p}{)}
\PY{n}{y} \PY{o}{=} \PY{n}{np}\PY{o}{.}\PY{n}{sin}\PY{p}{(}\PY{n}{x}\PY{p}{)}
\PY{n}{z} \PY{o}{=} \PY{n}{np}\PY{o}{.}\PY{n}{cos}\PY{p}{(}\PY{n}{x}\PY{p}{)}
\PY{n}{w} \PY{o}{=} \PY{n}{np}\PY{o}{.}\PY{n}{sin}\PY{p}{(}\PY{l+m+mi}{20}\PY{o}{*}\PY{n}{x}\PY{p}{)}\PY{o}{*}\PY{n}{np}\PY{o}{.}\PY{n}{exp}\PY{p}{(}\PY{o}{\PYZhy{}}\PY{n}{x}\PY{p}{)}
\end{Verbatim}
\end{tcolorbox}

    \begin{tcolorbox}[breakable, size=fbox, boxrule=1pt, pad at break*=1mm,colback=cellbackground, colframe=cellborder]
\prompt{In}{incolor}{43}{\boxspacing}
\begin{Verbatim}[commandchars=\\\{\}]
\PY{k+kn}{import} \PY{n+nn}{matplotlib}\PY{n+nn}{.}\PY{n+nn}{pyplot} \PY{k}{as} \PY{n+nn}{plt}

\PY{n}{plt}\PY{o}{.}\PY{n}{plot}\PY{p}{(}\PY{n}{x}\PY{p}{,} \PY{n}{y}\PY{p}{,} \PY{l+s+s1}{\PYZsq{}}\PY{l+s+s1}{r}\PY{l+s+s1}{\PYZsq{}}\PY{p}{)}
\PY{n}{plt}\PY{o}{.}\PY{n}{plot}\PY{p}{(}\PY{n}{x}\PY{p}{,} \PY{n}{z}\PY{p}{,} \PY{l+s+s1}{\PYZsq{}}\PY{l+s+s1}{b}\PY{l+s+s1}{\PYZsq{}}\PY{p}{)}
\PY{n}{plt}\PY{o}{.}\PY{n}{plot}\PY{p}{(}\PY{n}{x}\PY{p}{,} \PY{n}{w}\PY{p}{,} \PY{l+s+s1}{\PYZsq{}}\PY{l+s+s1}{g}\PY{l+s+s1}{\PYZsq{}}\PY{p}{)}
\PY{n}{plt}\PY{o}{.}\PY{n}{show}
\end{Verbatim}
\end{tcolorbox}

            \begin{tcolorbox}[breakable, size=fbox, boxrule=.5pt, pad at break*=1mm, opacityfill=0]
\prompt{Out}{outcolor}{43}{\boxspacing}
\begin{Verbatim}[commandchars=\\\{\}]
<function matplotlib.pyplot.show(*args, **kw)>
\end{Verbatim}
\end{tcolorbox}
        
    \begin{center}
    \adjustimage{max size={0.9\linewidth}{0.9\paperheight}}{output_67_1.png}
    \end{center}
    { \hspace*{\fill} \\}
    
    You can find the full list of mathematical functions provided by numpy
in the
\href{http://docs.scipy.org/doc/numpy/reference/routines.math.html}{documentation}.

Apart from computing mathematical functions using arrays, we frequently
need to reshape or otherwise manipulate data in arrays. The simplest
example of this type of operation is transposing a matrix; to transpose
a matrix, simply use the T attribute of an array object:

    \begin{tcolorbox}[breakable, size=fbox, boxrule=1pt, pad at break*=1mm,colback=cellbackground, colframe=cellborder]
\prompt{In}{incolor}{44}{\boxspacing}
\begin{Verbatim}[commandchars=\\\{\}]
\PY{n}{x} \PY{o}{=} \PY{n}{np}\PY{o}{.}\PY{n}{array}\PY{p}{(}\PY{p}{[}\PY{p}{[}\PY{l+m+mi}{1}\PY{p}{,}\PY{l+m+mi}{2}\PY{p}{]}\PY{p}{,}\PY{p}{[}\PY{l+m+mi}{3}\PY{p}{,}\PY{l+m+mi}{4}\PY{p}{]}\PY{p}{]}\PY{p}{)}
\PY{n+nb}{print}\PY{p}{(}\PY{n}{x}\PY{p}{)}
\PY{n+nb}{print}\PY{p}{(}\PY{l+s+s2}{\PYZdq{}}\PY{l+s+s2}{transpose}\PY{l+s+se}{\PYZbs{}n}\PY{l+s+s2}{\PYZdq{}}\PY{p}{,} \PY{n}{x}\PY{o}{.}\PY{n}{T}\PY{p}{)}
\end{Verbatim}
\end{tcolorbox}

    \begin{Verbatim}[commandchars=\\\{\}]
[[1 2]
 [3 4]]
transpose
 [[1 3]
 [2 4]]
    \end{Verbatim}

    \begin{tcolorbox}[breakable, size=fbox, boxrule=1pt, pad at break*=1mm,colback=cellbackground, colframe=cellborder]
\prompt{In}{incolor}{45}{\boxspacing}
\begin{Verbatim}[commandchars=\\\{\}]
\PY{n}{v} \PY{o}{=} \PY{n}{np}\PY{o}{.}\PY{n}{array}\PY{p}{(}\PY{p}{[}\PY{p}{[}\PY{l+m+mi}{1}\PY{p}{,}\PY{l+m+mi}{2}\PY{p}{,}\PY{l+m+mi}{3}\PY{p}{]}\PY{p}{]}\PY{p}{)}
\PY{n+nb}{print}\PY{p}{(}\PY{n}{v} \PY{p}{)}
\PY{n+nb}{print}\PY{p}{(}\PY{l+s+s2}{\PYZdq{}}\PY{l+s+s2}{transpose}\PY{l+s+se}{\PYZbs{}n}\PY{l+s+s2}{\PYZdq{}}\PY{p}{,} \PY{n}{v}\PY{o}{.}\PY{n}{T}\PY{p}{)}
\end{Verbatim}
\end{tcolorbox}

    \begin{Verbatim}[commandchars=\\\{\}]
[[1 2 3]]
transpose
 [[1]
 [2]
 [3]]
    \end{Verbatim}

    \hypertarget{broadcasting}{%
\subsubsection{Broadcasting}\label{broadcasting}}

    Suppose we want to add a constant vector to each row of a matrix. We
could do it like this:

    \begin{tcolorbox}[breakable, size=fbox, boxrule=1pt, pad at break*=1mm,colback=cellbackground, colframe=cellborder]
\prompt{In}{incolor}{46}{\boxspacing}
\begin{Verbatim}[commandchars=\\\{\}]
\PY{c+c1}{\PYZsh{} We will add the vector v to each row of the matrix x,}
\PY{c+c1}{\PYZsh{} storing the result in the matrix y}
\PY{n}{x} \PY{o}{=} \PY{n}{np}\PY{o}{.}\PY{n}{array}\PY{p}{(}\PY{p}{[}\PY{p}{[}\PY{l+m+mi}{1}\PY{p}{,}\PY{l+m+mi}{2}\PY{p}{,}\PY{l+m+mi}{3}\PY{p}{]}\PY{p}{,} \PY{p}{[}\PY{l+m+mi}{4}\PY{p}{,}\PY{l+m+mi}{5}\PY{p}{,}\PY{l+m+mi}{6}\PY{p}{]}\PY{p}{,} \PY{p}{[}\PY{l+m+mi}{7}\PY{p}{,}\PY{l+m+mi}{8}\PY{p}{,}\PY{l+m+mi}{9}\PY{p}{]}\PY{p}{,} \PY{p}{[}\PY{l+m+mi}{10}\PY{p}{,} \PY{l+m+mi}{11}\PY{p}{,} \PY{l+m+mi}{12}\PY{p}{]}\PY{p}{]}\PY{p}{)}
\PY{n}{v} \PY{o}{=} \PY{n}{np}\PY{o}{.}\PY{n}{array}\PY{p}{(}\PY{p}{[}\PY{l+m+mi}{42}\PY{p}{,} \PY{l+m+mi}{42}\PY{p}{,} \PY{l+m+mi}{42}\PY{p}{]}\PY{p}{)}
\PY{n}{y} \PY{o}{=} \PY{n}{np}\PY{o}{.}\PY{n}{empty\PYZus{}like}\PY{p}{(}\PY{n}{x}\PY{p}{)}   \PY{c+c1}{\PYZsh{} Create an empty matrix with the same shape as x}

\PY{c+c1}{\PYZsh{} Add the vector v to each row of the matrix x with an explicit loop}
\PY{k}{for} \PY{n}{i} \PY{o+ow}{in} \PY{n+nb}{range}\PY{p}{(}\PY{l+m+mi}{4}\PY{p}{)}\PY{p}{:}
    \PY{n}{y}\PY{p}{[}\PY{n}{i}\PY{p}{,} \PY{p}{:}\PY{p}{]} \PY{o}{=} \PY{n}{x}\PY{p}{[}\PY{n}{i}\PY{p}{,} \PY{p}{:}\PY{p}{]} \PY{o}{+} \PY{n}{v}

\PY{n+nb}{print}\PY{p}{(}\PY{n}{y}\PY{p}{)}
\end{Verbatim}
\end{tcolorbox}

    \begin{Verbatim}[commandchars=\\\{\}]
[[43 44 45]
 [46 47 48]
 [49 50 51]
 [52 53 54]]
    \end{Verbatim}

    This works; however when the matrix x is very large, computing an
explicit loop in Python could be slow. Note that adding the vector v to
each row of the matrix x is equivalent to forming a matrix vv by
stacking multiple copies of v vertically, then performing elementwise
summation of x and vv. We could implement this approach like this:

    \begin{tcolorbox}[breakable, size=fbox, boxrule=1pt, pad at break*=1mm,colback=cellbackground, colframe=cellborder]
\prompt{In}{incolor}{47}{\boxspacing}
\begin{Verbatim}[commandchars=\\\{\}]
\PY{n}{vv} \PY{o}{=} \PY{n}{np}\PY{o}{.}\PY{n}{tile}\PY{p}{(}\PY{n}{v}\PY{p}{,} \PY{p}{(}\PY{l+m+mi}{4}\PY{p}{,} \PY{l+m+mi}{1}\PY{p}{)}\PY{p}{)}  \PY{c+c1}{\PYZsh{} Stack 4 copies of v on top of each other}
\PY{n+nb}{print}\PY{p}{(}\PY{n}{vv}\PY{p}{)}                \PY{c+c1}{\PYZsh{} Prints \PYZdq{}[[42 42 42]}
                         \PY{c+c1}{\PYZsh{}          [42 42 42]}
                         \PY{c+c1}{\PYZsh{}          [42 42 42]}
                         \PY{c+c1}{\PYZsh{}          [42 42 42]]\PYZdq{}}
\end{Verbatim}
\end{tcolorbox}

    \begin{Verbatim}[commandchars=\\\{\}]
[[42 42 42]
 [42 42 42]
 [42 42 42]
 [42 42 42]]
    \end{Verbatim}

    \begin{tcolorbox}[breakable, size=fbox, boxrule=1pt, pad at break*=1mm,colback=cellbackground, colframe=cellborder]
\prompt{In}{incolor}{48}{\boxspacing}
\begin{Verbatim}[commandchars=\\\{\}]
\PY{n}{y} \PY{o}{=} \PY{n}{x} \PY{o}{+} \PY{n}{vv}  \PY{c+c1}{\PYZsh{} Add x and vv elementwise}
\PY{n+nb}{print}\PY{p}{(}\PY{n}{y}\PY{p}{)}
\end{Verbatim}
\end{tcolorbox}

    \begin{Verbatim}[commandchars=\\\{\}]
[[43 44 45]
 [46 47 48]
 [49 50 51]
 [52 53 54]]
    \end{Verbatim}

    \textbf{\emph{Broadcasting}} is a powerful mechanism that allows numpy
to work with arrays of different shapes when performing arithmetic
operations. Frequently we have a smaller array and a larger array, and
we want to use the smaller array multiple times to perform some
operation on the larger array. For example, Numpy broadcasting allows us
to perform this computation without actually creating multiple copies of
v. Consider this version, using broadcasting:

    \begin{tcolorbox}[breakable, size=fbox, boxrule=1pt, pad at break*=1mm,colback=cellbackground, colframe=cellborder]
\prompt{In}{incolor}{49}{\boxspacing}
\begin{Verbatim}[commandchars=\\\{\}]
\PY{k+kn}{import} \PY{n+nn}{numpy} \PY{k}{as} \PY{n+nn}{np}

\PY{c+c1}{\PYZsh{} We will add the vector v to each row of the matrix x,}
\PY{c+c1}{\PYZsh{} storing the result in the matrix y}
\PY{n}{x} \PY{o}{=} \PY{n}{np}\PY{o}{.}\PY{n}{array}\PY{p}{(}\PY{p}{[}\PY{p}{[}\PY{l+m+mi}{1}\PY{p}{,}\PY{l+m+mi}{2}\PY{p}{,}\PY{l+m+mi}{3}\PY{p}{]}\PY{p}{,} \PY{p}{[}\PY{l+m+mi}{4}\PY{p}{,}\PY{l+m+mi}{5}\PY{p}{,}\PY{l+m+mi}{6}\PY{p}{]}\PY{p}{,} \PY{p}{[}\PY{l+m+mi}{7}\PY{p}{,}\PY{l+m+mi}{8}\PY{p}{,}\PY{l+m+mi}{9}\PY{p}{]}\PY{p}{,} \PY{p}{[}\PY{l+m+mi}{10}\PY{p}{,} \PY{l+m+mi}{11}\PY{p}{,} \PY{l+m+mi}{12}\PY{p}{]}\PY{p}{]}\PY{p}{)}
\PY{n}{v} \PY{o}{=} \PY{n}{np}\PY{o}{.}\PY{n}{array}\PY{p}{(}\PY{p}{[}\PY{l+m+mi}{1}\PY{p}{,} \PY{l+m+mi}{0}\PY{p}{,} \PY{l+m+mi}{1}\PY{p}{]}\PY{p}{)}
\PY{n}{y} \PY{o}{=} \PY{n}{x} \PY{o}{+} \PY{n}{v}  \PY{c+c1}{\PYZsh{} Add v to each row of x using broadcasting}
\PY{n+nb}{print}\PY{p}{(}\PY{n}{y}\PY{p}{)}
\end{Verbatim}
\end{tcolorbox}

    \begin{Verbatim}[commandchars=\\\{\}]
[[ 2  2  4]
 [ 5  5  7]
 [ 8  8 10]
 [11 11 13]]
    \end{Verbatim}

    The line \texttt{y\ =\ x\ +\ v} works even though \texttt{x} has shape
\texttt{(4,\ 3)} and \texttt{v} has shape \texttt{(3,)} due to
broadcasting; this line works as if v actually had shape
\texttt{(4,\ 3)}, where each row was a copy of \texttt{v}, and the sum
was performed elementwise.

Broadcasting two arrays together follows these rules:

\begin{enumerate}
\def\labelenumi{\arabic{enumi}.}
\tightlist
\item
  If the arrays do not have the same rank, prepend the shape of the
  lower rank array with 1s until both shapes have the same length.
\item
  The two arrays are said to be compatible in a dimension if they have
  the same size in the dimension, or if one of the arrays has size 1 in
  that dimension.
\item
  The arrays can be broadcast together if they are compatible in all
  dimensions.
\item
  After broadcasting, each array behaves as if it had shape equal to the
  elementwise maximum of shapes of the two input arrays.
\item
  In any dimension where one array had size 1 and the other array had
  size greater than 1, the first array behaves as if it were copied
  along that dimension
\end{enumerate}

If this explanation does not make sense, try reading the explanation
from the
\href{http://docs.scipy.org/doc/numpy/user/basics.broadcasting.html}{documentation}
or this \href{http://wiki.scipy.org/EricsBroadcastingDoc}{explanation}.

Functions that support broadcasting are known as universal functions.
You can find the list of all universal functions in the
\href{http://docs.scipy.org/doc/numpy/reference/ufuncs.html\#available-ufuncs}{documentation}.

Here are some applications of broadcasting:

    \begin{tcolorbox}[breakable, size=fbox, boxrule=1pt, pad at break*=1mm,colback=cellbackground, colframe=cellborder]
\prompt{In}{incolor}{50}{\boxspacing}
\begin{Verbatim}[commandchars=\\\{\}]
\PY{c+c1}{\PYZsh{} Compute outer product of vectors}
\PY{n}{v} \PY{o}{=} \PY{n}{np}\PY{o}{.}\PY{n}{array}\PY{p}{(}\PY{p}{[}\PY{l+m+mi}{1}\PY{p}{,}\PY{l+m+mi}{2}\PY{p}{,}\PY{l+m+mi}{3}\PY{p}{]}\PY{p}{)}  \PY{c+c1}{\PYZsh{} v has shape (3,)}
\PY{n}{w} \PY{o}{=} \PY{n}{np}\PY{o}{.}\PY{n}{array}\PY{p}{(}\PY{p}{[}\PY{l+m+mi}{4}\PY{p}{,}\PY{l+m+mi}{5}\PY{p}{]}\PY{p}{)}    \PY{c+c1}{\PYZsh{} w has shape (2,)}
\PY{c+c1}{\PYZsh{} To compute an outer product, we first reshape v to be a column}
\PY{c+c1}{\PYZsh{} vector of shape (3, 1); we can then broadcast it against w to yield}
\PY{c+c1}{\PYZsh{} an output of shape (3, 2), which is the outer product of v and w:}

\PY{n+nb}{print}\PY{p}{(}\PY{n}{np}\PY{o}{.}\PY{n}{reshape}\PY{p}{(}\PY{n}{v}\PY{p}{,} \PY{p}{(}\PY{l+m+mi}{3}\PY{p}{,} \PY{l+m+mi}{1}\PY{p}{)}\PY{p}{)} \PY{o}{*} \PY{n}{w}\PY{p}{)}
\end{Verbatim}
\end{tcolorbox}

    \begin{Verbatim}[commandchars=\\\{\}]
[[ 4  5]
 [ 8 10]
 [12 15]]
    \end{Verbatim}

    \begin{tcolorbox}[breakable, size=fbox, boxrule=1pt, pad at break*=1mm,colback=cellbackground, colframe=cellborder]
\prompt{In}{incolor}{51}{\boxspacing}
\begin{Verbatim}[commandchars=\\\{\}]
\PY{c+c1}{\PYZsh{} Add a vector to each column of a matrix}
\PY{c+c1}{\PYZsh{} x has shape (2, 3) and w has shape (2,).}
\PY{c+c1}{\PYZsh{} If we transpose x then it has shape (3, 2) and can be broadcast}
\PY{c+c1}{\PYZsh{} against w to yield a result of shape (3, 2); transposing this result}
\PY{c+c1}{\PYZsh{} yields the final result of shape (2, 3) which is the matrix x with}
\PY{c+c1}{\PYZsh{} the vector w added to each column. Gives the following matrix:}
\PY{n}{x} \PY{o}{=} \PY{n}{np}\PY{o}{.}\PY{n}{array}\PY{p}{(}\PY{p}{[}\PY{p}{[}\PY{l+m+mi}{1}\PY{p}{,}\PY{l+m+mi}{2}\PY{p}{,}\PY{l+m+mi}{3}\PY{p}{]}\PY{p}{,} \PY{p}{[}\PY{l+m+mi}{4}\PY{p}{,}\PY{l+m+mi}{5}\PY{p}{,}\PY{l+m+mi}{6}\PY{p}{]}\PY{p}{]}\PY{p}{)}
\PY{n+nb}{print}\PY{p}{(}\PY{l+s+s1}{\PYZsq{}}\PY{l+s+s1}{\PYZhy{}\PYZhy{}\PYZhy{}\PYZhy{}\PYZhy{}\PYZgt{} w array:}\PY{l+s+se}{\PYZbs{}n}\PY{l+s+s1}{\PYZsq{}}\PY{p}{)}
\PY{n+nb}{print}\PY{p}{(}\PY{n}{w}\PY{p}{)}
\PY{n+nb}{print}\PY{p}{(}\PY{l+s+s1}{\PYZsq{}}\PY{l+s+se}{\PYZbs{}n}\PY{l+s+s1}{\PYZhy{}\PYZhy{}\PYZhy{}\PYZhy{}\PYZhy{}\PYZgt{} x array:}\PY{l+s+se}{\PYZbs{}n}\PY{l+s+s1}{\PYZsq{}}\PY{p}{)}
\PY{n+nb}{print}\PY{p}{(}\PY{n}{x}\PY{p}{)}
\PY{n+nb}{print}\PY{p}{(}\PY{l+s+s1}{\PYZsq{}}\PY{l+s+se}{\PYZbs{}n}\PY{l+s+s1}{\PYZhy{}\PYZhy{}\PYZhy{}\PYZhy{}\PYZhy{}\PYZgt{} x transpose:}\PY{l+s+se}{\PYZbs{}n}\PY{l+s+s1}{\PYZsq{}}\PY{p}{)}
\PY{n+nb}{print}\PY{p}{(}\PY{n}{x}\PY{o}{.}\PY{n}{T}\PY{p}{)}
\PY{n+nb}{print}\PY{p}{(}\PY{l+s+s1}{\PYZsq{}}\PY{l+s+se}{\PYZbs{}n}\PY{l+s+s1}{\PYZhy{}\PYZhy{}\PYZhy{}\PYZhy{}\PYZhy{}\PYZgt{} x transpose plus w:}\PY{l+s+se}{\PYZbs{}n}\PY{l+s+s1}{\PYZsq{}}\PY{p}{)}
\PY{n+nb}{print}\PY{p}{(}\PY{n}{x}\PY{o}{.}\PY{n}{T} \PY{o}{+} \PY{n}{w}\PY{p}{)}
\PY{n+nb}{print}\PY{p}{(}\PY{l+s+s1}{\PYZsq{}}\PY{l+s+se}{\PYZbs{}n}\PY{l+s+s1}{\PYZhy{}\PYZhy{}\PYZhy{}\PYZhy{}\PYZhy{}\PYZgt{} final result:}\PY{l+s+se}{\PYZbs{}n}\PY{l+s+s1}{\PYZsq{}}\PY{p}{)}
\PY{n+nb}{print}\PY{p}{(}\PY{p}{(}\PY{n}{x}\PY{o}{.}\PY{n}{T} \PY{o}{+} \PY{n}{w}\PY{p}{)}\PY{o}{.}\PY{n}{T}\PY{p}{)}
\end{Verbatim}
\end{tcolorbox}

    \begin{Verbatim}[commandchars=\\\{\}]
-----> w array:

[4 5]

-----> x array:

[[1 2 3]
 [4 5 6]]

-----> x transpose:

[[1 4]
 [2 5]
 [3 6]]

-----> x transpose plus w:

[[ 5  9]
 [ 6 10]
 [ 7 11]]

-----> final result:

[[ 5  6  7]
 [ 9 10 11]]
    \end{Verbatim}

    \begin{tcolorbox}[breakable, size=fbox, boxrule=1pt, pad at break*=1mm,colback=cellbackground, colframe=cellborder]
\prompt{In}{incolor}{52}{\boxspacing}
\begin{Verbatim}[commandchars=\\\{\}]
\PY{c+c1}{\PYZsh{} Another solution is to reshape w to be a row vector of shape (2, 1);}
\PY{c+c1}{\PYZsh{} we can then broadcast it directly against x to produce the same}
\PY{c+c1}{\PYZsh{} output.}
\PY{n+nb}{print}\PY{p}{(}\PY{n}{x} \PY{o}{+} \PY{n}{np}\PY{o}{.}\PY{n}{reshape}\PY{p}{(}\PY{n}{w}\PY{p}{,} \PY{p}{(}\PY{l+m+mi}{2}\PY{p}{,} \PY{l+m+mi}{1}\PY{p}{)}\PY{p}{)}\PY{p}{)}
\end{Verbatim}
\end{tcolorbox}

    \begin{Verbatim}[commandchars=\\\{\}]
[[ 5  6  7]
 [ 9 10 11]]
    \end{Verbatim}

    \begin{tcolorbox}[breakable, size=fbox, boxrule=1pt, pad at break*=1mm,colback=cellbackground, colframe=cellborder]
\prompt{In}{incolor}{53}{\boxspacing}
\begin{Verbatim}[commandchars=\\\{\}]
\PY{c+c1}{\PYZsh{} Multiply a matrix by a constant:}
\PY{c+c1}{\PYZsh{} x has shape (2, 3). Numpy treats scalars as arrays of shape ();}
\PY{c+c1}{\PYZsh{} these can be broadcast together to shape (2, 3), producing the}
\PY{c+c1}{\PYZsh{} following array:}
\PY{n+nb}{print}\PY{p}{(}\PY{n}{x} \PY{o}{*} \PY{l+m+mi}{2}\PY{p}{)}
\end{Verbatim}
\end{tcolorbox}

    \begin{Verbatim}[commandchars=\\\{\}]
[[ 2  4  6]
 [ 8 10 12]]
    \end{Verbatim}

    This brief overview has touched on many of the important things that you
need to know about numpy, but is far from complete. Check out the
\href{http://docs.scipy.org/doc/numpy/reference/}{numpy reference} to
find out much more about numpy.

    \hypertarget{numpy-in-data-science-machine-learning}{%
\subsubsection{NumPy in Data Science \& Machine
Learning}\label{numpy-in-data-science-machine-learning}}

NumPy is a very popular Python library for large multi-dimensional array
and matrix processing. With the help of a large collection of high-level
mathematical functions it is very useful for fundamental scientific
computations in Machine Learning.

It is particularly useful for,

\begin{itemize}
\tightlist
\item
  Linear Algebra
\item
  Fourier Transform
\item
  Random Number Generations
\end{itemize}

High-end libraries like TensorFlow uses NumPy internally for
manipulation of Tensors.

Lots of ML concepts are tied up with linear algebra. It helps in

\begin{itemize}
\tightlist
\item
  To understand PCA(Principal Component Analysis),
\item
  To build better ML algorithms from scratch,
\item
  For processing Graphics in ML,
\item
  It helps to understand Matrix factorization.
\end{itemize}

In fact, it could be said that ML completely uses matrix operations. The
Linear Algebra module of NumPy offers various methods to apply linear
algebra on any NumPy array. One can find:

\begin{itemize}
\tightlist
\item
  Rank, determinant, transpose, trace, inverse, etc. of an array.
\item
  Eigenvalues and eigenvectors of the given matrices
\item
  The dot product of two scalar values, as well as vector values.
\item
  Solve a linear matrix equation and much more!
\end{itemize}

    \textbf{Example} Calculating the inverse of a matrix

    \begin{tcolorbox}[breakable, size=fbox, boxrule=1pt, pad at break*=1mm,colback=cellbackground, colframe=cellborder]
\prompt{In}{incolor}{54}{\boxspacing}
\begin{Verbatim}[commandchars=\\\{\}]
\PY{n}{array} \PY{o}{=} \PY{n}{np}\PY{o}{.}\PY{n}{array}\PY{p}{(}\PY{p}{[}
    \PY{p}{[}\PY{l+m+mi}{6}\PY{p}{,} \PY{l+m+mi}{1}\PY{p}{,} \PY{l+m+mi}{1}\PY{p}{]}\PY{p}{,}
    \PY{p}{[}\PY{l+m+mi}{4}\PY{p}{,} \PY{o}{\PYZhy{}}\PY{l+m+mi}{2}\PY{p}{,} \PY{l+m+mi}{5}\PY{p}{]}\PY{p}{,}
    \PY{p}{[}\PY{l+m+mi}{2}\PY{p}{,} \PY{l+m+mi}{8}\PY{p}{,} \PY{l+m+mi}{7}\PY{p}{]}
\PY{p}{]}\PY{p}{)}

\PY{n}{inverse} \PY{o}{=} \PY{n}{np}\PY{o}{.}\PY{n}{linalg}\PY{o}{.}\PY{n}{inv}\PY{p}{(}\PY{n}{array}\PY{p}{)}
\PY{n+nb}{print}\PY{p}{(}\PY{n}{inverse}\PY{p}{)}
\end{Verbatim}
\end{tcolorbox}

    \begin{Verbatim}[commandchars=\\\{\}]
[[ 0.17647059 -0.00326797 -0.02287582]
 [ 0.05882353 -0.13071895  0.08496732]
 [-0.11764706  0.1503268   0.05228758]]
    \end{Verbatim}

    \begin{tcolorbox}[breakable, size=fbox, boxrule=1pt, pad at break*=1mm,colback=cellbackground, colframe=cellborder]
\prompt{In}{incolor}{55}{\boxspacing}
\begin{Verbatim}[commandchars=\\\{\}]
\PY{n+nb}{print}\PY{p}{(}\PY{n}{np}\PY{o}{.}\PY{n}{round}\PY{p}{(}\PY{n}{array}\PY{o}{.}\PY{n}{dot}\PY{p}{(}\PY{n}{inverse}\PY{p}{)}\PY{p}{,} \PY{l+m+mi}{8}\PY{p}{)}\PY{p}{)}
\end{Verbatim}
\end{tcolorbox}

    \begin{Verbatim}[commandchars=\\\{\}]
[[ 1.  0.  0.]
 [-0.  1.  0.]
 [-0.  0.  1.]]
    \end{Verbatim}

    \textbf{Example} Find eigenvalues and eigenvectors

    \begin{tcolorbox}[breakable, size=fbox, boxrule=1pt, pad at break*=1mm,colback=cellbackground, colframe=cellborder]
\prompt{In}{incolor}{56}{\boxspacing}
\begin{Verbatim}[commandchars=\\\{\}]
\PY{n}{eigenVal}\PY{p}{,} \PY{n}{eigenVec} \PY{o}{=} \PY{n}{np}\PY{o}{.}\PY{n}{linalg}\PY{o}{.}\PY{n}{eig}\PY{p}{(}\PY{n}{array}\PY{p}{)}
\PY{n+nb}{print}\PY{p}{(}\PY{n}{eigenVal}\PY{p}{)}
\PY{n+nb}{print}\PY{p}{(}\PY{n}{eigenVec}\PY{p}{)}
\end{Verbatim}
\end{tcolorbox}

    \begin{Verbatim}[commandchars=\\\{\}]
[11.24862343  5.09285054 -5.34147398]
[[ 0.24511338  0.75669314  0.02645665]
 [ 0.40622202 -0.03352363 -0.84078293]
 [ 0.88028581 -0.65291014  0.54072554]]
    \end{Verbatim}

    \textbf{Example} Solve a linear matrix equation

    \begin{tcolorbox}[breakable, size=fbox, boxrule=1pt, pad at break*=1mm,colback=cellbackground, colframe=cellborder]
\prompt{In}{incolor}{57}{\boxspacing}
\begin{Verbatim}[commandchars=\\\{\}]
\PY{n}{A} \PY{o}{=} \PY{n}{np}\PY{o}{.}\PY{n}{array}\PY{p}{(}\PY{p}{[}
    \PY{p}{[}\PY{l+m+mi}{1}\PY{p}{,} \PY{l+m+mi}{3}\PY{p}{]}\PY{p}{,}
    \PY{p}{[}\PY{l+m+mi}{2}\PY{p}{,} \PY{l+m+mi}{4}\PY{p}{]}
\PY{p}{]}\PY{p}{)}

\PY{n}{b} \PY{o}{=} \PY{n}{np}\PY{o}{.}\PY{n}{array}\PY{p}{(}\PY{p}{[}
    \PY{p}{[}\PY{l+m+mi}{7}\PY{p}{]}\PY{p}{,}
    \PY{p}{[}\PY{l+m+mi}{10}\PY{p}{]}
\PY{p}{]}\PY{p}{)}

\PY{n}{x} \PY{o}{=} \PY{n}{np}\PY{o}{.}\PY{n}{linalg}\PY{o}{.}\PY{n}{solve}\PY{p}{(}\PY{n}{A}\PY{p}{,} \PY{n}{b}\PY{p}{)}
\PY{n+nb}{print}\PY{p}{(}\PY{n}{x}\PY{p}{)}
\end{Verbatim}
\end{tcolorbox}

    \begin{Verbatim}[commandchars=\\\{\}]
[[1.]
 [2.]]
    \end{Verbatim}

    \hypertarget{intro-to-matplotlib}{%
\subsection{Intro to Matplotlib}\label{intro-to-matplotlib}}

    Matplotlib is a plotting library. In this section give a brief
introduction to the \texttt{matplotlib.pyplot} module, which provides a
plotting system similar to that of MATLAB.

    \begin{tcolorbox}[breakable, size=fbox, boxrule=1pt, pad at break*=1mm,colback=cellbackground, colframe=cellborder]
\prompt{In}{incolor}{58}{\boxspacing}
\begin{Verbatim}[commandchars=\\\{\}]
\PY{k+kn}{import} \PY{n+nn}{matplotlib}\PY{n+nn}{.}\PY{n+nn}{pyplot} \PY{k}{as} \PY{n+nn}{plt}
\end{Verbatim}
\end{tcolorbox}

    By running this special iPython command, we will be displaying plots
inline:

    \begin{tcolorbox}[breakable, size=fbox, boxrule=1pt, pad at break*=1mm,colback=cellbackground, colframe=cellborder]
\prompt{In}{incolor}{59}{\boxspacing}
\begin{Verbatim}[commandchars=\\\{\}]
\PY{o}{\PYZpc{}}\PY{k}{matplotlib} inline
\end{Verbatim}
\end{tcolorbox}

    \hypertarget{plotting}{%
\subsubsection{Plotting}\label{plotting}}

    The most important function in \texttt{matplotlib} is plot, which allows
you to plot 2D data. Here is a simple example:

    \begin{tcolorbox}[breakable, size=fbox, boxrule=1pt, pad at break*=1mm,colback=cellbackground, colframe=cellborder]
\prompt{In}{incolor}{60}{\boxspacing}
\begin{Verbatim}[commandchars=\\\{\}]
\PY{c+c1}{\PYZsh{} Compute the x and y coordinates for points on a sine curve}
\PY{n}{x} \PY{o}{=} \PY{n}{np}\PY{o}{.}\PY{n}{arange}\PY{p}{(}\PY{l+m+mi}{0}\PY{p}{,} \PY{l+m+mi}{3} \PY{o}{*} \PY{n}{np}\PY{o}{.}\PY{n}{pi}\PY{p}{,} \PY{l+m+mf}{0.1}\PY{p}{)}
\PY{n}{y} \PY{o}{=} \PY{n}{np}\PY{o}{.}\PY{n}{sin}\PY{p}{(}\PY{n}{x}\PY{p}{)}

\PY{c+c1}{\PYZsh{} Plot the points using matplotlib}
\PY{n}{plt}\PY{o}{.}\PY{n}{plot}\PY{p}{(}\PY{n}{x}\PY{p}{,} \PY{n}{y}\PY{p}{)}
\end{Verbatim}
\end{tcolorbox}

            \begin{tcolorbox}[breakable, size=fbox, boxrule=.5pt, pad at break*=1mm, opacityfill=0]
\prompt{Out}{outcolor}{60}{\boxspacing}
\begin{Verbatim}[commandchars=\\\{\}]
[<matplotlib.lines.Line2D at 0x1cb4b16f208>]
\end{Verbatim}
\end{tcolorbox}
        
    \begin{center}
    \adjustimage{max size={0.9\linewidth}{0.9\paperheight}}{output_100_1.png}
    \end{center}
    { \hspace*{\fill} \\}
    
    With just a little bit of extra work we can easily plot multiple lines
at once, and add a title, legend, and axis labels:

    \begin{tcolorbox}[breakable, size=fbox, boxrule=1pt, pad at break*=1mm,colback=cellbackground, colframe=cellborder]
\prompt{In}{incolor}{61}{\boxspacing}
\begin{Verbatim}[commandchars=\\\{\}]
\PY{n}{y\PYZus{}sin} \PY{o}{=} \PY{n}{np}\PY{o}{.}\PY{n}{sin}\PY{p}{(}\PY{n}{x}\PY{p}{)}
\PY{n}{y\PYZus{}cos} \PY{o}{=} \PY{n}{np}\PY{o}{.}\PY{n}{cos}\PY{p}{(}\PY{n}{x}\PY{p}{)}

\PY{c+c1}{\PYZsh{} Plot the points using matplotlib}
\PY{n}{plt}\PY{o}{.}\PY{n}{plot}\PY{p}{(}\PY{n}{x}\PY{p}{,} \PY{n}{y\PYZus{}sin}\PY{p}{)}
\PY{n}{plt}\PY{o}{.}\PY{n}{plot}\PY{p}{(}\PY{n}{x}\PY{p}{,} \PY{n}{y\PYZus{}cos}\PY{p}{)}
\PY{n}{plt}\PY{o}{.}\PY{n}{xlabel}\PY{p}{(}\PY{l+s+s1}{\PYZsq{}}\PY{l+s+s1}{x axis label}\PY{l+s+s1}{\PYZsq{}}\PY{p}{)}
\PY{n}{plt}\PY{o}{.}\PY{n}{ylabel}\PY{p}{(}\PY{l+s+s1}{\PYZsq{}}\PY{l+s+s1}{y axis label}\PY{l+s+s1}{\PYZsq{}}\PY{p}{)}
\PY{n}{plt}\PY{o}{.}\PY{n}{title}\PY{p}{(}\PY{l+s+s1}{\PYZsq{}}\PY{l+s+s1}{Sine and Cosine}\PY{l+s+s1}{\PYZsq{}}\PY{p}{)}
\PY{n}{plt}\PY{o}{.}\PY{n}{legend}\PY{p}{(}\PY{p}{[}\PY{l+s+s1}{\PYZsq{}}\PY{l+s+s1}{Sine}\PY{l+s+s1}{\PYZsq{}}\PY{p}{,} \PY{l+s+s1}{\PYZsq{}}\PY{l+s+s1}{Cosine}\PY{l+s+s1}{\PYZsq{}}\PY{p}{]}\PY{p}{)}
\end{Verbatim}
\end{tcolorbox}

            \begin{tcolorbox}[breakable, size=fbox, boxrule=.5pt, pad at break*=1mm, opacityfill=0]
\prompt{Out}{outcolor}{61}{\boxspacing}
\begin{Verbatim}[commandchars=\\\{\}]
<matplotlib.legend.Legend at 0x1cb4b1cbac8>
\end{Verbatim}
\end{tcolorbox}
        
    \begin{center}
    \adjustimage{max size={0.9\linewidth}{0.9\paperheight}}{output_102_1.png}
    \end{center}
    { \hspace*{\fill} \\}
    
    \hypertarget{subplots}{%
\subsubsection{Subplots}\label{subplots}}

    You can plot different things in the same figure using the subplot
function. Here is an example:

    \begin{tcolorbox}[breakable, size=fbox, boxrule=1pt, pad at break*=1mm,colback=cellbackground, colframe=cellborder]
\prompt{In}{incolor}{62}{\boxspacing}
\begin{Verbatim}[commandchars=\\\{\}]
\PY{c+c1}{\PYZsh{} Compute the x and y coordinates for points on sine and cosine curves}
\PY{n}{x} \PY{o}{=} \PY{n}{np}\PY{o}{.}\PY{n}{arange}\PY{p}{(}\PY{l+m+mi}{0}\PY{p}{,} \PY{l+m+mi}{3} \PY{o}{*} \PY{n}{np}\PY{o}{.}\PY{n}{pi}\PY{p}{,} \PY{l+m+mf}{0.1}\PY{p}{)}
\PY{n}{y\PYZus{}sin} \PY{o}{=} \PY{n}{np}\PY{o}{.}\PY{n}{sin}\PY{p}{(}\PY{n}{x}\PY{p}{)}
\PY{n}{y\PYZus{}cos} \PY{o}{=} \PY{n}{np}\PY{o}{.}\PY{n}{cos}\PY{p}{(}\PY{n}{x}\PY{p}{)}

\PY{c+c1}{\PYZsh{} Set up a subplot grid that has height 2 and width 1,}
\PY{c+c1}{\PYZsh{} and set the first such subplot as active.}
\PY{n}{plt}\PY{o}{.}\PY{n}{subplot}\PY{p}{(}\PY{l+m+mi}{2}\PY{p}{,} \PY{l+m+mi}{1}\PY{p}{,} \PY{l+m+mi}{1}\PY{p}{)}

\PY{c+c1}{\PYZsh{} Make the first plot}
\PY{n}{plt}\PY{o}{.}\PY{n}{plot}\PY{p}{(}\PY{n}{x}\PY{p}{,} \PY{n}{y\PYZus{}sin}\PY{p}{)}
\PY{n}{plt}\PY{o}{.}\PY{n}{title}\PY{p}{(}\PY{l+s+s1}{\PYZsq{}}\PY{l+s+s1}{Sine}\PY{l+s+s1}{\PYZsq{}}\PY{p}{)}

\PY{c+c1}{\PYZsh{} Set the second subplot as active, and make the second plot.}
\PY{n}{plt}\PY{o}{.}\PY{n}{subplot}\PY{p}{(}\PY{l+m+mi}{2}\PY{p}{,} \PY{l+m+mi}{1}\PY{p}{,} \PY{l+m+mi}{2}\PY{p}{)}
\PY{n}{plt}\PY{o}{.}\PY{n}{plot}\PY{p}{(}\PY{n}{x}\PY{p}{,} \PY{n}{y\PYZus{}cos}\PY{p}{)}
\PY{n}{plt}\PY{o}{.}\PY{n}{title}\PY{p}{(}\PY{l+s+s1}{\PYZsq{}}\PY{l+s+s1}{Cosine}\PY{l+s+s1}{\PYZsq{}}\PY{p}{)}

\PY{c+c1}{\PYZsh{} Show the figure.}
\PY{n}{plt}\PY{o}{.}\PY{n}{show}\PY{p}{(}\PY{p}{)}
\end{Verbatim}
\end{tcolorbox}

    \begin{center}
    \adjustimage{max size={0.9\linewidth}{0.9\paperheight}}{output_105_0.png}
    \end{center}
    { \hspace*{\fill} \\}
    
    You can read much more about the \texttt{subplot} function in the
\href{http://matplotlib.org/api/pyplot_api.html\#matplotlib.pyplot.subplot}{documentation}.

    \hypertarget{basic-pandas}{%
\subsection{Basic Pandas}\label{basic-pandas}}

    The pandas package is probably the most important tool at the disposal
of Data Scientists and Analysts working in Python today. The powerful
machine learning and glamorous visualization tools may get all the
attention, but pandas is the backbone of most data projects. To import
pandas we usually import it with a shorter name since it's used so much:

    \begin{tcolorbox}[breakable, size=fbox, boxrule=1pt, pad at break*=1mm,colback=cellbackground, colframe=cellborder]
\prompt{In}{incolor}{63}{\boxspacing}
\begin{Verbatim}[commandchars=\\\{\}]
\PY{k+kn}{import} \PY{n+nn}{pandas} \PY{k}{as} \PY{n+nn}{pd}
\end{Verbatim}
\end{tcolorbox}

    Now to the basic components of pandas.

    \hypertarget{core-components-of-pandas-series-and-dataframes}{%
\subsubsection{Core components of pandas: Series and
DataFrames}\label{core-components-of-pandas-series-and-dataframes}}

The primary two components of pandas are the \texttt{Series} and
\texttt{DataFrame}.

A \texttt{Series} is essentially a column, and a \texttt{DataFrame} is a
multi-dimensional table made up of a collection of Series.

DataFrames and Series are quite similar in that many operations that you
can do with one you can do with the other, such as filling in null
values and calculating the mean.

You'll see how these components work when we start working with data
below.

    \hypertarget{creating-dataframes-from-scratch}{%
\subsubsection{Creating DataFrames from
scratch}\label{creating-dataframes-from-scratch}}

Creating DataFrames right in Python is good to know and quite useful
when testing new methods and functions you find in the pandas docs.

There are \emph{many} ways to create a DataFrame from scratch, a first
option is to just use a simple \texttt{dict}.

    \begin{tcolorbox}[breakable, size=fbox, boxrule=1pt, pad at break*=1mm,colback=cellbackground, colframe=cellborder]
\prompt{In}{incolor}{64}{\boxspacing}
\begin{Verbatim}[commandchars=\\\{\}]
\PY{n}{data} \PY{o}{=} \PY{p}{\PYZob{}}
    \PY{l+s+s1}{\PYZsq{}}\PY{l+s+s1}{Open}\PY{l+s+s1}{\PYZsq{}}  \PY{p}{:}\PY{p}{[}\PY{l+m+mf}{1.20575}\PY{p}{,}\PY{l+m+mf}{1.20566}\PY{p}{,}\PY{l+m+mf}{1.20582}\PY{p}{,}\PY{l+m+mf}{1.20574}\PY{p}{,}\PY{l+m+mf}{1.20596}\PY{p}{,}\PY{l+m+mf}{1.20590}\PY{p}{]}\PY{p}{,} 
    \PY{l+s+s1}{\PYZsq{}}\PY{l+s+s1}{High}\PY{l+s+s1}{\PYZsq{}}  \PY{p}{:}\PY{p}{[}\PY{l+m+mf}{1.20576}\PY{p}{,}\PY{l+m+mf}{1.20586}\PY{p}{,}\PY{l+m+mf}{1.20592}\PY{p}{,}\PY{l+m+mf}{1.20601}\PY{p}{,}\PY{l+m+mf}{1.20615}\PY{p}{,}\PY{l+m+mf}{1.20593}\PY{p}{]}\PY{p}{,}
    \PY{l+s+s1}{\PYZsq{}}\PY{l+s+s1}{Low}\PY{l+s+s1}{\PYZsq{}}   \PY{p}{:}\PY{p}{[}\PY{l+m+mf}{1.20560}\PY{p}{,}\PY{l+m+mf}{1.20565}\PY{p}{,}\PY{l+m+mf}{1.20571}\PY{p}{,}\PY{l+m+mf}{1.20569}\PY{p}{,}\PY{l+m+mf}{1.20582}\PY{p}{,}\PY{l+m+mf}{1.20580}\PY{p}{]}\PY{p}{,}
    \PY{l+s+s1}{\PYZsq{}}\PY{l+s+s1}{Close}\PY{l+s+s1}{\PYZsq{}} \PY{p}{:}\PY{p}{[}\PY{l+m+mf}{1.20566}\PY{p}{,}\PY{l+m+mf}{1.20582}\PY{p}{,}\PY{l+m+mf}{1.20572}\PY{p}{,}\PY{l+m+mf}{1.20597}\PY{p}{,}\PY{l+m+mf}{1.20592}\PY{p}{,}\PY{l+m+mf}{1.20588}\PY{p}{]}\PY{p}{,}
    \PY{l+s+s1}{\PYZsq{}}\PY{l+s+s1}{Volume}\PY{l+s+s1}{\PYZsq{}}\PY{p}{:}\PY{p}{[}\PY{l+m+mi}{212}\PY{p}{,}\PY{l+m+mi}{88}\PY{p}{,}\PY{l+m+mi}{83}\PY{p}{,}\PY{l+m+mi}{184}\PY{p}{,}\PY{l+m+mi}{246}\PY{p}{,}\PY{l+m+mi}{131}\PY{p}{]}
\PY{p}{\PYZcb{}}
\end{Verbatim}
\end{tcolorbox}

    And then pass it to the pandas DataFrame constructor:

    \begin{tcolorbox}[breakable, size=fbox, boxrule=1pt, pad at break*=1mm,colback=cellbackground, colframe=cellborder]
\prompt{In}{incolor}{65}{\boxspacing}
\begin{Verbatim}[commandchars=\\\{\}]
\PY{n}{fx\PYZus{}eur\PYZus{}usd} \PY{o}{=} \PY{n}{pd}\PY{o}{.}\PY{n}{DataFrame}\PY{p}{(}\PY{n}{data}\PY{p}{)}
\PY{n}{fx\PYZus{}eur\PYZus{}usd}
\end{Verbatim}
\end{tcolorbox}

            \begin{tcolorbox}[breakable, size=fbox, boxrule=.5pt, pad at break*=1mm, opacityfill=0]
\prompt{Out}{outcolor}{65}{\boxspacing}
\begin{Verbatim}[commandchars=\\\{\}]
      Open     High      Low    Close  Volume
0  1.20575  1.20576  1.20560  1.20566     212
1  1.20566  1.20586  1.20565  1.20582      88
2  1.20582  1.20592  1.20571  1.20572      83
3  1.20574  1.20601  1.20569  1.20597     184
4  1.20596  1.20615  1.20582  1.20592     246
5  1.20590  1.20593  1.20580  1.20588     131
\end{Verbatim}
\end{tcolorbox}
        
    \textbf{How did that work?}

Each \emph{(key, value)} item in \texttt{data} corresponds to a
\emph{column} in the resulting DataFrame.

The \textbf{Index} of this DataFrame was given to us on creation as the
numbers 0-3, but we could also create our own when we initialize the
DataFrame.

Let's have time as our index:

    \begin{tcolorbox}[breakable, size=fbox, boxrule=1pt, pad at break*=1mm,colback=cellbackground, colframe=cellborder]
\prompt{In}{incolor}{66}{\boxspacing}
\begin{Verbatim}[commandchars=\\\{\}]
\PY{n}{fx\PYZus{}eur\PYZus{}usd}\PY{o}{=}\PY{n}{pd}\PY{o}{.}\PY{n}{DataFrame}\PY{p}{(}\PY{n}{data}\PY{p}{,} \PY{n}{index}\PY{o}{=}\PY{p}{[}\PY{l+s+s1}{\PYZsq{}}\PY{l+s+s1}{08/02/2021 15:25}\PY{l+s+s1}{\PYZsq{}}\PY{p}{,} \PYZbs{}
                                     \PY{l+s+s1}{\PYZsq{}}\PY{l+s+s1}{08/02/2021 15:26}\PY{l+s+s1}{\PYZsq{}}\PY{p}{,} \PYZbs{}
                                     \PY{l+s+s1}{\PYZsq{}}\PY{l+s+s1}{08/02/2021 15:27}\PY{l+s+s1}{\PYZsq{}}\PY{p}{,} \PYZbs{}
                                     \PY{l+s+s1}{\PYZsq{}}\PY{l+s+s1}{08/02/2021 15:28}\PY{l+s+s1}{\PYZsq{}}\PY{p}{,} \PYZbs{}
                                     \PY{l+s+s1}{\PYZsq{}}\PY{l+s+s1}{08/02/2021 15:29}\PY{l+s+s1}{\PYZsq{}}\PY{p}{,} \PYZbs{}
                                     \PY{l+s+s1}{\PYZsq{}}\PY{l+s+s1}{08/02/2021 15:30}\PY{l+s+s1}{\PYZsq{}}\PY{p}{]}\PY{p}{)}
\PY{n}{fx\PYZus{}eur\PYZus{}usd}
\end{Verbatim}
\end{tcolorbox}

            \begin{tcolorbox}[breakable, size=fbox, boxrule=.5pt, pad at break*=1mm, opacityfill=0]
\prompt{Out}{outcolor}{66}{\boxspacing}
\begin{Verbatim}[commandchars=\\\{\}]
                     Open     High      Low    Close  Volume
08/02/2021 15:25  1.20575  1.20576  1.20560  1.20566     212
08/02/2021 15:26  1.20566  1.20586  1.20565  1.20582      88
08/02/2021 15:27  1.20582  1.20592  1.20571  1.20572      83
08/02/2021 15:28  1.20574  1.20601  1.20569  1.20597     184
08/02/2021 15:29  1.20596  1.20615  1.20582  1.20592     246
08/02/2021 15:30  1.20590  1.20593  1.20580  1.20588     131
\end{Verbatim}
\end{tcolorbox}
        
    So now we could \textbf{loc}ate a price by using their time:

    \begin{tcolorbox}[breakable, size=fbox, boxrule=1pt, pad at break*=1mm,colback=cellbackground, colframe=cellborder]
\prompt{In}{incolor}{67}{\boxspacing}
\begin{Verbatim}[commandchars=\\\{\}]
\PY{n}{fx\PYZus{}eur\PYZus{}usd}\PY{o}{.}\PY{n}{loc}\PY{p}{[}\PY{l+s+s1}{\PYZsq{}}\PY{l+s+s1}{08/02/2021 15:27}\PY{l+s+s1}{\PYZsq{}}\PY{p}{]}
\end{Verbatim}
\end{tcolorbox}

            \begin{tcolorbox}[breakable, size=fbox, boxrule=.5pt, pad at break*=1mm, opacityfill=0]
\prompt{Out}{outcolor}{67}{\boxspacing}
\begin{Verbatim}[commandchars=\\\{\}]
Open       1.20582
High       1.20592
Low        1.20571
Close      1.20572
Volume    83.00000
Name: 08/02/2021 15:27, dtype: float64
\end{Verbatim}
\end{tcolorbox}
        
    There's more on locating and extracting data from the DataFrame later,
but now you should be able to create a DataFrame with any random data to
learn on.

Let's move on to some quick methods for creating DataFrames from various
other sources.

    \hypertarget{how-to-read-in-data}{%
\subsubsection{How to read in data}\label{how-to-read-in-data}}

It's quite simple to load data from various file formats into a
DataFrame. In the following examples we'll keep using our eur/usd forex
data, but this time it's coming from various files.

    \hypertarget{reading-data-from-csvs}{%
\paragraph{Reading data from CSVs}\label{reading-data-from-csvs}}

With CSV files all you need is a single line to load in the data:

    \begin{tcolorbox}[breakable, size=fbox, boxrule=1pt, pad at break*=1mm,colback=cellbackground, colframe=cellborder]
\prompt{In}{incolor}{68}{\boxspacing}
\begin{Verbatim}[commandchars=\\\{\}]
\PY{n}{path} \PY{o}{=} \PY{l+s+s1}{\PYZsq{}}\PY{l+s+s1}{./data/csv/}\PY{l+s+s1}{\PYZsq{}}

\PY{n}{df} \PY{o}{=} \PY{n}{pd}\PY{o}{.}\PY{n}{read\PYZus{}csv}\PY{p}{(}\PY{n}{path} \PY{o}{+} \PY{l+s+s1}{\PYZsq{}}\PY{l+s+s1}{EURUSD\PYZus{}M1.csv}\PY{l+s+s1}{\PYZsq{}}\PY{p}{)}
\PY{n}{df}
\end{Verbatim}
\end{tcolorbox}

            \begin{tcolorbox}[breakable, size=fbox, boxrule=.5pt, pad at break*=1mm, opacityfill=0]
\prompt{Out}{outcolor}{68}{\boxspacing}
\begin{Verbatim}[commandchars=\\\{\}]
                    Time\textbackslash{}tOpen\textbackslash{}tHigh\textbackslash{}tLow\textbackslash{}tClose\textbackslash{}tVolume
0      2021-02-08 15:25:00\textbackslash{}t1.20575\textbackslash{}t1.20576\textbackslash{}t1.2056\textbackslash{}{\ldots}
1      2021-02-08 15:26:00\textbackslash{}t1.20566\textbackslash{}t1.20586\textbackslash{}t1.20565{\ldots}
2      2021-02-08 15:27:00\textbackslash{}t1.20582\textbackslash{}t1.20592\textbackslash{}t1.20571{\ldots}
3      2021-02-08 15:28:00\textbackslash{}t1.20574\textbackslash{}t1.20601\textbackslash{}t1.20569{\ldots}
4      2021-02-08 15:29:00\textbackslash{}t1.20596\textbackslash{}t1.20615\textbackslash{}t1.20582{\ldots}
{\ldots}                                                  {\ldots}
49995  2021-03-29 12:55:00\textbackslash{}t1.17814\textbackslash{}t1.17827\textbackslash{}t1.17812{\ldots}
49996  2021-03-29 12:56:00\textbackslash{}t1.17824\textbackslash{}t1.17825\textbackslash{}t1.17819{\ldots}
49997  2021-03-29 12:57:00\textbackslash{}t1.17825\textbackslash{}t1.17847\textbackslash{}t1.1781\textbackslash{}{\ldots}
49998  2021-03-29 12:58:00\textbackslash{}t1.17846\textbackslash{}t1.17855\textbackslash{}t1.17839{\ldots}
49999  2021-03-29 12:59:00\textbackslash{}t1.17858\textbackslash{}t1.17859\textbackslash{}t1.17844{\ldots}

[50000 rows x 1 columns]
\end{Verbatim}
\end{tcolorbox}
        
    \begin{tcolorbox}[breakable, size=fbox, boxrule=1pt, pad at break*=1mm,colback=cellbackground, colframe=cellborder]
\prompt{In}{incolor}{69}{\boxspacing}
\begin{Verbatim}[commandchars=\\\{\}]
\PY{n}{df} \PY{o}{=} \PY{n}{pd}\PY{o}{.}\PY{n}{read\PYZus{}csv}\PY{p}{(}\PY{n}{path} \PY{o}{+} \PY{l+s+s1}{\PYZsq{}}\PY{l+s+s1}{EURUSD\PYZus{}M1.csv}\PY{l+s+s1}{\PYZsq{}}\PY{p}{,} \PY{n}{sep} \PY{o}{=} \PY{l+s+s1}{\PYZsq{}}\PY{l+s+se}{\PYZbs{}t}\PY{l+s+s1}{\PYZsq{}}\PY{p}{)}
\PY{n}{df}
\end{Verbatim}
\end{tcolorbox}

            \begin{tcolorbox}[breakable, size=fbox, boxrule=.5pt, pad at break*=1mm, opacityfill=0]
\prompt{Out}{outcolor}{69}{\boxspacing}
\begin{Verbatim}[commandchars=\\\{\}]
                      Time     Open     High      Low    Close  Volume
0      2021-02-08 15:25:00  1.20575  1.20576  1.20560  1.20566     212
1      2021-02-08 15:26:00  1.20566  1.20586  1.20565  1.20582      88
2      2021-02-08 15:27:00  1.20582  1.20592  1.20571  1.20572      83
3      2021-02-08 15:28:00  1.20574  1.20601  1.20569  1.20597     184
4      2021-02-08 15:29:00  1.20596  1.20615  1.20582  1.20592     246
{\ldots}                    {\ldots}      {\ldots}      {\ldots}      {\ldots}      {\ldots}     {\ldots}
49995  2021-03-29 12:55:00  1.17814  1.17827  1.17812  1.17823      88
49996  2021-03-29 12:56:00  1.17824  1.17825  1.17819  1.17825      58
49997  2021-03-29 12:57:00  1.17825  1.17847  1.17810  1.17845     109
49998  2021-03-29 12:58:00  1.17846  1.17855  1.17839  1.17855      92
49999  2021-03-29 12:59:00  1.17858  1.17859  1.17844  1.17855     154

[50000 rows x 6 columns]
\end{Verbatim}
\end{tcolorbox}
        
    CSVs don't have indexes like our DataFrames, so all we need to do is
just designate the \texttt{index\_col} when reading:

    \begin{tcolorbox}[breakable, size=fbox, boxrule=1pt, pad at break*=1mm,colback=cellbackground, colframe=cellborder]
\prompt{In}{incolor}{70}{\boxspacing}
\begin{Verbatim}[commandchars=\\\{\}]
\PY{n}{df} \PY{o}{=} \PY{n}{pd}\PY{o}{.}\PY{n}{read\PYZus{}csv}\PY{p}{(}\PY{n}{path} \PY{o}{+} \PY{l+s+s1}{\PYZsq{}}\PY{l+s+s1}{EURUSD\PYZus{}M1.csv}\PY{l+s+s1}{\PYZsq{}}\PY{p}{,} \PY{n}{sep} \PY{o}{=} \PY{l+s+s1}{\PYZsq{}}\PY{l+s+se}{\PYZbs{}t}\PY{l+s+s1}{\PYZsq{}}\PY{p}{,} \PY{n}{index\PYZus{}col}\PY{o}{=}\PY{l+m+mi}{0}\PY{p}{)}
\PY{n}{df}
\end{Verbatim}
\end{tcolorbox}

            \begin{tcolorbox}[breakable, size=fbox, boxrule=.5pt, pad at break*=1mm, opacityfill=0]
\prompt{Out}{outcolor}{70}{\boxspacing}
\begin{Verbatim}[commandchars=\\\{\}]
                        Open     High      Low    Close  Volume
Time
2021-02-08 15:25:00  1.20575  1.20576  1.20560  1.20566     212
2021-02-08 15:26:00  1.20566  1.20586  1.20565  1.20582      88
2021-02-08 15:27:00  1.20582  1.20592  1.20571  1.20572      83
2021-02-08 15:28:00  1.20574  1.20601  1.20569  1.20597     184
2021-02-08 15:29:00  1.20596  1.20615  1.20582  1.20592     246
{\ldots}                      {\ldots}      {\ldots}      {\ldots}      {\ldots}     {\ldots}
2021-03-29 12:55:00  1.17814  1.17827  1.17812  1.17823      88
2021-03-29 12:56:00  1.17824  1.17825  1.17819  1.17825      58
2021-03-29 12:57:00  1.17825  1.17847  1.17810  1.17845     109
2021-03-29 12:58:00  1.17846  1.17855  1.17839  1.17855      92
2021-03-29 12:59:00  1.17858  1.17859  1.17844  1.17855     154

[50000 rows x 5 columns]
\end{Verbatim}
\end{tcolorbox}
        
    \hypertarget{most-important-dataframe-operations}{%
\subsubsection{Most important DataFrame
operations}\label{most-important-dataframe-operations}}

DataFrames possess hundreds of methods and other operations that are
crucial to any analysis. As a beginner, you should know the operations
that perform simple transformations of your data and those that provide
fundamental statistical analysis.

Let's load in the IMDB movies dataset to begin:

    \begin{tcolorbox}[breakable, size=fbox, boxrule=1pt, pad at break*=1mm,colback=cellbackground, colframe=cellborder]
\prompt{In}{incolor}{71}{\boxspacing}
\begin{Verbatim}[commandchars=\\\{\}]
\PY{n}{movies\PYZus{}df} \PY{o}{=} \PY{n}{pd}\PY{o}{.}\PY{n}{read\PYZus{}csv}\PY{p}{(}\PY{n}{path} \PY{o}{+} \PY{l+s+s2}{\PYZdq{}}\PY{l+s+s2}{IMDB\PYZhy{}Movie\PYZhy{}Data.csv}\PY{l+s+s2}{\PYZdq{}}\PY{p}{,} \PY{n}{index\PYZus{}col}\PY{o}{=}\PY{l+s+s2}{\PYZdq{}}\PY{l+s+s2}{Title}\PY{l+s+s2}{\PYZdq{}}\PY{p}{)}
\end{Verbatim}
\end{tcolorbox}

    We're loading this dataset from a CSV and designating the movie titles
to be our index.

    \hypertarget{viewing-your-data}{%
\paragraph{Viewing your data}\label{viewing-your-data}}

The first thing to do when opening a new dataset is print out a few rows
to keep as a visual reference. We accomplish this with \texttt{.head()}:

    \begin{tcolorbox}[breakable, size=fbox, boxrule=1pt, pad at break*=1mm,colback=cellbackground, colframe=cellborder]
\prompt{In}{incolor}{1}{\boxspacing}
\begin{Verbatim}[commandchars=\\\{\}]
\PY{n}{movies\PYZus{}df}\PY{o}{.}\PY{n}{head}\PY{p}{(}\PY{p}{)}
\end{Verbatim}
\end{tcolorbox}

    \begin{Verbatim}[commandchars=\\\{\}, frame=single, framerule=2mm, rulecolor=\color{outerrorbackground}]
\textcolor{ansi-red-intense}{\textbf{---------------------------------------------------------------------------}}
\textcolor{ansi-red-intense}{\textbf{NameError}}                                 Traceback (most recent call last)
\textcolor{ansi-green-intense}{\textbf{<ipython-input-1-b687c1d924a0>}} in \textcolor{ansi-cyan}{<module>}
\textcolor{ansi-green-intense}{\textbf{----> 1}}\textcolor{ansi-yellow-intense}{\textbf{ }}movies\_df\textcolor{ansi-yellow-intense}{\textbf{.}}head\textcolor{ansi-yellow-intense}{\textbf{(}}\textcolor{ansi-yellow-intense}{\textbf{)}}

\textcolor{ansi-red-intense}{\textbf{NameError}}: name 'movies\_df' is not defined
    \end{Verbatim}

    \texttt{.head()} outputs the \textbf{first} five rows of your DataFrame
by default, but we could also pass a number as well:
\texttt{movies\_df.head(10)} would output the top ten rows, for example.

To see the \textbf{last} five rows use \texttt{.tail()}. \texttt{tail()}
also accepts a number, and in this case we printing the bottom two
rows.:

    \begin{tcolorbox}[breakable, size=fbox, boxrule=1pt, pad at break*=1mm,colback=cellbackground, colframe=cellborder]
\prompt{In}{incolor}{2}{\boxspacing}
\begin{Verbatim}[commandchars=\\\{\}]
\PY{n}{movies\PYZus{}df}\PY{o}{.}\PY{n}{tail}\PY{p}{(}\PY{l+m+mi}{2}\PY{p}{)}
\end{Verbatim}
\end{tcolorbox}

    \begin{Verbatim}[commandchars=\\\{\}, frame=single, framerule=2mm, rulecolor=\color{outerrorbackground}]
\textcolor{ansi-red-intense}{\textbf{---------------------------------------------------------------------------}}
\textcolor{ansi-red-intense}{\textbf{NameError}}                                 Traceback (most recent call last)
\textcolor{ansi-green-intense}{\textbf{<ipython-input-2-46b7a669ef61>}} in \textcolor{ansi-cyan}{<module>}
\textcolor{ansi-green-intense}{\textbf{----> 1}}\textcolor{ansi-yellow-intense}{\textbf{ }}movies\_df\textcolor{ansi-yellow-intense}{\textbf{.}}tail\textcolor{ansi-yellow-intense}{\textbf{(}}\textcolor{ansi-cyan-intense}{\textbf{2}}\textcolor{ansi-yellow-intense}{\textbf{)}}

\textcolor{ansi-red-intense}{\textbf{NameError}}: name 'movies\_df' is not defined
    \end{Verbatim}

    Typically when we load in a dataset, we like to view the first five or
so rows to see what's under the hood. Here we can see the names of each
column, the index, and examples of values in each row.

You'll notice that the index in our DataFrame is the \emph{Title}
column, which you can tell by how the word \emph{Title} is slightly
lower than the rest of the columns.

    \hypertarget{getting-info-about-your-data}{%
\paragraph{Getting info about your
data}\label{getting-info-about-your-data}}

\texttt{.info()} should be one of the very first commands you run after
loading your data:

    \begin{tcolorbox}[breakable, size=fbox, boxrule=1pt, pad at break*=1mm,colback=cellbackground, colframe=cellborder]
\prompt{In}{incolor}{3}{\boxspacing}
\begin{Verbatim}[commandchars=\\\{\}]
\PY{n}{movies\PYZus{}df}\PY{o}{.}\PY{n}{info}\PY{p}{(}\PY{p}{)}
\end{Verbatim}
\end{tcolorbox}

    \begin{Verbatim}[commandchars=\\\{\}, frame=single, framerule=2mm, rulecolor=\color{outerrorbackground}]
\textcolor{ansi-red-intense}{\textbf{---------------------------------------------------------------------------}}
\textcolor{ansi-red-intense}{\textbf{NameError}}                                 Traceback (most recent call last)
\textcolor{ansi-green-intense}{\textbf{<ipython-input-3-0dc782dfd5a1>}} in \textcolor{ansi-cyan}{<module>}
\textcolor{ansi-green-intense}{\textbf{----> 1}}\textcolor{ansi-yellow-intense}{\textbf{ }}movies\_df\textcolor{ansi-yellow-intense}{\textbf{.}}info\textcolor{ansi-yellow-intense}{\textbf{(}}\textcolor{ansi-yellow-intense}{\textbf{)}}

\textcolor{ansi-red-intense}{\textbf{NameError}}: name 'movies\_df' is not defined
    \end{Verbatim}

    \texttt{.info()} provides the essential details about your dataset, such
as the number of rows and columns, the number of non-null values, what
type of data is in each column, and how much memory your DataFrame is
using.

Notice in our movies dataset we have some obvious missing values in the
\texttt{Revenue} and \texttt{Metascore} columns. We'll look at how to
handle those in a bit.

Seeing the datatype quickly is actually quite useful. Imagine you just
imported some JSON and the integers were recorded as strings. You go to
do some arithmetic and find an ``unsupported operand'' Exception because
you can't do math with strings. Calling \texttt{.info()} will quickly
point out that your column you thought was all integers are actually
string objects.

Another fast and useful attribute is \texttt{.shape}, which outputs just
a tuple of (rows, columns):

    \begin{tcolorbox}[breakable, size=fbox, boxrule=1pt, pad at break*=1mm,colback=cellbackground, colframe=cellborder]
\prompt{In}{incolor}{4}{\boxspacing}
\begin{Verbatim}[commandchars=\\\{\}]
\PY{n}{movies\PYZus{}df}\PY{o}{.}\PY{n}{shape}
\end{Verbatim}
\end{tcolorbox}

    \begin{Verbatim}[commandchars=\\\{\}, frame=single, framerule=2mm, rulecolor=\color{outerrorbackground}]
\textcolor{ansi-red-intense}{\textbf{---------------------------------------------------------------------------}}
\textcolor{ansi-red-intense}{\textbf{NameError}}                                 Traceback (most recent call last)
\textcolor{ansi-green-intense}{\textbf{<ipython-input-4-971005451c9b>}} in \textcolor{ansi-cyan}{<module>}
\textcolor{ansi-green-intense}{\textbf{----> 1}}\textcolor{ansi-yellow-intense}{\textbf{ }}movies\_df\textcolor{ansi-yellow-intense}{\textbf{.}}shape

\textcolor{ansi-red-intense}{\textbf{NameError}}: name 'movies\_df' is not defined
    \end{Verbatim}

    Note that \texttt{.shape} has no parentheses and is a simple tuple of
format (rows, columns). So we have \textbf{1000 rows} and \textbf{11
columns} in our movies DataFrame.

You'll be going to \texttt{.shape} a lot when cleaning and transforming
data. For example, you might filter some rows based on some criteria and
then want to know quickly how many rows were removed.

    \hypertarget{handling-duplicates}{%
\paragraph{Handling duplicates}\label{handling-duplicates}}

    This dataset does not have duplicate rows, but it is always important to
verify you aren't aggregating duplicate rows.

To demonstrate, let's simply just double up our movies DataFrame by
appending it to itself:

    \begin{tcolorbox}[breakable, size=fbox, boxrule=1pt, pad at break*=1mm,colback=cellbackground, colframe=cellborder]
\prompt{In}{incolor}{5}{\boxspacing}
\begin{Verbatim}[commandchars=\\\{\}]
\PY{n}{temp\PYZus{}df} \PY{o}{=} \PY{n}{movies\PYZus{}df}\PY{o}{.}\PY{n}{append}\PY{p}{(}\PY{n}{movies\PYZus{}df}\PY{p}{)}

\PY{n}{temp\PYZus{}df}\PY{o}{.}\PY{n}{shape}
\end{Verbatim}
\end{tcolorbox}

    \begin{Verbatim}[commandchars=\\\{\}, frame=single, framerule=2mm, rulecolor=\color{outerrorbackground}]
\textcolor{ansi-red-intense}{\textbf{---------------------------------------------------------------------------}}
\textcolor{ansi-red-intense}{\textbf{NameError}}                                 Traceback (most recent call last)
\textcolor{ansi-green-intense}{\textbf{<ipython-input-5-682e9f259182>}} in \textcolor{ansi-cyan}{<module>}
\textcolor{ansi-green-intense}{\textbf{----> 1}}\textcolor{ansi-yellow-intense}{\textbf{ }}temp\_df \textcolor{ansi-yellow-intense}{\textbf{=}} movies\_df\textcolor{ansi-yellow-intense}{\textbf{.}}append\textcolor{ansi-yellow-intense}{\textbf{(}}movies\_df\textcolor{ansi-yellow-intense}{\textbf{)}}
\textcolor{ansi-green}{      2} 
\textcolor{ansi-green}{      3} temp\_df\textcolor{ansi-yellow-intense}{\textbf{.}}shape

\textcolor{ansi-red-intense}{\textbf{NameError}}: name 'movies\_df' is not defined
    \end{Verbatim}

    Using \texttt{append()} will return a copy without affecting the
original DataFrame. We are capturing this copy in \texttt{temp} so we
aren't working with the real data.

Notice call \texttt{.shape} quickly proves our DataFrame rows have
doubled.

Now we can try dropping duplicates:

    \begin{tcolorbox}[breakable, size=fbox, boxrule=1pt, pad at break*=1mm,colback=cellbackground, colframe=cellborder]
\prompt{In}{incolor}{6}{\boxspacing}
\begin{Verbatim}[commandchars=\\\{\}]
\PY{n}{temp\PYZus{}df} \PY{o}{=} \PY{n}{temp\PYZus{}df}\PY{o}{.}\PY{n}{drop\PYZus{}duplicates}\PY{p}{(}\PY{p}{)}

\PY{n}{temp\PYZus{}df}\PY{o}{.}\PY{n}{shape}
\end{Verbatim}
\end{tcolorbox}

    \begin{Verbatim}[commandchars=\\\{\}, frame=single, framerule=2mm, rulecolor=\color{outerrorbackground}]
\textcolor{ansi-red-intense}{\textbf{---------------------------------------------------------------------------}}
\textcolor{ansi-red-intense}{\textbf{NameError}}                                 Traceback (most recent call last)
\textcolor{ansi-green-intense}{\textbf{<ipython-input-6-2e135e66f439>}} in \textcolor{ansi-cyan}{<module>}
\textcolor{ansi-green-intense}{\textbf{----> 1}}\textcolor{ansi-yellow-intense}{\textbf{ }}temp\_df \textcolor{ansi-yellow-intense}{\textbf{=}} temp\_df\textcolor{ansi-yellow-intense}{\textbf{.}}drop\_duplicates\textcolor{ansi-yellow-intense}{\textbf{(}}\textcolor{ansi-yellow-intense}{\textbf{)}}
\textcolor{ansi-green}{      2} 
\textcolor{ansi-green}{      3} temp\_df\textcolor{ansi-yellow-intense}{\textbf{.}}shape

\textcolor{ansi-red-intense}{\textbf{NameError}}: name 'temp\_df' is not defined
    \end{Verbatim}

    Just like \texttt{append()}, the \texttt{drop\_duplicates()} method will
also return a copy of your DataFrame, but this time with duplicates
removed. Calling \texttt{.shape} confirms we're back to the 1000 rows of
our original dataset.

It's a little verbose to keep assigning DataFrames to the same variable
like in this example. For this reason, pandas has the \texttt{inplace}
keyword argument on many of its methods. Using \texttt{inplace=True}
will modify the DataFrame object in place:

    \begin{tcolorbox}[breakable, size=fbox, boxrule=1pt, pad at break*=1mm,colback=cellbackground, colframe=cellborder]
\prompt{In}{incolor}{7}{\boxspacing}
\begin{Verbatim}[commandchars=\\\{\}]
\PY{n}{temp\PYZus{}df}\PY{o}{.}\PY{n}{drop\PYZus{}duplicates}\PY{p}{(}\PY{n}{inplace}\PY{o}{=}\PY{k+kc}{True}\PY{p}{)}
\end{Verbatim}
\end{tcolorbox}

    \begin{Verbatim}[commandchars=\\\{\}, frame=single, framerule=2mm, rulecolor=\color{outerrorbackground}]
\textcolor{ansi-red-intense}{\textbf{---------------------------------------------------------------------------}}
\textcolor{ansi-red-intense}{\textbf{NameError}}                                 Traceback (most recent call last)
\textcolor{ansi-green-intense}{\textbf{<ipython-input-7-d185a6b6c56f>}} in \textcolor{ansi-cyan}{<module>}
\textcolor{ansi-green-intense}{\textbf{----> 1}}\textcolor{ansi-yellow-intense}{\textbf{ }}temp\_df\textcolor{ansi-yellow-intense}{\textbf{.}}drop\_duplicates\textcolor{ansi-yellow-intense}{\textbf{(}}inplace\textcolor{ansi-yellow-intense}{\textbf{=}}\textcolor{ansi-green-intense}{\textbf{True}}\textcolor{ansi-yellow-intense}{\textbf{)}}

\textcolor{ansi-red-intense}{\textbf{NameError}}: name 'temp\_df' is not defined
    \end{Verbatim}

    Now our \texttt{temp\_df} \emph{will} have the transformed data
automatically.

Another important argument for \texttt{drop\_duplicates()} is
\texttt{keep}, which has three possible options:

\begin{itemize}
\tightlist
\item
  \texttt{first}: (default) Drop duplicates except for the first
  occurrence.
\item
  \texttt{last}: Drop duplicates except for the last occurrence.
\item
  \texttt{False}: Drop all duplicates.
\end{itemize}

Since we didn't define the \texttt{keep} arugment in the previous
example it was defaulted to \texttt{first}. This means that if two rows
are the same pandas will drop the second row and keep the first row.
Using \texttt{last} has the opposite effect: the first row is dropped.

\texttt{keep}, on the other hand, will drop all duplicates. If two rows
are the same then both will be dropped. Watch what happens to
\texttt{temp\_df}:

    \begin{tcolorbox}[breakable, size=fbox, boxrule=1pt, pad at break*=1mm,colback=cellbackground, colframe=cellborder]
\prompt{In}{incolor}{8}{\boxspacing}
\begin{Verbatim}[commandchars=\\\{\}]
\PY{n}{temp\PYZus{}df} \PY{o}{=} \PY{n}{movies\PYZus{}df}\PY{o}{.}\PY{n}{append}\PY{p}{(}\PY{n}{movies\PYZus{}df}\PY{p}{)}  \PY{c+c1}{\PYZsh{} make a new copy}

\PY{n}{temp\PYZus{}df}\PY{o}{.}\PY{n}{drop\PYZus{}duplicates}\PY{p}{(}\PY{n}{inplace}\PY{o}{=}\PY{k+kc}{True}\PY{p}{,} \PY{n}{keep}\PY{o}{=}\PY{k+kc}{False}\PY{p}{)}

\PY{n}{temp\PYZus{}df}\PY{o}{.}\PY{n}{shape}
\end{Verbatim}
\end{tcolorbox}

    \begin{Verbatim}[commandchars=\\\{\}, frame=single, framerule=2mm, rulecolor=\color{outerrorbackground}]
\textcolor{ansi-red-intense}{\textbf{---------------------------------------------------------------------------}}
\textcolor{ansi-red-intense}{\textbf{NameError}}                                 Traceback (most recent call last)
\textcolor{ansi-green-intense}{\textbf{<ipython-input-8-b0eef36cd1a0>}} in \textcolor{ansi-cyan}{<module>}
\textcolor{ansi-green-intense}{\textbf{----> 1}}\textcolor{ansi-yellow-intense}{\textbf{ }}temp\_df \textcolor{ansi-yellow-intense}{\textbf{=}} movies\_df\textcolor{ansi-yellow-intense}{\textbf{.}}append\textcolor{ansi-yellow-intense}{\textbf{(}}movies\_df\textcolor{ansi-yellow-intense}{\textbf{)}}  \textcolor{ansi-red-intense}{\textbf{\# make a new copy}}
\textcolor{ansi-green}{      2} 
\textcolor{ansi-green}{      3} temp\_df\textcolor{ansi-yellow-intense}{\textbf{.}}drop\_duplicates\textcolor{ansi-yellow-intense}{\textbf{(}}inplace\textcolor{ansi-yellow-intense}{\textbf{=}}\textcolor{ansi-green-intense}{\textbf{True}}\textcolor{ansi-yellow-intense}{\textbf{,}} keep\textcolor{ansi-yellow-intense}{\textbf{=}}\textcolor{ansi-green-intense}{\textbf{False}}\textcolor{ansi-yellow-intense}{\textbf{)}}
\textcolor{ansi-green}{      4} 
\textcolor{ansi-green}{      5} temp\_df\textcolor{ansi-yellow-intense}{\textbf{.}}shape

\textcolor{ansi-red-intense}{\textbf{NameError}}: name 'movies\_df' is not defined
    \end{Verbatim}

    Since all rows were duplicates, \texttt{keep=False} dropped them all
resulting in zero rows being left over. If you're wondering why you
would want to do this, one reason is that it allows you to locate all
duplicates in your dataset. When conditional selections are shown below
you'll see how to do that.

    \hypertarget{column-cleanup}{%
\paragraph{Column cleanup}\label{column-cleanup}}

Many times datasets will have verbose column names with symbols, upper
and lowercase words, spaces, and typos. To make selecting data by column
name easier we can spend a little time cleaning up their names.

Here's how to print the column names of our dataset:

    \begin{tcolorbox}[breakable, size=fbox, boxrule=1pt, pad at break*=1mm,colback=cellbackground, colframe=cellborder]
\prompt{In}{incolor}{20}{\boxspacing}
\begin{Verbatim}[commandchars=\\\{\}]
\PY{n}{movies\PYZus{}df}\PY{o}{.}\PY{n}{columns}
\end{Verbatim}
\end{tcolorbox}

            \begin{tcolorbox}[breakable, size=fbox, boxrule=.5pt, pad at break*=1mm, opacityfill=0]
\prompt{Out}{outcolor}{20}{\boxspacing}
\begin{Verbatim}[commandchars=\\\{\}]
Index(['Rank', 'Genre', 'Description', 'Director', 'Actors', 'Year',
       'Runtime (Minutes)', 'Rating', 'Votes', 'Revenue (Millions)',
       'Metascore'],
      dtype='object')
\end{Verbatim}
\end{tcolorbox}
        
    Not only does \texttt{.columns} come in handy if you want to rename
columns by allowing for simple copy and paste, it's also useful if you
need to understand why you are receiving a \texttt{Key\ Error} when
selecting data by column.

We can use the \texttt{.rename()} method to rename certain or all
columns via a \texttt{dict}. We don't want parentheses, so let's rename
those:

    \begin{tcolorbox}[breakable, size=fbox, boxrule=1pt, pad at break*=1mm,colback=cellbackground, colframe=cellborder]
\prompt{In}{incolor}{21}{\boxspacing}
\begin{Verbatim}[commandchars=\\\{\}]
\PY{n}{movies\PYZus{}df}\PY{o}{.}\PY{n}{rename}\PY{p}{(}\PY{n}{columns}\PY{o}{=}\PY{p}{\PYZob{}}
        \PY{l+s+s1}{\PYZsq{}}\PY{l+s+s1}{Runtime (Minutes)}\PY{l+s+s1}{\PYZsq{}}\PY{p}{:} \PY{l+s+s1}{\PYZsq{}}\PY{l+s+s1}{Runtime}\PY{l+s+s1}{\PYZsq{}}\PY{p}{,} 
        \PY{l+s+s1}{\PYZsq{}}\PY{l+s+s1}{Revenue (Millions)}\PY{l+s+s1}{\PYZsq{}}\PY{p}{:} \PY{l+s+s1}{\PYZsq{}}\PY{l+s+s1}{Revenue\PYZus{}millions}\PY{l+s+s1}{\PYZsq{}}
    \PY{p}{\PYZcb{}}\PY{p}{,} \PY{n}{inplace}\PY{o}{=}\PY{k+kc}{True}\PY{p}{)}


\PY{n}{movies\PYZus{}df}\PY{o}{.}\PY{n}{columns}
\end{Verbatim}
\end{tcolorbox}

            \begin{tcolorbox}[breakable, size=fbox, boxrule=.5pt, pad at break*=1mm, opacityfill=0]
\prompt{Out}{outcolor}{21}{\boxspacing}
\begin{Verbatim}[commandchars=\\\{\}]
Index(['Rank', 'Genre', 'Description', 'Director', 'Actors', 'Year', 'Runtime',
       'Rating', 'Votes', 'Revenue\_millions', 'Metascore'],
      dtype='object')
\end{Verbatim}
\end{tcolorbox}
        
    But what if we want to lowercase all names? Instead of using
\texttt{.rename()} we could also set a list of names to the columns like
so:

    \begin{tcolorbox}[breakable, size=fbox, boxrule=1pt, pad at break*=1mm,colback=cellbackground, colframe=cellborder]
\prompt{In}{incolor}{22}{\boxspacing}
\begin{Verbatim}[commandchars=\\\{\}]
\PY{n}{movies\PYZus{}df}\PY{o}{.}\PY{n}{columns} \PY{o}{=} \PY{p}{[}\PY{l+s+s1}{\PYZsq{}}\PY{l+s+s1}{rank}\PY{l+s+s1}{\PYZsq{}}\PY{p}{,} \PY{l+s+s1}{\PYZsq{}}\PY{l+s+s1}{genre}\PY{l+s+s1}{\PYZsq{}}\PY{p}{,} \PY{l+s+s1}{\PYZsq{}}\PY{l+s+s1}{description}\PY{l+s+s1}{\PYZsq{}}\PY{p}{,} \PY{l+s+s1}{\PYZsq{}}\PY{l+s+s1}{director}\PY{l+s+s1}{\PYZsq{}}\PY{p}{,} \PY{l+s+s1}{\PYZsq{}}\PY{l+s+s1}{actors}\PY{l+s+s1}{\PYZsq{}}\PY{p}{,} \PY{l+s+s1}{\PYZsq{}}\PY{l+s+s1}{year}\PY{l+s+s1}{\PYZsq{}}\PY{p}{,} \PY{l+s+s1}{\PYZsq{}}\PY{l+s+s1}{runtime}\PY{l+s+s1}{\PYZsq{}}\PY{p}{,} 
                     \PY{l+s+s1}{\PYZsq{}}\PY{l+s+s1}{rating}\PY{l+s+s1}{\PYZsq{}}\PY{p}{,} \PY{l+s+s1}{\PYZsq{}}\PY{l+s+s1}{votes}\PY{l+s+s1}{\PYZsq{}}\PY{p}{,} \PY{l+s+s1}{\PYZsq{}}\PY{l+s+s1}{revenue\PYZus{}millions}\PY{l+s+s1}{\PYZsq{}}\PY{p}{,} \PY{l+s+s1}{\PYZsq{}}\PY{l+s+s1}{metascore}\PY{l+s+s1}{\PYZsq{}}\PY{p}{]}


\PY{n}{movies\PYZus{}df}\PY{o}{.}\PY{n}{columns}
\end{Verbatim}
\end{tcolorbox}

            \begin{tcolorbox}[breakable, size=fbox, boxrule=.5pt, pad at break*=1mm, opacityfill=0]
\prompt{Out}{outcolor}{22}{\boxspacing}
\begin{Verbatim}[commandchars=\\\{\}]
Index(['rank', 'genre', 'description', 'director', 'actors', 'year', 'runtime',
       'rating', 'votes', 'revenue\_millions', 'metascore'],
      dtype='object')
\end{Verbatim}
\end{tcolorbox}
        
    But that's too much work. Instead of just renaming each column manually
we can do a list comprehension:

    \begin{tcolorbox}[breakable, size=fbox, boxrule=1pt, pad at break*=1mm,colback=cellbackground, colframe=cellborder]
\prompt{In}{incolor}{23}{\boxspacing}
\begin{Verbatim}[commandchars=\\\{\}]
\PY{n}{movies\PYZus{}df}\PY{o}{.}\PY{n}{columns} \PY{o}{=} \PY{p}{[}\PY{n}{col}\PY{o}{.}\PY{n}{lower}\PY{p}{(}\PY{p}{)} \PY{k}{for} \PY{n}{col} \PY{o+ow}{in} \PY{n}{movies\PYZus{}df}\PY{p}{]}

\PY{n}{movies\PYZus{}df}\PY{o}{.}\PY{n}{columns}
\end{Verbatim}
\end{tcolorbox}

            \begin{tcolorbox}[breakable, size=fbox, boxrule=.5pt, pad at break*=1mm, opacityfill=0]
\prompt{Out}{outcolor}{23}{\boxspacing}
\begin{Verbatim}[commandchars=\\\{\}]
Index(['rank', 'genre', 'description', 'director', 'actors', 'year', 'runtime',
       'rating', 'votes', 'revenue\_millions', 'metascore'],
      dtype='object')
\end{Verbatim}
\end{tcolorbox}
        
    \texttt{list} (and \texttt{dict}) comprehensions come in handy a lot
when working with pandas and data in general.

It's a good idea to lowercase, remove special characters, and replace
spaces with underscores if you'll be working with a dataset for some
time.

    \hypertarget{how-to-work-with-missing-values}{%
\subsubsection{How to work with missing
values}\label{how-to-work-with-missing-values}}

When exploring data, you'll most likely encounter missing or null
values, which are essentially placeholders for non-existent values. Most
commonly you'll see Python's \texttt{None} or NumPy's \texttt{np.nan},
each of which are handled differently in some situations.

There are two options in dealing with nulls:

\begin{enumerate}
\def\labelenumi{\arabic{enumi}.}
\tightlist
\item
  Get rid of rows or columns with nulls
\item
  Replace nulls with non-null values, a technique known as
  \textbf{imputation}
\end{enumerate}

Let's calculate to total number of nulls in each column of our dataset.
The first step is to check which cells in our DataFrame are null:

    \begin{tcolorbox}[breakable, size=fbox, boxrule=1pt, pad at break*=1mm,colback=cellbackground, colframe=cellborder]
\prompt{In}{incolor}{24}{\boxspacing}
\begin{Verbatim}[commandchars=\\\{\}]
\PY{n}{movies\PYZus{}df}\PY{o}{.}\PY{n}{isnull}\PY{p}{(}\PY{p}{)}
\end{Verbatim}
\end{tcolorbox}

            \begin{tcolorbox}[breakable, size=fbox, boxrule=.5pt, pad at break*=1mm, opacityfill=0]
\prompt{Out}{outcolor}{24}{\boxspacing}
\begin{Verbatim}[commandchars=\\\{\}]
                          rank  genre  description  director  actors   year  \textbackslash{}
Title
Guardians of the Galaxy  False  False        False     False   False  False
Prometheus               False  False        False     False   False  False
Split                    False  False        False     False   False  False
Sing                     False  False        False     False   False  False
Suicide Squad            False  False        False     False   False  False
{\ldots}                        {\ldots}    {\ldots}          {\ldots}       {\ldots}     {\ldots}    {\ldots}
Secret in Their Eyes     False  False        False     False   False  False
Hostel: Part II          False  False        False     False   False  False
Step Up 2: The Streets   False  False        False     False   False  False
Search Party             False  False        False     False   False  False
Nine Lives               False  False        False     False   False  False

                         runtime  rating  votes  revenue\_millions  metascore
Title
Guardians of the Galaxy    False   False  False             False      False
Prometheus                 False   False  False             False      False
Split                      False   False  False             False      False
Sing                       False   False  False             False      False
Suicide Squad              False   False  False             False      False
{\ldots}                          {\ldots}     {\ldots}    {\ldots}               {\ldots}        {\ldots}
Secret in Their Eyes       False   False  False              True      False
Hostel: Part II            False   False  False             False      False
Step Up 2: The Streets     False   False  False             False      False
Search Party               False   False  False              True      False
Nine Lives                 False   False  False             False      False

[1000 rows x 11 columns]
\end{Verbatim}
\end{tcolorbox}
        
    Notice \texttt{isnull()} returns a DataFrame where each cell is either
True or False depending on that cell's null status.

To count the number of nulls in each column we use an aggregate function
for summing:

    \begin{tcolorbox}[breakable, size=fbox, boxrule=1pt, pad at break*=1mm,colback=cellbackground, colframe=cellborder]
\prompt{In}{incolor}{25}{\boxspacing}
\begin{Verbatim}[commandchars=\\\{\}]
\PY{n}{movies\PYZus{}df}\PY{o}{.}\PY{n}{isnull}\PY{p}{(}\PY{p}{)}\PY{o}{.}\PY{n}{sum}\PY{p}{(}\PY{p}{)}
\end{Verbatim}
\end{tcolorbox}

            \begin{tcolorbox}[breakable, size=fbox, boxrule=.5pt, pad at break*=1mm, opacityfill=0]
\prompt{Out}{outcolor}{25}{\boxspacing}
\begin{Verbatim}[commandchars=\\\{\}]
rank                  0
genre                 0
description           0
director              0
actors                0
year                  0
runtime               0
rating                0
votes                 0
revenue\_millions    128
metascore            64
dtype: int64
\end{Verbatim}
\end{tcolorbox}
        
    \texttt{.isnull()} just by iteself isn't very useful, and is usually
used in conjunction with other methods, like \texttt{sum()}.

We can see now that our data has \textbf{128} missing values for
\texttt{revenue\_millions} and \textbf{64} missing values for
\texttt{metascore}.

    \hypertarget{removing-null-values}{%
\paragraph{Removing null values}\label{removing-null-values}}

Data Scientists and Analysts regularly face the dilemma of dropping or
imputing null values, and is a decision that requires intimate knowledge
of your data and its context. Overall, removing null data is only
suggested if you have a small amount of missing data.

Remove nulls is pretty simple:

    \begin{tcolorbox}[breakable, size=fbox, boxrule=1pt, pad at break*=1mm,colback=cellbackground, colframe=cellborder]
\prompt{In}{incolor}{26}{\boxspacing}
\begin{Verbatim}[commandchars=\\\{\}]
\PY{n}{movies\PYZus{}df}\PY{o}{.}\PY{n}{dropna}\PY{p}{(}\PY{p}{)}
\end{Verbatim}
\end{tcolorbox}

            \begin{tcolorbox}[breakable, size=fbox, boxrule=.5pt, pad at break*=1mm, opacityfill=0]
\prompt{Out}{outcolor}{26}{\boxspacing}
\begin{Verbatim}[commandchars=\\\{\}]
                          rank                     genre  \textbackslash{}
Title
Guardians of the Galaxy      1   Action,Adventure,Sci-Fi
Prometheus                   2  Adventure,Mystery,Sci-Fi
Split                        3           Horror,Thriller
Sing                         4   Animation,Comedy,Family
Suicide Squad                5  Action,Adventure,Fantasy
{\ldots}                        {\ldots}                       {\ldots}
Resident Evil: Afterlife   994   Action,Adventure,Horror
Project X                  995                    Comedy
Hostel: Part II            997                    Horror
Step Up 2: The Streets     998       Drama,Music,Romance
Nine Lives                1000     Comedy,Family,Fantasy

                                                                description  \textbackslash{}
Title
Guardians of the Galaxy   A group of intergalactic criminals are forced {\ldots}
Prometheus                Following clues to the origin of mankind, a te{\ldots}
Split                     Three girls are kidnapped by a man with a diag{\ldots}
Sing                      In a city of humanoid animals, a hustling thea{\ldots}
Suicide Squad             A secret government agency recruits some of th{\ldots}
{\ldots}                                                                     {\ldots}
Resident Evil: Afterlife  While still out to destroy the evil Umbrella C{\ldots}
Project X                 3 high school seniors throw a birthday party t{\ldots}
Hostel: Part II           Three American college students studying abroa{\ldots}
Step Up 2: The Streets    Romantic sparks occur between two dance studen{\ldots}
Nine Lives                A stuffy businessman finds himself trapped ins{\ldots}

                                      director  \textbackslash{}
Title
Guardians of the Galaxy             James Gunn
Prometheus                        Ridley Scott
Split                       M. Night Shyamalan
Sing                      Christophe Lourdelet
Suicide Squad                       David Ayer
{\ldots}                                        {\ldots}
Resident Evil: Afterlife    Paul W.S. Anderson
Project X                      Nima Nourizadeh
Hostel: Part II                       Eli Roth
Step Up 2: The Streets              Jon M. Chu
Nine Lives                    Barry Sonnenfeld

                                                                     actors  \textbackslash{}
Title
Guardians of the Galaxy   Chris Pratt, Vin Diesel, Bradley Cooper, Zoe S{\ldots}
Prometheus                Noomi Rapace, Logan Marshall-Green, Michael Fa{\ldots}
Split                     James McAvoy, Anya Taylor-Joy, Haley Lu Richar{\ldots}
Sing                      Matthew McConaughey,Reese Witherspoon, Seth Ma{\ldots}
Suicide Squad             Will Smith, Jared Leto, Margot Robbie, Viola D{\ldots}
{\ldots}                                                                     {\ldots}
Resident Evil: Afterlife  Milla Jovovich, Ali Larter, Wentworth Miller,K{\ldots}
Project X                 Thomas Mann, Oliver Cooper, Jonathan Daniel Br{\ldots}
Hostel: Part II           Lauren German, Heather Matarazzo, Bijou Philli{\ldots}
Step Up 2: The Streets    Robert Hoffman, Briana Evigan, Cassie Ventura,{\ldots}
Nine Lives                Kevin Spacey, Jennifer Garner, Robbie Amell,Ch{\ldots}

                          year  runtime  rating   votes  revenue\_millions  \textbackslash{}
Title
Guardians of the Galaxy   2014      121     8.1  757074            333.13
Prometheus                2012      124     7.0  485820            126.46
Split                     2016      117     7.3  157606            138.12
Sing                      2016      108     7.2   60545            270.32
Suicide Squad             2016      123     6.2  393727            325.02
{\ldots}                        {\ldots}      {\ldots}     {\ldots}     {\ldots}               {\ldots}
Resident Evil: Afterlife  2010       97     5.9  140900             60.13
Project X                 2012       88     6.7  164088             54.72
Hostel: Part II           2007       94     5.5   73152             17.54
Step Up 2: The Streets    2008       98     6.2   70699             58.01
Nine Lives                2016       87     5.3   12435             19.64

                          metascore
Title
Guardians of the Galaxy        76.0
Prometheus                     65.0
Split                          62.0
Sing                           59.0
Suicide Squad                  40.0
{\ldots}                             {\ldots}
Resident Evil: Afterlife       37.0
Project X                      48.0
Hostel: Part II                46.0
Step Up 2: The Streets         50.0
Nine Lives                     11.0

[838 rows x 11 columns]
\end{Verbatim}
\end{tcolorbox}
        
    This operation will delete any \textbf{row} with at least a single null
value, but it will return a new DataFrame without altering the original
one. You could specify \texttt{inplace=True} in this method as well.

So in the case of our dataset, this operation would remove 128 rows
where \texttt{revenue\_millions} is null and 64 rows where
\texttt{metascore} is null. This obviously seems like a waste since
there's perfectly good data in the other columns of those dropped rows.
That's why we'll look at imputation next.

Other than just dropping rows, you can also drop columns with null
values by setting \texttt{axis=1}:

    \begin{tcolorbox}[breakable, size=fbox, boxrule=1pt, pad at break*=1mm,colback=cellbackground, colframe=cellborder]
\prompt{In}{incolor}{27}{\boxspacing}
\begin{Verbatim}[commandchars=\\\{\}]
\PY{n}{movies\PYZus{}df}\PY{o}{.}\PY{n}{dropna}\PY{p}{(}\PY{n}{axis}\PY{o}{=}\PY{l+m+mi}{1}\PY{p}{)}
\end{Verbatim}
\end{tcolorbox}

            \begin{tcolorbox}[breakable, size=fbox, boxrule=.5pt, pad at break*=1mm, opacityfill=0]
\prompt{Out}{outcolor}{27}{\boxspacing}
\begin{Verbatim}[commandchars=\\\{\}]
                         rank                     genre  \textbackslash{}
Title
Guardians of the Galaxy     1   Action,Adventure,Sci-Fi
Prometheus                  2  Adventure,Mystery,Sci-Fi
Split                       3           Horror,Thriller
Sing                        4   Animation,Comedy,Family
Suicide Squad               5  Action,Adventure,Fantasy
{\ldots}                       {\ldots}                       {\ldots}
Secret in Their Eyes      996       Crime,Drama,Mystery
Hostel: Part II           997                    Horror
Step Up 2: The Streets    998       Drama,Music,Romance
Search Party              999          Adventure,Comedy
Nine Lives               1000     Comedy,Family,Fantasy

                                                               description  \textbackslash{}
Title
Guardians of the Galaxy  A group of intergalactic criminals are forced {\ldots}
Prometheus               Following clues to the origin of mankind, a te{\ldots}
Split                    Three girls are kidnapped by a man with a diag{\ldots}
Sing                     In a city of humanoid animals, a hustling thea{\ldots}
Suicide Squad            A secret government agency recruits some of th{\ldots}
{\ldots}                                                                    {\ldots}
Secret in Their Eyes     A tight-knit team of rising investigators, alo{\ldots}
Hostel: Part II          Three American college students studying abroa{\ldots}
Step Up 2: The Streets   Romantic sparks occur between two dance studen{\ldots}
Search Party             A pair of friends embark on a mission to reuni{\ldots}
Nine Lives               A stuffy businessman finds himself trapped ins{\ldots}

                                     director  \textbackslash{}
Title
Guardians of the Galaxy            James Gunn
Prometheus                       Ridley Scott
Split                      M. Night Shyamalan
Sing                     Christophe Lourdelet
Suicide Squad                      David Ayer
{\ldots}                                       {\ldots}
Secret in Their Eyes                Billy Ray
Hostel: Part II                      Eli Roth
Step Up 2: The Streets             Jon M. Chu
Search Party                   Scot Armstrong
Nine Lives                   Barry Sonnenfeld

                                                                    actors  \textbackslash{}
Title
Guardians of the Galaxy  Chris Pratt, Vin Diesel, Bradley Cooper, Zoe S{\ldots}
Prometheus               Noomi Rapace, Logan Marshall-Green, Michael Fa{\ldots}
Split                    James McAvoy, Anya Taylor-Joy, Haley Lu Richar{\ldots}
Sing                     Matthew McConaughey,Reese Witherspoon, Seth Ma{\ldots}
Suicide Squad            Will Smith, Jared Leto, Margot Robbie, Viola D{\ldots}
{\ldots}                                                                    {\ldots}
Secret in Their Eyes     Chiwetel Ejiofor, Nicole Kidman, Julia Roberts{\ldots}
Hostel: Part II          Lauren German, Heather Matarazzo, Bijou Philli{\ldots}
Step Up 2: The Streets   Robert Hoffman, Briana Evigan, Cassie Ventura,{\ldots}
Search Party             Adam Pally, T.J. Miller, Thomas Middleditch,Sh{\ldots}
Nine Lives               Kevin Spacey, Jennifer Garner, Robbie Amell,Ch{\ldots}

                         year  runtime  rating   votes
Title
Guardians of the Galaxy  2014      121     8.1  757074
Prometheus               2012      124     7.0  485820
Split                    2016      117     7.3  157606
Sing                     2016      108     7.2   60545
Suicide Squad            2016      123     6.2  393727
{\ldots}                       {\ldots}      {\ldots}     {\ldots}     {\ldots}
Secret in Their Eyes     2015      111     6.2   27585
Hostel: Part II          2007       94     5.5   73152
Step Up 2: The Streets   2008       98     6.2   70699
Search Party             2014       93     5.6    4881
Nine Lives               2016       87     5.3   12435

[1000 rows x 9 columns]
\end{Verbatim}
\end{tcolorbox}
        
    In our dataset, this operation would drop the \texttt{revenue\_millions}
and \texttt{metascore} columns.

\textbf{Intuition side note}: What's with this \texttt{axis=1}
parameter?

It's not immediately obvious where \texttt{axis} comes from and why you
need it to be 1 for it to affect columns. To see why, just look at the
\texttt{.shape} output:

    \begin{tcolorbox}[breakable, size=fbox, boxrule=1pt, pad at break*=1mm,colback=cellbackground, colframe=cellborder]
\prompt{In}{incolor}{28}{\boxspacing}
\begin{Verbatim}[commandchars=\\\{\}]
\PY{n}{movies\PYZus{}df}\PY{o}{.}\PY{n}{shape}
\end{Verbatim}
\end{tcolorbox}

            \begin{tcolorbox}[breakable, size=fbox, boxrule=.5pt, pad at break*=1mm, opacityfill=0]
\prompt{Out}{outcolor}{28}{\boxspacing}
\begin{Verbatim}[commandchars=\\\{\}]
(1000, 11)
\end{Verbatim}
\end{tcolorbox}
        
    As we learned above, this is a tuple that represents the shape of the
DataFrame, i.e.~1000 rows and 11 columns. Note that the \emph{rows} are
at index zero of this tuple and \emph{columns} are at \textbf{index one}
of this tuple. This is why \texttt{axis=1} affects columns. This comes
from NumPy, and is a great example of why learning NumPy is worth your
time.

    \hypertarget{imputation}{%
\subsubsection{Imputation}\label{imputation}}

Imputation is a conventional feature engineering technique used to keep
valuable data that have null values.

There may be instances where dropping every row with a null value
removes too big a chunk from your dataset, so instead we can impute that
null with another value, usually the \textbf{mean} or the
\textbf{median} of that column.

Let's look at imputing the missing values in the
\texttt{revenue\_millions} column. First we'll extract that column into
its own variable:

    \begin{tcolorbox}[breakable, size=fbox, boxrule=1pt, pad at break*=1mm,colback=cellbackground, colframe=cellborder]
\prompt{In}{incolor}{29}{\boxspacing}
\begin{Verbatim}[commandchars=\\\{\}]
\PY{n}{revenue} \PY{o}{=} \PY{n}{movies\PYZus{}df}\PY{p}{[}\PY{l+s+s1}{\PYZsq{}}\PY{l+s+s1}{revenue\PYZus{}millions}\PY{l+s+s1}{\PYZsq{}}\PY{p}{]}
\end{Verbatim}
\end{tcolorbox}

    \textbf{\emph{Using square brackets is the general way we select columns
in a DataFrame}}.

If you remember back to when we created DataFrames from scratch, the
keys of the \texttt{dict} ended up as column names. Now when we select
columns of a DataFrame, we use brackets just like if we were accessing a
Python dictionary.

\texttt{revenue} now contains a Series:

    \begin{tcolorbox}[breakable, size=fbox, boxrule=1pt, pad at break*=1mm,colback=cellbackground, colframe=cellborder]
\prompt{In}{incolor}{30}{\boxspacing}
\begin{Verbatim}[commandchars=\\\{\}]
\PY{n}{revenue}\PY{o}{.}\PY{n}{head}\PY{p}{(}\PY{p}{)}
\end{Verbatim}
\end{tcolorbox}

            \begin{tcolorbox}[breakable, size=fbox, boxrule=.5pt, pad at break*=1mm, opacityfill=0]
\prompt{Out}{outcolor}{30}{\boxspacing}
\begin{Verbatim}[commandchars=\\\{\}]
Title
Guardians of the Galaxy    333.13
Prometheus                 126.46
Split                      138.12
Sing                       270.32
Suicide Squad              325.02
Name: revenue\_millions, dtype: float64
\end{Verbatim}
\end{tcolorbox}
        
    Slightly different formatting than a DataFrame, but we still have our
\texttt{Title} index.

We'll impute the missing values of revenue using the mean. Here's the
mean value:

    \begin{tcolorbox}[breakable, size=fbox, boxrule=1pt, pad at break*=1mm,colback=cellbackground, colframe=cellborder]
\prompt{In}{incolor}{31}{\boxspacing}
\begin{Verbatim}[commandchars=\\\{\}]
\PY{n}{revenue\PYZus{}mean} \PY{o}{=} \PY{n}{revenue}\PY{o}{.}\PY{n}{mean}\PY{p}{(}\PY{p}{)}

\PY{n}{revenue\PYZus{}mean}
\end{Verbatim}
\end{tcolorbox}

            \begin{tcolorbox}[breakable, size=fbox, boxrule=.5pt, pad at break*=1mm, opacityfill=0]
\prompt{Out}{outcolor}{31}{\boxspacing}
\begin{Verbatim}[commandchars=\\\{\}]
82.95637614678897
\end{Verbatim}
\end{tcolorbox}
        
    With the mean, let's fill the nulls using \texttt{fillna()}:

    \begin{tcolorbox}[breakable, size=fbox, boxrule=1pt, pad at break*=1mm,colback=cellbackground, colframe=cellborder]
\prompt{In}{incolor}{32}{\boxspacing}
\begin{Verbatim}[commandchars=\\\{\}]
\PY{n}{revenue}\PY{o}{.}\PY{n}{fillna}\PY{p}{(}\PY{n}{revenue\PYZus{}mean}\PY{p}{,} \PY{n}{inplace}\PY{o}{=}\PY{k+kc}{True}\PY{p}{)}
\end{Verbatim}
\end{tcolorbox}

    We have now replaced all nulls in \texttt{revenue} with the mean of the
column. Notice that by using \texttt{inplace=True} we have actually
affected the original \texttt{movies\_df}:

    \begin{tcolorbox}[breakable, size=fbox, boxrule=1pt, pad at break*=1mm,colback=cellbackground, colframe=cellborder]
\prompt{In}{incolor}{33}{\boxspacing}
\begin{Verbatim}[commandchars=\\\{\}]
\PY{n}{movies\PYZus{}df}\PY{o}{.}\PY{n}{isnull}\PY{p}{(}\PY{p}{)}\PY{o}{.}\PY{n}{sum}\PY{p}{(}\PY{p}{)}
\end{Verbatim}
\end{tcolorbox}

            \begin{tcolorbox}[breakable, size=fbox, boxrule=.5pt, pad at break*=1mm, opacityfill=0]
\prompt{Out}{outcolor}{33}{\boxspacing}
\begin{Verbatim}[commandchars=\\\{\}]
rank                 0
genre                0
description          0
director             0
actors               0
year                 0
runtime              0
rating               0
votes                0
revenue\_millions     0
metascore           64
dtype: int64
\end{Verbatim}
\end{tcolorbox}
        
    Imputing an entire column with the same value like this is a basic
example. It would be a better idea to try a more granular imputation by
Genre or Director.

For example, you would find the mean of the revenue generated in each
genre individually and impute the nulls in each genre with that genre's
mean.

Let's now look at more ways to examine and understand the dataset.

    \hypertarget{understanding-your-variables}{%
\subsubsection{Understanding your
variables}\label{understanding-your-variables}}

    Using \texttt{describe()} on an entire DataFrame we can get a summary of
the distribution of continuous variables:

    \begin{tcolorbox}[breakable, size=fbox, boxrule=1pt, pad at break*=1mm,colback=cellbackground, colframe=cellborder]
\prompt{In}{incolor}{34}{\boxspacing}
\begin{Verbatim}[commandchars=\\\{\}]
\PY{n}{movies\PYZus{}df}\PY{o}{.}\PY{n}{describe}\PY{p}{(}\PY{p}{)}
\end{Verbatim}
\end{tcolorbox}

            \begin{tcolorbox}[breakable, size=fbox, boxrule=.5pt, pad at break*=1mm, opacityfill=0]
\prompt{Out}{outcolor}{34}{\boxspacing}
\begin{Verbatim}[commandchars=\\\{\}]
              rank         year      runtime       rating         votes  \textbackslash{}
count  1000.000000  1000.000000  1000.000000  1000.000000  1.000000e+03
mean    500.500000  2012.783000   113.172000     6.723200  1.698083e+05
std     288.819436     3.205962    18.810908     0.945429  1.887626e+05
min       1.000000  2006.000000    66.000000     1.900000  6.100000e+01
25\%     250.750000  2010.000000   100.000000     6.200000  3.630900e+04
50\%     500.500000  2014.000000   111.000000     6.800000  1.107990e+05
75\%     750.250000  2016.000000   123.000000     7.400000  2.399098e+05
max    1000.000000  2016.000000   191.000000     9.000000  1.791916e+06

       revenue\_millions   metascore
count       1000.000000  936.000000
mean          82.956376   58.985043
std           96.412043   17.194757
min            0.000000   11.000000
25\%           17.442500   47.000000
50\%           60.375000   59.500000
75\%           99.177500   72.000000
max          936.630000  100.000000
\end{Verbatim}
\end{tcolorbox}
        
    Understanding which numbers are continuous also comes in handy when
thinking about the type of plot to use to represent your data visually.

\texttt{.describe()} can also be used on a categorical variable to get
the count of rows, unique count of categories, top category, and freq of
top category:

    \begin{tcolorbox}[breakable, size=fbox, boxrule=1pt, pad at break*=1mm,colback=cellbackground, colframe=cellborder]
\prompt{In}{incolor}{35}{\boxspacing}
\begin{Verbatim}[commandchars=\\\{\}]
\PY{n}{movies\PYZus{}df}\PY{p}{[}\PY{l+s+s1}{\PYZsq{}}\PY{l+s+s1}{genre}\PY{l+s+s1}{\PYZsq{}}\PY{p}{]}\PY{o}{.}\PY{n}{describe}\PY{p}{(}\PY{p}{)}
\end{Verbatim}
\end{tcolorbox}

            \begin{tcolorbox}[breakable, size=fbox, boxrule=.5pt, pad at break*=1mm, opacityfill=0]
\prompt{Out}{outcolor}{35}{\boxspacing}
\begin{Verbatim}[commandchars=\\\{\}]
count                        1000
unique                        207
top       Action,Adventure,Sci-Fi
freq                           50
Name: genre, dtype: object
\end{Verbatim}
\end{tcolorbox}
        
    This tells us that the genre column has 207 unique values, the top value
is Action/Adventure/Sci-Fi, which shows up 50 times (freq).

\texttt{.value\_counts()} can tell us the frequency of all values in a
column:

    \begin{tcolorbox}[breakable, size=fbox, boxrule=1pt, pad at break*=1mm,colback=cellbackground, colframe=cellborder]
\prompt{In}{incolor}{36}{\boxspacing}
\begin{Verbatim}[commandchars=\\\{\}]
\PY{n}{movies\PYZus{}df}\PY{p}{[}\PY{l+s+s1}{\PYZsq{}}\PY{l+s+s1}{genre}\PY{l+s+s1}{\PYZsq{}}\PY{p}{]}\PY{o}{.}\PY{n}{value\PYZus{}counts}\PY{p}{(}\PY{p}{)}\PY{o}{.}\PY{n}{head}\PY{p}{(}\PY{l+m+mi}{10}\PY{p}{)}
\end{Verbatim}
\end{tcolorbox}

            \begin{tcolorbox}[breakable, size=fbox, boxrule=.5pt, pad at break*=1mm, opacityfill=0]
\prompt{Out}{outcolor}{36}{\boxspacing}
\begin{Verbatim}[commandchars=\\\{\}]
Action,Adventure,Sci-Fi       50
Drama                         48
Comedy,Drama,Romance          35
Comedy                        32
Drama,Romance                 31
Animation,Adventure,Comedy    27
Action,Adventure,Fantasy      27
Comedy,Drama                  27
Comedy,Romance                26
Crime,Drama,Thriller          24
Name: genre, dtype: int64
\end{Verbatim}
\end{tcolorbox}
        
    \hypertarget{relationships-between-continuous-variables}{%
\paragraph{Relationships between continuous
variables}\label{relationships-between-continuous-variables}}

    By using the correlation method \texttt{.corr()} we can generate the
relationship between each continuous variable:

    \begin{tcolorbox}[breakable, size=fbox, boxrule=1pt, pad at break*=1mm,colback=cellbackground, colframe=cellborder]
\prompt{In}{incolor}{37}{\boxspacing}
\begin{Verbatim}[commandchars=\\\{\}]
\PY{n}{movies\PYZus{}df}\PY{o}{.}\PY{n}{corr}\PY{p}{(}\PY{p}{)}
\end{Verbatim}
\end{tcolorbox}

            \begin{tcolorbox}[breakable, size=fbox, boxrule=.5pt, pad at break*=1mm, opacityfill=0]
\prompt{Out}{outcolor}{37}{\boxspacing}
\begin{Verbatim}[commandchars=\\\{\}]
                      rank      year   runtime    rating     votes  \textbackslash{}
rank              1.000000 -0.261605 -0.221739 -0.219555 -0.283876
year             -0.261605  1.000000 -0.164900 -0.211219 -0.411904
runtime          -0.221739 -0.164900  1.000000  0.392214  0.407062
rating           -0.219555 -0.211219  0.392214  1.000000  0.511537
votes            -0.283876 -0.411904  0.407062  0.511537  1.000000
revenue\_millions -0.252996 -0.117562  0.247834  0.189527  0.607941
metascore        -0.191869 -0.079305  0.211978  0.631897  0.325684

                  revenue\_millions  metascore
rank                     -0.252996  -0.191869
year                     -0.117562  -0.079305
runtime                   0.247834   0.211978
rating                    0.189527   0.631897
votes                     0.607941   0.325684
revenue\_millions          1.000000   0.133328
metascore                 0.133328   1.000000
\end{Verbatim}
\end{tcolorbox}
        
    Correlation tables are a numerical representation of the bivariate
relationships in the dataset.

Positive numbers indicate a positive correlation --- one goes up the
other goes up --- and negative numbers represent an inverse correlation
--- one goes up the other goes down. 1.0 indicates a perfect
correlation.

So looking in the first row, first column we see \texttt{rank} has a
perfect correlation with itself, which is obvious. On the other hand,
the correlation between \texttt{votes} and \texttt{revenue\_millions} is
0.6. A little more interesting.

Examining bivariate relationships comes in handy when you have an
outcome or dependent variable in mind and would like to see the features
most correlated to the increase or decrease of the outcome. You can
visually represent bivariate relationships with scatterplots (seen below
in the plotting section).

For a deeper look into data summarizations check out
\href{https://www.learndatasci.com/tutorials/data-science-statistics-using-python/}{Essential
Statistics for Data Science}.

Let's now look more at manipulating DataFrames.

    \hypertarget{dataframe-slicing-selecting-extracting}{%
\subsubsection{DataFrame slicing, selecting,
extracting}\label{dataframe-slicing-selecting-extracting}}

Up until now we've focused on some basic summaries of our data. We've
learned about simple column extraction using single brackets, and we
imputed null values in a column using \texttt{fillna()}. Below are the
other methods of slicing, selecting, and extracting you'll need to use
constantly.

It's important to note that, although many methods are the same,
DataFrames and Series have different attributes, so you'll need be sure
to know which type you are working with or else you will receive
attribute errors.

Let's look at working with columns first.

    \hypertarget{by-column}{%
\paragraph{By column}\label{by-column}}

You already saw how to extract a column using square brackets like this:

    \begin{tcolorbox}[breakable, size=fbox, boxrule=1pt, pad at break*=1mm,colback=cellbackground, colframe=cellborder]
\prompt{In}{incolor}{38}{\boxspacing}
\begin{Verbatim}[commandchars=\\\{\}]
\PY{n}{genre\PYZus{}col} \PY{o}{=} \PY{n}{movies\PYZus{}df}\PY{p}{[}\PY{l+s+s1}{\PYZsq{}}\PY{l+s+s1}{genre}\PY{l+s+s1}{\PYZsq{}}\PY{p}{]}

\PY{n+nb}{type}\PY{p}{(}\PY{n}{genre\PYZus{}col}\PY{p}{)}
\end{Verbatim}
\end{tcolorbox}

            \begin{tcolorbox}[breakable, size=fbox, boxrule=.5pt, pad at break*=1mm, opacityfill=0]
\prompt{Out}{outcolor}{38}{\boxspacing}
\begin{Verbatim}[commandchars=\\\{\}]
pandas.core.series.Series
\end{Verbatim}
\end{tcolorbox}
        
    This will return a \emph{Series}. To extract a column as a
\emph{DataFrame}, you need to pass a list of column names. In our case
that's just a single column:

    \begin{tcolorbox}[breakable, size=fbox, boxrule=1pt, pad at break*=1mm,colback=cellbackground, colframe=cellborder]
\prompt{In}{incolor}{39}{\boxspacing}
\begin{Verbatim}[commandchars=\\\{\}]
\PY{n}{genre\PYZus{}col} \PY{o}{=} \PY{n}{movies\PYZus{}df}\PY{p}{[}\PY{p}{[}\PY{l+s+s1}{\PYZsq{}}\PY{l+s+s1}{genre}\PY{l+s+s1}{\PYZsq{}}\PY{p}{]}\PY{p}{]}

\PY{n+nb}{type}\PY{p}{(}\PY{n}{genre\PYZus{}col}\PY{p}{)}
\end{Verbatim}
\end{tcolorbox}

            \begin{tcolorbox}[breakable, size=fbox, boxrule=.5pt, pad at break*=1mm, opacityfill=0]
\prompt{Out}{outcolor}{39}{\boxspacing}
\begin{Verbatim}[commandchars=\\\{\}]
pandas.core.frame.DataFrame
\end{Verbatim}
\end{tcolorbox}
        
    Since it's just a list, adding another column name is easy:

    \begin{tcolorbox}[breakable, size=fbox, boxrule=1pt, pad at break*=1mm,colback=cellbackground, colframe=cellborder]
\prompt{In}{incolor}{40}{\boxspacing}
\begin{Verbatim}[commandchars=\\\{\}]
\PY{n}{subset} \PY{o}{=} \PY{n}{movies\PYZus{}df}\PY{p}{[}\PY{p}{[}\PY{l+s+s1}{\PYZsq{}}\PY{l+s+s1}{genre}\PY{l+s+s1}{\PYZsq{}}\PY{p}{,} \PY{l+s+s1}{\PYZsq{}}\PY{l+s+s1}{rating}\PY{l+s+s1}{\PYZsq{}}\PY{p}{]}\PY{p}{]}

\PY{n}{subset}\PY{o}{.}\PY{n}{head}\PY{p}{(}\PY{p}{)}
\end{Verbatim}
\end{tcolorbox}

            \begin{tcolorbox}[breakable, size=fbox, boxrule=.5pt, pad at break*=1mm, opacityfill=0]
\prompt{Out}{outcolor}{40}{\boxspacing}
\begin{Verbatim}[commandchars=\\\{\}]
                                            genre  rating
Title
Guardians of the Galaxy   Action,Adventure,Sci-Fi     8.1
Prometheus               Adventure,Mystery,Sci-Fi     7.0
Split                             Horror,Thriller     7.3
Sing                      Animation,Comedy,Family     7.2
Suicide Squad            Action,Adventure,Fantasy     6.2
\end{Verbatim}
\end{tcolorbox}
        
    Now we'll look at getting data by rows.

    \hypertarget{by-rows}{%
\paragraph{By rows}\label{by-rows}}

    For rows, we have two options:

\begin{itemize}
\tightlist
\item
  \texttt{.loc} - \textbf{loc}ates by name
\item
  \texttt{.iloc}- \textbf{loc}ates by numerical \textbf{i}ndex
\end{itemize}

Remember that we are still indexed by movie Title, so to use
\texttt{.loc} we give it the Title of a movie:

    \begin{tcolorbox}[breakable, size=fbox, boxrule=1pt, pad at break*=1mm,colback=cellbackground, colframe=cellborder]
\prompt{In}{incolor}{41}{\boxspacing}
\begin{Verbatim}[commandchars=\\\{\}]
\PY{n}{prom} \PY{o}{=} \PY{n}{movies\PYZus{}df}\PY{o}{.}\PY{n}{loc}\PY{p}{[}\PY{l+s+s2}{\PYZdq{}}\PY{l+s+s2}{Prometheus}\PY{l+s+s2}{\PYZdq{}}\PY{p}{]}

\PY{n}{prom}
\end{Verbatim}
\end{tcolorbox}

            \begin{tcolorbox}[breakable, size=fbox, boxrule=.5pt, pad at break*=1mm, opacityfill=0]
\prompt{Out}{outcolor}{41}{\boxspacing}
\begin{Verbatim}[commandchars=\\\{\}]
rank                                                                2
genre                                        Adventure,Mystery,Sci-Fi
description         Following clues to the origin of mankind, a te{\ldots}
director                                                 Ridley Scott
actors              Noomi Rapace, Logan Marshall-Green, Michael Fa{\ldots}
year                                                             2012
runtime                                                           124
rating                                                              7
votes                                                          485820
revenue\_millions                                               126.46
metascore                                                          65
Name: Prometheus, dtype: object
\end{Verbatim}
\end{tcolorbox}
        
    On the other hand, with \texttt{iloc} we give it the numerical index of
Prometheus:

    \begin{tcolorbox}[breakable, size=fbox, boxrule=1pt, pad at break*=1mm,colback=cellbackground, colframe=cellborder]
\prompt{In}{incolor}{42}{\boxspacing}
\begin{Verbatim}[commandchars=\\\{\}]
\PY{n}{prom} \PY{o}{=} \PY{n}{movies\PYZus{}df}\PY{o}{.}\PY{n}{iloc}\PY{p}{[}\PY{l+m+mi}{1}\PY{p}{]}
\end{Verbatim}
\end{tcolorbox}

    \texttt{loc} and \texttt{iloc} can be thought of as similar to Python
\texttt{list} slicing. To show this even further, let's select multiple
rows.

How would you do it with a list? In Python, just slice with brackets
like \texttt{example\_list{[}1:4{]}}. It's works the same way in pandas:

    \begin{tcolorbox}[breakable, size=fbox, boxrule=1pt, pad at break*=1mm,colback=cellbackground, colframe=cellborder]
\prompt{In}{incolor}{43}{\boxspacing}
\begin{Verbatim}[commandchars=\\\{\}]
\PY{n}{movie\PYZus{}subset} \PY{o}{=} \PY{n}{movies\PYZus{}df}\PY{o}{.}\PY{n}{loc}\PY{p}{[}\PY{l+s+s1}{\PYZsq{}}\PY{l+s+s1}{Prometheus}\PY{l+s+s1}{\PYZsq{}}\PY{p}{:}\PY{l+s+s1}{\PYZsq{}}\PY{l+s+s1}{Sing}\PY{l+s+s1}{\PYZsq{}}\PY{p}{]}

\PY{n}{movie\PYZus{}subset} \PY{o}{=} \PY{n}{movies\PYZus{}df}\PY{o}{.}\PY{n}{iloc}\PY{p}{[}\PY{l+m+mi}{1}\PY{p}{:}\PY{l+m+mi}{4}\PY{p}{]}

\PY{n}{movie\PYZus{}subset}
\end{Verbatim}
\end{tcolorbox}

            \begin{tcolorbox}[breakable, size=fbox, boxrule=.5pt, pad at break*=1mm, opacityfill=0]
\prompt{Out}{outcolor}{43}{\boxspacing}
\begin{Verbatim}[commandchars=\\\{\}]
            rank                     genre  \textbackslash{}
Title
Prometheus     2  Adventure,Mystery,Sci-Fi
Split          3           Horror,Thriller
Sing           4   Animation,Comedy,Family

                                                  description  \textbackslash{}
Title
Prometheus  Following clues to the origin of mankind, a te{\ldots}
Split       Three girls are kidnapped by a man with a diag{\ldots}
Sing        In a city of humanoid animals, a hustling thea{\ldots}

                        director  \textbackslash{}
Title
Prometheus          Ridley Scott
Split         M. Night Shyamalan
Sing        Christophe Lourdelet

                                                       actors  year  runtime  \textbackslash{}
Title
Prometheus  Noomi Rapace, Logan Marshall-Green, Michael Fa{\ldots}  2012      124
Split       James McAvoy, Anya Taylor-Joy, Haley Lu Richar{\ldots}  2016      117
Sing        Matthew McConaughey,Reese Witherspoon, Seth Ma{\ldots}  2016      108

            rating   votes  revenue\_millions  metascore
Title
Prometheus     7.0  485820            126.46       65.0
Split          7.3  157606            138.12       62.0
Sing           7.2   60545            270.32       59.0
\end{Verbatim}
\end{tcolorbox}
        
    One important distinction between using \texttt{.loc} and \texttt{.iloc}
to select multiple rows is that \texttt{.loc} includes the movie
\emph{Sing} in the result, but when using \texttt{.iloc} we're getting
rows 1:4 but the movie at index 4 (\emph{Suicide Squad}) is not
included.

Slicing with \texttt{.iloc} follows the same rules as slicing with
lists, the object at the index at the end is not included.

\hypertarget{conditional-selections}{%
\paragraph{Conditional selections}\label{conditional-selections}}

We've gone over how to select columns and rows, but what if we want to
make a conditional selection?

For example, what if we want to filter our movies DataFrame to show only
films directed by Ridley Scott or films with a rating greater than or
equal to 8.0?

To do that, we take a column from the DataFrame and apply a Boolean
condition to it. Here's an example of a Boolean condition:

    \begin{tcolorbox}[breakable, size=fbox, boxrule=1pt, pad at break*=1mm,colback=cellbackground, colframe=cellborder]
\prompt{In}{incolor}{44}{\boxspacing}
\begin{Verbatim}[commandchars=\\\{\}]
\PY{n}{condition} \PY{o}{=} \PY{p}{(}\PY{n}{movies\PYZus{}df}\PY{p}{[}\PY{l+s+s1}{\PYZsq{}}\PY{l+s+s1}{director}\PY{l+s+s1}{\PYZsq{}}\PY{p}{]} \PY{o}{==} \PY{l+s+s2}{\PYZdq{}}\PY{l+s+s2}{Ridley Scott}\PY{l+s+s2}{\PYZdq{}}\PY{p}{)}

\PY{n}{condition}\PY{o}{.}\PY{n}{head}\PY{p}{(}\PY{p}{)}
\end{Verbatim}
\end{tcolorbox}

            \begin{tcolorbox}[breakable, size=fbox, boxrule=.5pt, pad at break*=1mm, opacityfill=0]
\prompt{Out}{outcolor}{44}{\boxspacing}
\begin{Verbatim}[commandchars=\\\{\}]
Title
Guardians of the Galaxy    False
Prometheus                  True
Split                      False
Sing                       False
Suicide Squad              False
Name: director, dtype: bool
\end{Verbatim}
\end{tcolorbox}
        
    Similar to \texttt{isnull()}, this returns a Series of True and False
values: True for films directed by Ridley Scott and False for ones not
directed by him.

We want to filter out all movies not directed by Ridley Scott, in other
words, we don't want the False films. To return the rows where that
condition is True we have to pass this operation into the DataFrame:

    \begin{tcolorbox}[breakable, size=fbox, boxrule=1pt, pad at break*=1mm,colback=cellbackground, colframe=cellborder]
\prompt{In}{incolor}{45}{\boxspacing}
\begin{Verbatim}[commandchars=\\\{\}]
\PY{n}{movies\PYZus{}df}\PY{p}{[}\PY{n}{movies\PYZus{}df}\PY{p}{[}\PY{l+s+s1}{\PYZsq{}}\PY{l+s+s1}{director}\PY{l+s+s1}{\PYZsq{}}\PY{p}{]} \PY{o}{==} \PY{l+s+s2}{\PYZdq{}}\PY{l+s+s2}{Ridley Scott}\PY{l+s+s2}{\PYZdq{}}\PY{p}{]}\PY{o}{.}\PY{n}{head}\PY{p}{(}\PY{p}{)}
\end{Verbatim}
\end{tcolorbox}

            \begin{tcolorbox}[breakable, size=fbox, boxrule=.5pt, pad at break*=1mm, opacityfill=0]
\prompt{Out}{outcolor}{45}{\boxspacing}
\begin{Verbatim}[commandchars=\\\{\}]
                        rank                     genre  \textbackslash{}
Title
Prometheus                 2  Adventure,Mystery,Sci-Fi
The Martian              103    Adventure,Drama,Sci-Fi
Robin Hood               388    Action,Adventure,Drama
American Gangster        471     Biography,Crime,Drama
Exodus: Gods and Kings   517    Action,Adventure,Drama

                                                              description  \textbackslash{}
Title
Prometheus              Following clues to the origin of mankind, a te{\ldots}
The Martian             An astronaut becomes stranded on Mars after hi{\ldots}
Robin Hood              In 12th century England, Robin and his band of{\ldots}
American Gangster       In 1970s America, a detective works to bring d{\ldots}
Exodus: Gods and Kings  The defiant leader Moses rises up against the {\ldots}

                            director  \textbackslash{}
Title
Prometheus              Ridley Scott
The Martian             Ridley Scott
Robin Hood              Ridley Scott
American Gangster       Ridley Scott
Exodus: Gods and Kings  Ridley Scott

                                                                   actors  \textbackslash{}
Title
Prometheus              Noomi Rapace, Logan Marshall-Green, Michael Fa{\ldots}
The Martian             Matt Damon, Jessica Chastain, Kristen Wiig, Ka{\ldots}
Robin Hood              Russell Crowe, Cate Blanchett, Matthew Macfady{\ldots}
American Gangster       Denzel Washington, Russell Crowe, Chiwetel Eji{\ldots}
Exodus: Gods and Kings  Christian Bale, Joel Edgerton, Ben Kingsley, S{\ldots}

                        year  runtime  rating   votes  revenue\_millions  \textbackslash{}
Title
Prometheus              2012      124     7.0  485820            126.46
The Martian             2015      144     8.0  556097            228.43
Robin Hood              2010      140     6.7  221117            105.22
American Gangster       2007      157     7.8  337835            130.13
Exodus: Gods and Kings  2014      150     6.0  137299             65.01

                        metascore
Title
Prometheus                   65.0
The Martian                  80.0
Robin Hood                   53.0
American Gangster            76.0
Exodus: Gods and Kings       52.0
\end{Verbatim}
\end{tcolorbox}
        
    You can get used to looking at these conditionals by reading it like:

\begin{quote}
Select movies\_df where movies\_df director equals Ridley Scott
\end{quote}

Let's look at conditional selections using numerical values by filtering
the DataFrame by ratings:

    \begin{tcolorbox}[breakable, size=fbox, boxrule=1pt, pad at break*=1mm,colback=cellbackground, colframe=cellborder]
\prompt{In}{incolor}{46}{\boxspacing}
\begin{Verbatim}[commandchars=\\\{\}]
\PY{n}{movies\PYZus{}df}\PY{p}{[}\PY{n}{movies\PYZus{}df}\PY{p}{[}\PY{l+s+s1}{\PYZsq{}}\PY{l+s+s1}{rating}\PY{l+s+s1}{\PYZsq{}}\PY{p}{]} \PY{o}{\PYZgt{}}\PY{o}{=} \PY{l+m+mf}{8.6}\PY{p}{]}\PY{o}{.}\PY{n}{head}\PY{p}{(}\PY{l+m+mi}{3}\PY{p}{)}
\end{Verbatim}
\end{tcolorbox}

            \begin{tcolorbox}[breakable, size=fbox, boxrule=.5pt, pad at break*=1mm, opacityfill=0]
\prompt{Out}{outcolor}{46}{\boxspacing}
\begin{Verbatim}[commandchars=\\\{\}]
                 rank                    genre  \textbackslash{}
Title
Interstellar       37   Adventure,Drama,Sci-Fi
The Dark Knight    55       Action,Crime,Drama
Inception          81  Action,Adventure,Sci-Fi

                                                       description  \textbackslash{}
Title
Interstellar     A team of explorers travel through a wormhole {\ldots}
The Dark Knight  When the menace known as the Joker wreaks havo{\ldots}
Inception        A thief, who steals corporate secrets through {\ldots}

                          director  \textbackslash{}
Title
Interstellar     Christopher Nolan
The Dark Knight  Christopher Nolan
Inception        Christopher Nolan

                                                            actors  year  \textbackslash{}
Title
Interstellar     Matthew McConaughey, Anne Hathaway, Jessica Ch{\ldots}  2014
The Dark Knight  Christian Bale, Heath Ledger, Aaron Eckhart,Mi{\ldots}  2008
Inception        Leonardo DiCaprio, Joseph Gordon-Levitt, Ellen{\ldots}  2010

                 runtime  rating    votes  revenue\_millions  metascore
Title
Interstellar         169     8.6  1047747            187.99       74.0
The Dark Knight      152     9.0  1791916            533.32       82.0
Inception            148     8.8  1583625            292.57       74.0
\end{Verbatim}
\end{tcolorbox}
        
    We can make some richer conditionals by using logical operators
\texttt{\textbar{}} for ``or'' and \texttt{\&} for ``and''.

Let's filter the the DataFrame to show only movies by Christopher Nolan
OR Ridley Scott:

    \begin{tcolorbox}[breakable, size=fbox, boxrule=1pt, pad at break*=1mm,colback=cellbackground, colframe=cellborder]
\prompt{In}{incolor}{47}{\boxspacing}
\begin{Verbatim}[commandchars=\\\{\}]
\PY{n}{movies\PYZus{}df}\PY{p}{[}\PY{p}{(}\PY{n}{movies\PYZus{}df}\PY{p}{[}\PY{l+s+s1}{\PYZsq{}}\PY{l+s+s1}{director}\PY{l+s+s1}{\PYZsq{}}\PY{p}{]} \PY{o}{==} \PY{l+s+s1}{\PYZsq{}}\PY{l+s+s1}{Christopher Nolan}\PY{l+s+s1}{\PYZsq{}}\PY{p}{)} \PY{o}{|} \PY{p}{(}\PY{n}{movies\PYZus{}df}\PY{p}{[}\PY{l+s+s1}{\PYZsq{}}\PY{l+s+s1}{director}\PY{l+s+s1}{\PYZsq{}}\PY{p}{]} \PY{o}{==} \PY{l+s+s1}{\PYZsq{}}\PY{l+s+s1}{Ridley Scott}\PY{l+s+s1}{\PYZsq{}}\PY{p}{)}\PY{p}{]}\PY{o}{.}\PY{n}{head}\PY{p}{(}\PY{p}{)}
\end{Verbatim}
\end{tcolorbox}

            \begin{tcolorbox}[breakable, size=fbox, boxrule=.5pt, pad at break*=1mm, opacityfill=0]
\prompt{Out}{outcolor}{47}{\boxspacing}
\begin{Verbatim}[commandchars=\\\{\}]
                 rank                     genre  \textbackslash{}
Title
Prometheus          2  Adventure,Mystery,Sci-Fi
Interstellar       37    Adventure,Drama,Sci-Fi
The Dark Knight    55        Action,Crime,Drama
The Prestige       65      Drama,Mystery,Sci-Fi
Inception          81   Action,Adventure,Sci-Fi

                                                       description  \textbackslash{}
Title
Prometheus       Following clues to the origin of mankind, a te{\ldots}
Interstellar     A team of explorers travel through a wormhole {\ldots}
The Dark Knight  When the menace known as the Joker wreaks havo{\ldots}
The Prestige     Two stage magicians engage in competitive one-{\ldots}
Inception        A thief, who steals corporate secrets through {\ldots}

                          director  \textbackslash{}
Title
Prometheus            Ridley Scott
Interstellar     Christopher Nolan
The Dark Knight  Christopher Nolan
The Prestige     Christopher Nolan
Inception        Christopher Nolan

                                                            actors  year  \textbackslash{}
Title
Prometheus       Noomi Rapace, Logan Marshall-Green, Michael Fa{\ldots}  2012
Interstellar     Matthew McConaughey, Anne Hathaway, Jessica Ch{\ldots}  2014
The Dark Knight  Christian Bale, Heath Ledger, Aaron Eckhart,Mi{\ldots}  2008
The Prestige     Christian Bale, Hugh Jackman, Scarlett Johanss{\ldots}  2006
Inception        Leonardo DiCaprio, Joseph Gordon-Levitt, Ellen{\ldots}  2010

                 runtime  rating    votes  revenue\_millions  metascore
Title
Prometheus           124     7.0   485820            126.46       65.0
Interstellar         169     8.6  1047747            187.99       74.0
The Dark Knight      152     9.0  1791916            533.32       82.0
The Prestige         130     8.5   913152             53.08       66.0
Inception            148     8.8  1583625            292.57       74.0
\end{Verbatim}
\end{tcolorbox}
        
    We need to make sure to group evaluations with parentheses so Python
knows how to evaluate the conditional.

Using the \texttt{isin()} method we could make this more concise though:

    \begin{tcolorbox}[breakable, size=fbox, boxrule=1pt, pad at break*=1mm,colback=cellbackground, colframe=cellborder]
\prompt{In}{incolor}{48}{\boxspacing}
\begin{Verbatim}[commandchars=\\\{\}]
\PY{n}{movies\PYZus{}df}\PY{p}{[}\PY{n}{movies\PYZus{}df}\PY{p}{[}\PY{l+s+s1}{\PYZsq{}}\PY{l+s+s1}{director}\PY{l+s+s1}{\PYZsq{}}\PY{p}{]}\PY{o}{.}\PY{n}{isin}\PY{p}{(}\PY{p}{[}\PY{l+s+s1}{\PYZsq{}}\PY{l+s+s1}{Christopher Nolan}\PY{l+s+s1}{\PYZsq{}}\PY{p}{,} \PY{l+s+s1}{\PYZsq{}}\PY{l+s+s1}{Ridley Scott}\PY{l+s+s1}{\PYZsq{}}\PY{p}{]}\PY{p}{)}\PY{p}{]}\PY{o}{.}\PY{n}{head}\PY{p}{(}\PY{p}{)}
\end{Verbatim}
\end{tcolorbox}

            \begin{tcolorbox}[breakable, size=fbox, boxrule=.5pt, pad at break*=1mm, opacityfill=0]
\prompt{Out}{outcolor}{48}{\boxspacing}
\begin{Verbatim}[commandchars=\\\{\}]
                 rank                     genre  \textbackslash{}
Title
Prometheus          2  Adventure,Mystery,Sci-Fi
Interstellar       37    Adventure,Drama,Sci-Fi
The Dark Knight    55        Action,Crime,Drama
The Prestige       65      Drama,Mystery,Sci-Fi
Inception          81   Action,Adventure,Sci-Fi

                                                       description  \textbackslash{}
Title
Prometheus       Following clues to the origin of mankind, a te{\ldots}
Interstellar     A team of explorers travel through a wormhole {\ldots}
The Dark Knight  When the menace known as the Joker wreaks havo{\ldots}
The Prestige     Two stage magicians engage in competitive one-{\ldots}
Inception        A thief, who steals corporate secrets through {\ldots}

                          director  \textbackslash{}
Title
Prometheus            Ridley Scott
Interstellar     Christopher Nolan
The Dark Knight  Christopher Nolan
The Prestige     Christopher Nolan
Inception        Christopher Nolan

                                                            actors  year  \textbackslash{}
Title
Prometheus       Noomi Rapace, Logan Marshall-Green, Michael Fa{\ldots}  2012
Interstellar     Matthew McConaughey, Anne Hathaway, Jessica Ch{\ldots}  2014
The Dark Knight  Christian Bale, Heath Ledger, Aaron Eckhart,Mi{\ldots}  2008
The Prestige     Christian Bale, Hugh Jackman, Scarlett Johanss{\ldots}  2006
Inception        Leonardo DiCaprio, Joseph Gordon-Levitt, Ellen{\ldots}  2010

                 runtime  rating    votes  revenue\_millions  metascore
Title
Prometheus           124     7.0   485820            126.46       65.0
Interstellar         169     8.6  1047747            187.99       74.0
The Dark Knight      152     9.0  1791916            533.32       82.0
The Prestige         130     8.5   913152             53.08       66.0
Inception            148     8.8  1583625            292.57       74.0
\end{Verbatim}
\end{tcolorbox}
        
    Let's say we want all movies that were released between 2005 and 2010,
have a rating above 8.0, but made below the 25th percentile in revenue.

Here's how we could do all of that:

    \begin{tcolorbox}[breakable, size=fbox, boxrule=1pt, pad at break*=1mm,colback=cellbackground, colframe=cellborder]
\prompt{In}{incolor}{49}{\boxspacing}
\begin{Verbatim}[commandchars=\\\{\}]
\PY{n}{movies\PYZus{}df}\PY{p}{[}
    \PY{p}{(}\PY{p}{(}\PY{n}{movies\PYZus{}df}\PY{p}{[}\PY{l+s+s1}{\PYZsq{}}\PY{l+s+s1}{year}\PY{l+s+s1}{\PYZsq{}}\PY{p}{]} \PY{o}{\PYZgt{}}\PY{o}{=} \PY{l+m+mi}{2005}\PY{p}{)} \PY{o}{\PYZam{}} \PY{p}{(}\PY{n}{movies\PYZus{}df}\PY{p}{[}\PY{l+s+s1}{\PYZsq{}}\PY{l+s+s1}{year}\PY{l+s+s1}{\PYZsq{}}\PY{p}{]} \PY{o}{\PYZlt{}}\PY{o}{=} \PY{l+m+mi}{2010}\PY{p}{)}\PY{p}{)}
    \PY{o}{\PYZam{}} \PY{p}{(}\PY{n}{movies\PYZus{}df}\PY{p}{[}\PY{l+s+s1}{\PYZsq{}}\PY{l+s+s1}{rating}\PY{l+s+s1}{\PYZsq{}}\PY{p}{]} \PY{o}{\PYZgt{}} \PY{l+m+mf}{8.0}\PY{p}{)}
    \PY{o}{\PYZam{}} \PY{p}{(}\PY{n}{movies\PYZus{}df}\PY{p}{[}\PY{l+s+s1}{\PYZsq{}}\PY{l+s+s1}{revenue\PYZus{}millions}\PY{l+s+s1}{\PYZsq{}}\PY{p}{]} \PY{o}{\PYZlt{}} \PY{n}{movies\PYZus{}df}\PY{p}{[}\PY{l+s+s1}{\PYZsq{}}\PY{l+s+s1}{revenue\PYZus{}millions}\PY{l+s+s1}{\PYZsq{}}\PY{p}{]}\PY{o}{.}\PY{n}{quantile}\PY{p}{(}\PY{l+m+mf}{0.25}\PY{p}{)}\PY{p}{)}
\PY{p}{]}
\end{Verbatim}
\end{tcolorbox}

            \begin{tcolorbox}[breakable, size=fbox, boxrule=.5pt, pad at break*=1mm, opacityfill=0]
\prompt{Out}{outcolor}{49}{\boxspacing}
\begin{Verbatim}[commandchars=\\\{\}]
                     rank               genre  \textbackslash{}
Title
3 Idiots              431        Comedy,Drama
The Lives of Others   477      Drama,Thriller
Incendies             714   Drama,Mystery,War
Taare Zameen Par      992  Drama,Family,Music

                                                           description  \textbackslash{}
Title
3 Idiots             Two friends are searching for their long lost {\ldots}
The Lives of Others  In 1984 East Berlin, an agent of the secret po{\ldots}
Incendies            Twins journey to the Middle East to discover t{\ldots}
Taare Zameen Par     An eight-year-old boy is thought to be a lazy {\ldots}

                                             director  \textbackslash{}
Title
3 Idiots                              Rajkumar Hirani
The Lives of Others  Florian Henckel von Donnersmarck
Incendies                            Denis Villeneuve
Taare Zameen Par                           Aamir Khan

                                                                actors  year  \textbackslash{}
Title
3 Idiots               Aamir Khan, Madhavan, Mona Singh, Sharman Joshi  2009
The Lives of Others  Ulrich Mühe, Martina Gedeck,Sebastian Koch, Ul{\ldots}  2006
Incendies            Lubna Azabal, Mélissa Désormeaux-Poulin, Maxim{\ldots}  2010
Taare Zameen Par     Darsheel Safary, Aamir Khan, Tanay Chheda, Sac{\ldots}  2007

                     runtime  rating   votes  revenue\_millions  metascore
Title
3 Idiots                 170     8.4  238789              6.52       67.0
The Lives of Others      137     8.5  278103             11.28       89.0
Incendies                131     8.2   92863              6.86       80.0
Taare Zameen Par         165     8.5  102697              1.20       42.0
\end{Verbatim}
\end{tcolorbox}
        
    If you recall up when we used \texttt{.describe()} the 25th percentile
for revenue was about 17.4, and we can access this value directly by
using the \texttt{quantile()} method with a float of 0.25.

So here we have only four movies that match that criteria.

    \hypertarget{applying-functions}{%
\subsubsection{Applying functions}\label{applying-functions}}

It is possible to iterate over a DataFrame or Series as you would with a
list, but doing so --- especially on large datasets --- is very slow.

An efficient alternative is to \texttt{apply()} a function to the
dataset. For example, we could use a function to convert movies with an
8.0 or greater to a string value of ``good'' and the rest to ``bad'' and
use this transformed values to create a new column.

First we would create a function that, when given a rating, determines
if it's good or bad:

    \begin{tcolorbox}[breakable, size=fbox, boxrule=1pt, pad at break*=1mm,colback=cellbackground, colframe=cellborder]
\prompt{In}{incolor}{9}{\boxspacing}
\begin{Verbatim}[commandchars=\\\{\}]
\PY{k}{def} \PY{n+nf}{rating\PYZus{}function}\PY{p}{(}\PY{n}{x}\PY{p}{)}\PY{p}{:}
    \PY{k}{if} \PY{n}{x} \PY{o}{\PYZgt{}}\PY{o}{=} \PY{l+m+mf}{8.0}\PY{p}{:}
        \PY{k}{return} \PY{l+s+s2}{\PYZdq{}}\PY{l+s+s2}{good}\PY{l+s+s2}{\PYZdq{}}
    \PY{k}{else}\PY{p}{:}
        \PY{k}{return} \PY{l+s+s2}{\PYZdq{}}\PY{l+s+s2}{bad}\PY{l+s+s2}{\PYZdq{}}
\end{Verbatim}
\end{tcolorbox}

    Now we want to send the entire rating column through this function,
which is what \texttt{apply()} does:

    \begin{tcolorbox}[breakable, size=fbox, boxrule=1pt, pad at break*=1mm,colback=cellbackground, colframe=cellborder]
\prompt{In}{incolor}{10}{\boxspacing}
\begin{Verbatim}[commandchars=\\\{\}]
\PY{n}{movies\PYZus{}df}\PY{p}{[}\PY{l+s+s2}{\PYZdq{}}\PY{l+s+s2}{rating\PYZus{}category}\PY{l+s+s2}{\PYZdq{}}\PY{p}{]} \PY{o}{=} \PY{n}{movies\PYZus{}df}\PY{p}{[}\PY{l+s+s2}{\PYZdq{}}\PY{l+s+s2}{rating}\PY{l+s+s2}{\PYZdq{}}\PY{p}{]}\PY{o}{.}\PY{n}{apply}\PY{p}{(}\PY{n}{rating\PYZus{}function}\PY{p}{)}

\PY{n}{movies\PYZus{}df}\PY{o}{.}\PY{n}{head}\PY{p}{(}\PY{l+m+mi}{2}\PY{p}{)}
\end{Verbatim}
\end{tcolorbox}

    \begin{Verbatim}[commandchars=\\\{\}, frame=single, framerule=2mm, rulecolor=\color{outerrorbackground}]
\textcolor{ansi-red-intense}{\textbf{---------------------------------------------------------------------------}}
\textcolor{ansi-red-intense}{\textbf{NameError}}                                 Traceback (most recent call last)
\textcolor{ansi-green-intense}{\textbf{<ipython-input-10-14645f4710d4>}} in \textcolor{ansi-cyan}{<module>}
\textcolor{ansi-green-intense}{\textbf{----> 1}}\textcolor{ansi-yellow-intense}{\textbf{ }}movies\_df\textcolor{ansi-yellow-intense}{\textbf{[}}\textcolor{ansi-blue-intense}{\textbf{"rating\_category"}}\textcolor{ansi-yellow-intense}{\textbf{]}} \textcolor{ansi-yellow-intense}{\textbf{=}} movies\_df\textcolor{ansi-yellow-intense}{\textbf{[}}\textcolor{ansi-blue-intense}{\textbf{"rating"}}\textcolor{ansi-yellow-intense}{\textbf{]}}\textcolor{ansi-yellow-intense}{\textbf{.}}apply\textcolor{ansi-yellow-intense}{\textbf{(}}rating\_function\textcolor{ansi-yellow-intense}{\textbf{)}}
\textcolor{ansi-green}{      2} 
\textcolor{ansi-green}{      3} movies\_df\textcolor{ansi-yellow-intense}{\textbf{.}}head\textcolor{ansi-yellow-intense}{\textbf{(}}\textcolor{ansi-cyan-intense}{\textbf{2}}\textcolor{ansi-yellow-intense}{\textbf{)}}

\textcolor{ansi-red-intense}{\textbf{NameError}}: name 'movies\_df' is not defined
    \end{Verbatim}

    The \texttt{.apply()} method passes every value in the \texttt{rating}
column through the \texttt{rating\_function} and then returns a new
Series. This Series is then assigned to a new column called
\texttt{rating\_category}.

You can also use anonymous functions as well. This lambda function
achieves the same result as \texttt{rating\_function}:

    \begin{tcolorbox}[breakable, size=fbox, boxrule=1pt, pad at break*=1mm,colback=cellbackground, colframe=cellborder]
\prompt{In}{incolor}{11}{\boxspacing}
\begin{Verbatim}[commandchars=\\\{\}]
\PY{n}{movies\PYZus{}df}\PY{p}{[}\PY{l+s+s2}{\PYZdq{}}\PY{l+s+s2}{rating\PYZus{}category}\PY{l+s+s2}{\PYZdq{}}\PY{p}{]} \PY{o}{=} \PY{n}{movies\PYZus{}df}\PY{p}{[}\PY{l+s+s2}{\PYZdq{}}\PY{l+s+s2}{rating}\PY{l+s+s2}{\PYZdq{}}\PY{p}{]}\PY{o}{.}\PY{n}{apply}\PY{p}{(}\PY{k}{lambda} \PY{n}{x}\PY{p}{:} \PY{l+s+s1}{\PYZsq{}}\PY{l+s+s1}{good}\PY{l+s+s1}{\PYZsq{}} \PY{k}{if} \PY{n}{x} \PY{o}{\PYZgt{}}\PY{o}{=} \PY{l+m+mf}{8.0} \PY{k}{else} \PY{l+s+s1}{\PYZsq{}}\PY{l+s+s1}{bad}\PY{l+s+s1}{\PYZsq{}}\PY{p}{)}

\PY{n}{movies\PYZus{}df}\PY{o}{.}\PY{n}{head}\PY{p}{(}\PY{l+m+mi}{2}\PY{p}{)}
\end{Verbatim}
\end{tcolorbox}

    \begin{Verbatim}[commandchars=\\\{\}, frame=single, framerule=2mm, rulecolor=\color{outerrorbackground}]
\textcolor{ansi-red-intense}{\textbf{---------------------------------------------------------------------------}}
\textcolor{ansi-red-intense}{\textbf{NameError}}                                 Traceback (most recent call last)
\textcolor{ansi-green-intense}{\textbf{<ipython-input-11-2e503968b610>}} in \textcolor{ansi-cyan}{<module>}
\textcolor{ansi-green-intense}{\textbf{----> 1}}\textcolor{ansi-yellow-intense}{\textbf{ }}movies\_df\textcolor{ansi-yellow-intense}{\textbf{[}}\textcolor{ansi-blue-intense}{\textbf{"rating\_category"}}\textcolor{ansi-yellow-intense}{\textbf{]}} \textcolor{ansi-yellow-intense}{\textbf{=}} movies\_df\textcolor{ansi-yellow-intense}{\textbf{[}}\textcolor{ansi-blue-intense}{\textbf{"rating"}}\textcolor{ansi-yellow-intense}{\textbf{]}}\textcolor{ansi-yellow-intense}{\textbf{.}}apply\textcolor{ansi-yellow-intense}{\textbf{(}}\textcolor{ansi-green-intense}{\textbf{lambda}} x\textcolor{ansi-yellow-intense}{\textbf{:}} \textcolor{ansi-blue-intense}{\textbf{'good'}} \textcolor{ansi-green-intense}{\textbf{if}} x \textcolor{ansi-yellow-intense}{\textbf{>=}} \textcolor{ansi-cyan-intense}{\textbf{8.0}} \textcolor{ansi-green-intense}{\textbf{else}} \textcolor{ansi-blue-intense}{\textbf{'bad'}}\textcolor{ansi-yellow-intense}{\textbf{)}}
\textcolor{ansi-green}{      2} 
\textcolor{ansi-green}{      3} movies\_df\textcolor{ansi-yellow-intense}{\textbf{.}}head\textcolor{ansi-yellow-intense}{\textbf{(}}\textcolor{ansi-cyan-intense}{\textbf{2}}\textcolor{ansi-yellow-intense}{\textbf{)}}

\textcolor{ansi-red-intense}{\textbf{NameError}}: name 'movies\_df' is not defined
    \end{Verbatim}

    Overall, using \texttt{apply()} will be much faster than iterating
manually over rows because pandas is utilizing vectorization.

\begin{quote}
Vectorization: a style of computer programming where operations are
applied to whole arrays instead of individual elements
---\href{https://en.wikipedia.org/wiki/Vectorization}{Wikipedia}
\end{quote}

A good example of high usage of \texttt{apply()} is during natural
language processing (NLP) work. You'll need to apply all sorts of text
cleaning functions to strings to prepare for machine learning.

    \hypertarget{brief-plotting}{%
\subsubsection{Brief Plotting}\label{brief-plotting}}

Another great thing about pandas is that it integrates with Matplotlib,
so you get the ability to plot directly off DataFrames and Series. To
get started we need to import Matplotlib
(\texttt{pip\ install\ matplotlib}):

    \begin{tcolorbox}[breakable, size=fbox, boxrule=1pt, pad at break*=1mm,colback=cellbackground, colframe=cellborder]
\prompt{In}{incolor}{53}{\boxspacing}
\begin{Verbatim}[commandchars=\\\{\}]
\PY{k+kn}{import} \PY{n+nn}{matplotlib}\PY{n+nn}{.}\PY{n+nn}{pyplot} \PY{k}{as} \PY{n+nn}{plt}
\PY{n}{plt}\PY{o}{.}\PY{n}{rcParams}\PY{o}{.}\PY{n}{update}\PY{p}{(}\PY{p}{\PYZob{}}\PY{l+s+s1}{\PYZsq{}}\PY{l+s+s1}{font.size}\PY{l+s+s1}{\PYZsq{}}\PY{p}{:} \PY{l+m+mi}{20}\PY{p}{,} \PY{l+s+s1}{\PYZsq{}}\PY{l+s+s1}{figure.figsize}\PY{l+s+s1}{\PYZsq{}}\PY{p}{:} \PY{p}{(}\PY{l+m+mi}{10}\PY{p}{,} \PY{l+m+mi}{8}\PY{p}{)}\PY{p}{\PYZcb{}}\PY{p}{)} \PY{c+c1}{\PYZsh{} set font and plot size to be larger}
\end{Verbatim}
\end{tcolorbox}

    Now we can begin. There won't be a lot of coverage on plotting, but it
should be enough to explore you're data easily.

\textbf{Side note:} For categorical variables utilize Bar Charts* and
Boxplots. For continuous variables utilize Histograms, Scatterplots,
Line graphs, and Boxplots.

Let's plot the relationship between ratings and revenue. All we need to
do is call \texttt{.plot()} on \texttt{movies\_df} with some info about
how to construct the plot:

    \begin{tcolorbox}[breakable, size=fbox, boxrule=1pt, pad at break*=1mm,colback=cellbackground, colframe=cellborder]
\prompt{In}{incolor}{54}{\boxspacing}
\begin{Verbatim}[commandchars=\\\{\}]
\PY{n}{movies\PYZus{}df}\PY{o}{.}\PY{n}{plot}\PY{p}{(}\PY{n}{kind}\PY{o}{=}\PY{l+s+s1}{\PYZsq{}}\PY{l+s+s1}{scatter}\PY{l+s+s1}{\PYZsq{}}\PY{p}{,} \PY{n}{x}\PY{o}{=}\PY{l+s+s1}{\PYZsq{}}\PY{l+s+s1}{rating}\PY{l+s+s1}{\PYZsq{}}\PY{p}{,} \PY{n}{y}\PY{o}{=}\PY{l+s+s1}{\PYZsq{}}\PY{l+s+s1}{revenue\PYZus{}millions}\PY{l+s+s1}{\PYZsq{}}\PY{p}{,} \PY{n}{title}\PY{o}{=}\PY{l+s+s1}{\PYZsq{}}\PY{l+s+s1}{Revenue (millions) vs Rating}\PY{l+s+s1}{\PYZsq{}}\PY{p}{)}\PY{p}{;}
\end{Verbatim}
\end{tcolorbox}

    \begin{center}
    \adjustimage{max size={0.9\linewidth}{0.9\paperheight}}{output_231_0.png}
    \end{center}
    { \hspace*{\fill} \\}
    
    What's with the semicolon? It's not a syntax error, just a way to hide
the
\texttt{\textless{}matplotlib.axes.\_subplots.AxesSubplot\ at\ 0x26613b5cc18\textgreater{}}
output when plotting in Jupyter notebooks.

If we want to plot a simple Histogram based on a single column, we can
call plot on a column:

    \begin{tcolorbox}[breakable, size=fbox, boxrule=1pt, pad at break*=1mm,colback=cellbackground, colframe=cellborder]
\prompt{In}{incolor}{55}{\boxspacing}
\begin{Verbatim}[commandchars=\\\{\}]
\PY{n}{movies\PYZus{}df}\PY{p}{[}\PY{l+s+s1}{\PYZsq{}}\PY{l+s+s1}{rating}\PY{l+s+s1}{\PYZsq{}}\PY{p}{]}\PY{o}{.}\PY{n}{plot}\PY{p}{(}\PY{n}{kind}\PY{o}{=}\PY{l+s+s1}{\PYZsq{}}\PY{l+s+s1}{hist}\PY{l+s+s1}{\PYZsq{}}\PY{p}{,} \PY{n}{title}\PY{o}{=}\PY{l+s+s1}{\PYZsq{}}\PY{l+s+s1}{Rating}\PY{l+s+s1}{\PYZsq{}}\PY{p}{)}\PY{p}{;}
\end{Verbatim}
\end{tcolorbox}

    \begin{center}
    \adjustimage{max size={0.9\linewidth}{0.9\paperheight}}{output_233_0.png}
    \end{center}
    { \hspace*{\fill} \\}
    
    Do you remember the \texttt{.describe()} example at the beginning of
this tutorial? Well, there's a graphical representation of the
interquartile range, called the Boxplot. Let's recall what
\texttt{describe()} gives us on the ratings column:

    \begin{tcolorbox}[breakable, size=fbox, boxrule=1pt, pad at break*=1mm,colback=cellbackground, colframe=cellborder]
\prompt{In}{incolor}{56}{\boxspacing}
\begin{Verbatim}[commandchars=\\\{\}]
\PY{n}{movies\PYZus{}df}\PY{p}{[}\PY{l+s+s1}{\PYZsq{}}\PY{l+s+s1}{rating}\PY{l+s+s1}{\PYZsq{}}\PY{p}{]}\PY{o}{.}\PY{n}{describe}\PY{p}{(}\PY{p}{)}
\end{Verbatim}
\end{tcolorbox}

            \begin{tcolorbox}[breakable, size=fbox, boxrule=.5pt, pad at break*=1mm, opacityfill=0]
\prompt{Out}{outcolor}{56}{\boxspacing}
\begin{Verbatim}[commandchars=\\\{\}]
count    1000.000000
mean        6.723200
std         0.945429
min         1.900000
25\%         6.200000
50\%         6.800000
75\%         7.400000
max         9.000000
Name: rating, dtype: float64
\end{Verbatim}
\end{tcolorbox}
        
    Using a Boxplot we can visualize this data:

    \begin{tcolorbox}[breakable, size=fbox, boxrule=1pt, pad at break*=1mm,colback=cellbackground, colframe=cellborder]
\prompt{In}{incolor}{57}{\boxspacing}
\begin{Verbatim}[commandchars=\\\{\}]
\PY{n}{movies\PYZus{}df}\PY{p}{[}\PY{l+s+s1}{\PYZsq{}}\PY{l+s+s1}{rating}\PY{l+s+s1}{\PYZsq{}}\PY{p}{]}\PY{o}{.}\PY{n}{plot}\PY{p}{(}\PY{n}{kind}\PY{o}{=}\PY{l+s+s2}{\PYZdq{}}\PY{l+s+s2}{box}\PY{l+s+s2}{\PYZdq{}}\PY{p}{)}\PY{p}{;}
\end{Verbatim}
\end{tcolorbox}

    \begin{center}
    \adjustimage{max size={0.9\linewidth}{0.9\paperheight}}{output_237_0.png}
    \end{center}
    { \hspace*{\fill} \\}
    
    Source: \emph{Flowing Data}

By combining categorical and continuous data, we can create a Boxplot of
revenue that is grouped by the Rating Category we created above:

    \begin{tcolorbox}[breakable, size=fbox, boxrule=1pt, pad at break*=1mm,colback=cellbackground, colframe=cellborder]
\prompt{In}{incolor}{58}{\boxspacing}
\begin{Verbatim}[commandchars=\\\{\}]
\PY{n}{movies\PYZus{}df}\PY{o}{.}\PY{n}{boxplot}\PY{p}{(}\PY{n}{column}\PY{o}{=}\PY{l+s+s1}{\PYZsq{}}\PY{l+s+s1}{revenue\PYZus{}millions}\PY{l+s+s1}{\PYZsq{}}\PY{p}{,} \PY{n}{by}\PY{o}{=}\PY{l+s+s1}{\PYZsq{}}\PY{l+s+s1}{rating\PYZus{}category}\PY{l+s+s1}{\PYZsq{}}\PY{p}{)}\PY{p}{;}
\end{Verbatim}
\end{tcolorbox}

    \begin{Verbatim}[commandchars=\\\{\}]
C:\textbackslash{}Users\textbackslash{}User\textbackslash{}Anaconda3\textbackslash{}lib\textbackslash{}site-packages\textbackslash{}numpy\textbackslash{}core\textbackslash{}\_asarray.py:83:
VisibleDeprecationWarning: Creating an ndarray from ragged nested sequences
(which is a list-or-tuple of lists-or-tuples-or ndarrays with different lengths
or shapes) is deprecated. If you meant to do this, you must specify
'dtype=object' when creating the ndarray
  return array(a, dtype, copy=False, order=order)
    \end{Verbatim}

    \begin{center}
    \adjustimage{max size={0.9\linewidth}{0.9\paperheight}}{output_239_1.png}
    \end{center}
    { \hspace*{\fill} \\}
    
    That's the general idea of plotting with pandas. There's too many plots
to mention, so definitely take a look at the \texttt{plot()}
\href{https://pandas.pydata.org/pandas-docs/stable/generated/pandas.DataFrame.plot.html}{docs
here} for more information on what it can do.

    \hypertarget{references-credits}{%
\section{References \& Credits}\label{references-credits}}

    To keep improving, view the
\href{https://pandas.pydata.org/pandas-docs/stable/tutorials.html}{extensive
tutorials} offered by the official pandas docs, follow along with a few
\href{https://www.kaggle.com/kernels}{Kaggle kernels}, and keep working
on your own projects!


    % Add a bibliography block to the postdoc
    
    
    
\end{document}
